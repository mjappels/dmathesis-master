\documentclass[
% thesisdraft,
twoside,
openright,
frontopenright,
]{dmathesis}
\usepackage[utf8]{inputenc}
\usepackage[T1]{fontenc}
\usepackage{lmodern}
\usepackage{lipsum}
\usepackage[british]{babel}
\usepackage[figuresright]{rotating}
\usepackage[draft=false,protrusion=true,expansion,shrink=10,stretch=10,]{microtype}
\makeatletter
\@ifpackageloaded{microtype}{%
	\providecommand{\disableprotrusion}{\microtypesetup{protrusion=false}}
	\providecommand{\enableprotrusion}{\microtypesetup{protrusion=true}}
}{%
	\providecommand{\disableprotrusion}{}
	\providecommand{\enableprotrusion}{}
}
\makeatother

\usepackage{csquotes}
% \usepackage[backend=biber,maxbibnames=10,maxcitenames=4,maxalphanames=4,style=nature,sorting=none,isbn=false]{biblatex}
% % \usepackage{bibstyle-patch}
% \addbibresource{main.bib}
% \appto\bibsetup{\raggedright}

\usepackage{amsmath,amsthm,amssymb}
\usepackage[hidelinks,draft=false]{hyperref}
\usepackage{booktabs}
\usepackage{xfrac}

% The folder in which images are stored for this project.
% If this is enabled, the folder doesn't need to be specified in each
% call to \includegraphics, i.e
%   \includegraphics{picturename}
% rather than
% \includegraphics{img/picturename}
\graphicspath{{figs/}}

% Thmtools sets up theorem-like environments, and modifies the autoref
% command from hyperref to work when these environments share a counter
% Thmtools also defines an Autoref command for capitalising at the start
% of a sentence
\usepackage{thmtools}
\declaretheorem[
style=plain,
name=Theorem,
numberwithin=section
]{thm}
\declaretheorem[
style=plain,
name=Proposition,
numberlike=thm
]{prop}
\declaretheorem[
style=plain,
name=Lemma,
numberlike=thm
]{lem}
\declaretheorem[
style=plain,
name=Corollary,
numberlike=thm
]{cor}
\declaretheorem[
style=definition,
name=Definition,
numberlike=thm
]{mdef}
\declaretheorem[
style=definition,
name=Example,
numberlike=thm
]{example}
\declaretheorem[
style=definition,
name=Remark,
numberlike=thm
]{rem}
\declaretheorem[
style=plain,
name=Conjecture,
numberlike=thm
]{conj}
\declaretheorem[
style=plain,
name=Question,
numberlike=thm
]{question}

% \numberwithin{equation}{section}
% \allowdisplaybreaks

% Makes the last line of a page flush wtih the bottom margin for neatness
\raggedbottom
%\emergencystretch=1em
% \flushbottom

% use these to make paragraphs not indented,
% and to separate consecutive paragraphs
\setlength{\parindent}{12pt}
\setlength{\parskip}{0.5em plus 3pt minus 3pt}

% Provide itemize without extra spacing
\newenvironment{itemize*}%
{\begin{itemize}%
	\setlength{\itemsep}{0pt}%
	\setlength{\parskip}{0pt}}%
{\end{itemize}}
\newenvironment{enumerate*}%
{\begin{enumerate}%
	\setlength{\itemsep}{0pt}%
	\setlength{\parskip}{0pt}}%
{\end{enumerate}}

% Format captions
\usepackage[margin=15mm,labelfont=bf,
% hang
]{caption}
\usepackage[labelfont=normalfont]{subcaption}

\setfloatlocations{figure}{tbp}
\setfloatlocations{table}{tbp}

% In align*, use this to number a particular line
% Rather than using align, and \nonumber-ing every other line
\newcommand\numberthis{\addtocounter{equation}{1}\tag{\theequation}}
% Change autoref names. Generally I want sections to be capitalised at all
% times, not just when starting a sentence.
% (I'm not sure whether all these are necessary nor what the defaults are)
\renewcommand{\equationautorefname}{equation}
\newcommand{\equationAutorefname}{Equation}
\newcommand{\sectionAutorefname}{Section}
\newcommand{\chapterAutorefname}{Chapter}
\newcommand{\subsectionAutorefname}{Subsection}
\newcommand{\subsubsectionAutorefname}{Subsection}
\newcommand{\algorithmAutorefname}{Algorithm}
% Annoyingly these defs need to come *after* \begin{document}, so add to begin
%   document hook.

\AtBeginDocument{%
  \def\sectionautorefname{Section}%
  \def\chapterautorefname{Chapter}%
  \def\subsectionautorefname{Subsection}%
  \def\subsubsectionautorefname{Subsection}%
  \def\algorithmautorefname{Algorithm}%
}

\DeclareMathOperator{\Tr}{Tr}
      

     


%%% Local Variables:
%%% mode: latex
%%% TeX-master: "main"
%%% End:

\usepackage{setspace}
\onehalfspacing

\usepackage{todonotes}
\usepackage{cleveref}
\usepackage[numbers,sort&compress]{natbib}
\usepackage{bibentry}
\nobibliography*

\newcommand{\nn}{\nonumber}
\newcommand{\PD}{Pleba\'nski-Demia\'nski}

\newcommand{\dr}{\mathrm{dr}}
\newcommand{\bd}{\mathrm{bd}}
\newcommand{\ta}{\tilde{a}}
\newcommand{\tm}{\tilde{m}}
\newcommand{\tA}{\tilde{A}}

\newcommand{\todoopt}[2][]{\todo[color=blue!20,size=\footnotesize,#1]{#2}}
\newcommand{\placehold}[2][]{\todo[color=green!20,size=\small,inline,#1]{#2}}
\newcommand{\todoruth}[2][]{\todo[color=magenta!20,size=\footnotesize,#1]{#2}}
\newcommand{\todomike}[2][]{\todo[color=yellow!40,size=\footnotesize,#1]{#2}}

\let\normalcolor\relax
\newcommand{\tcb}{\textcolor{blue}}
\newcommand{\tcr}{\textcolor{red}}
\newcommand{\tcg}{\textcolor{green}}
\newcommand{\tcm}{\textcolor{magenta}}

\begin{document}
\title{Thermodynamics of Accelerating Black Holes}
\subtitle{Three years with the C-metric}
\author{Michael Appels}
\researchgroup{Centre for Particle Theory}
\pagenumbering{roman}
\maketitlepage*

\begin{abstract*}
  %
  We address a long-standing problem of describing the thermodynamics of an
  accelerating black hole. We derive a standard first law of black hole
  thermodynamics, with the usual identification of entropy proportional to the
  area of the event horizon---even though the event horizon contains a conical
  singularity. We show how to generalise this result, formulating thermodynamics
  for black holes with varying conical deficits. We derive a new potential for
  the varying tension defects: the thermodynamic length, both for accelerating
  and static black holes. We discuss possible physical processes in which the
  tension of a string ending on a black hole might vary, and also map out the
  thermodynamic phase space of accelerating black holes and explore their
  critical phenomena. We then revisit the critical limit in which
  asymptotically-AdS black holes develop maximal conical deficits, first for a
  stationary rotating black hole, and then for an accelerated black hole, by
  taking various upper bounds for the parameters in the spacetimes presented. We
  explore the thermodynamics of these geometries and evaluate the reverse
  isoperimetric inequality, and argue that the ultra-spinning black hole only
  violates this condition when it is non-accelerating. Finally, we return to
  some of our earlier findings and adjust them in light of new results; a new
  expression for the mass is obtained by computing the dual stress-energy tensor
  for the spacetime and finding that it corresponds to a relativistic fluid with
  a non-trivial viscous shear tensor. We compare the holographic computation
  with the method of conformal completion showing it yields the same result for
  the mass. This result not only extends the applicability of black hole
  thermodynamics to realms previously not anticipated, it also opens a
  possibility for studying novel properties of an important class of exact
  radiative solutions of Einstein equations describing accelerated objects.
%
\end{abstract*}

\begin{declaration*}
%
  The work in this thesis is based on research carried out in the Department of
  Mathematical Sciences at Durham University. The results presented have in part
  already appeared in the following publications:
  \begin{itemize}
  \item \bibentry{Appels:2016uha}
  \item \bibentry{Appels:2017xoe}
  \item \bibentry{Anabalon:2018ydc}
  \end{itemize}
  Any section reproduced from these articles has been done so with the full
  consent and permission from the relevant collaborators. No part of this thesis
  has been submitted elsewhere for any degree or qualification.
%
\end{declaration*}

\begin{acknowledgements*}
  %
  Completing this PhD has been a long journey, and there are certain people whom
  I must recognise as being vital parts of that time spent. Furthermore, I would
  like to also take this opportunity to recognise people who I count as having
  contributed to this accomplishment either during or prior to my time as a
  postgraduate.

  First and foremost, I would like to extend my deepest thanks to Ruth Gregory
  for her invaluable guidance and supervision throughout the past three
  years. Ruth has given me the freedom to flourish as a physicist while giving
  me sound and most welcome advice in unfamiliar situations. My time as a PhD
  has been thoroughly exciting thanks to her, affording me countless
  opportunities to work alongside and interact with the brightest minds in the
  field, as well as providing me with near-VIP treatment at many international
  airport lounges. 
  
  \begin{itemize}
  \item Ruth
  \item Collaborators, David
  \item Seb Franco, Nigel Glover,
  \item Aristos, Simon and Paul
  \item word about SWH?
  \item family
  \item Office mates, Leo Akash and Omar
  \item Friends back home, Undergrad, Durham
  \item Lara
  \item Administrative and support staff in durham, at PI
  \item STFC
  \end{itemize}
%
\end{acknowledgements*}

\begin{epigraph*}
%
  Done is better than perfect.
%\source{Some book}{A. Author}
%
\end{epigraph*}

%\dedicationtext{This thesis is dedicated to}
\begin{dedication*}
%
  Mum \& Dad,\\ Grandpa Ian \& Granny Liz,\\ Grandpa Bob \& Granny Anne.
%
\end{dedication*}

\disableprotrusion
\tableofcontents*
% \listoffigures
% \listoftables
\listoftodos
\enableprotrusion

\cleardoublepage
\pagenumbering{arabic}



\chapter*{Preface}
\addcontentsline{toc}{chapter}{Preface}

The 1970s brought in a new era for research into black holes by drawing
parallels between the classical theory of thermodynamics and these gravitational
solutions. This newly found application for century old machinery opened many
avenues for this research. Indeed, black holes have been discovered to possess a
rich and diverse catalogue phenomena. Even more recently, this has provided
string theorists with many avenues to explore via the AdS/CFT correspondence.

In this thesis we seek to ascertain the validity of this framework when applied
to exotic solutions discovered around the same time to describe accelerating
black holes. The work was initially motivated by the discovery that rotating
black holes in asymptotically anti-de Sitter space, in a special limit, exhibit
extraordinary thermodynamic behaviour. These solutions are unique in particular
as they possess two severe conical defects at each pole. Conical defects are an
inherent feature to accelerating black holes, hence the desire to investigate
the thermodynamics of accelerating black holes in a similar limit. To do so, we
needed to first develop a rigid framework for thermodynamics of these black
holes.

As we will demonstrate, we have been able to propose a consistent set of
thermodynamic relations and quantities for static non-rotating charged accelerating
black holes. The situation is somewhat more problematic with the inclusion of
rotation and while we are unable to provide a full picture, we are able to form
a sufficient picture to investigate the aforementioned limit.

The layout is as follows. We will begin by reviewing aspects of black hole
thermodynamics in \cref{chap:BHTD}, covering some of the important more
historical discoveries as well as more recent developments that we have used
elsewhere. In \cref{chap:cmet}, we similarly review the C-metric, which
describes a uniformly accelerating black hole, its history as well as certain
derivations leading to the metrics we use further. We also outline certain
constraints that must be taken into consideration when working with these
solutions. In \cref{chap:td-conical}, we develop the thermodynamical framework
for the accelerating black hole, and include the tensions of cosmic strings as
new extensive variables of our ensemble. This introduces a new potential, the
thermodynamic length and finally we also explore the thermodynamic phase
structure of the solution. We attempt to extend this framework in
\cref{chap:crit-bh} to include rotating solutions and investigate the
thermodynamic properties of critical accelerating solutions, which possess the
aforementioned severe conical defects. Finally, in \cref{chap:holoTD}, we
revisit some of our original conclusions and alter some of our results in light
of new calculations that use different techniques.

\section*{Conventions}
\addcontentsline{toc}{section}{Conventions} Throughout this thesis we have used
the mostly plus convention for metric signatures. These metrics are expressed as
the line element $ds^2 = g_{ab}dx^adx^b$ in most cases. Unless stated otherwise,
we are working in 3+1 spacetime dimensions.

Quantities are expressed in Planck units such that
$c = G = 4\pi\varepsilon_0 = k_\mathrm{B} = 1$, where $c$ is the speed of light
in a vacuum, $G$ is Newton's gravitational constant in four dimensions,
$\varepsilon_0$ is the permittivity of free space and $k_\mathrm{B}$ is
Boltzmann's constant, and $\hbar = \ell_p^2$, Planck's constant, is the only
dimensionful fundamental constant. This implies that physical quantities of
length, time, mass and charge have the same dimensions, and temperatures have
dimensions $L^{-1}$.

When a coordinate transformation is used, or any general transformation or
relabelling, it will be represented in either one of the following manners. In
going from the coordinates/parameters $\{x^i\}$ to the coordinates/parameters
$\{y^i\}$, a simple relation will be provided as either
\begin{align*}
  x^i = x^i(\{y^i\})\qquad \mbox{or} \qquad y^i = y^i(\{x^i\}),
\end{align*}
where clarity will be the determining factor. alternatively, it will be more
desirable to preserve some particular characters on either ends of the
transformation. when this is the case, the set of coordinates/parameters
$\{x^i\}$ is to be replaced, as they appear in any expression, with, for
example, a primed set $\{x'^i\}$, for which a relation such as those above
will be provided. The replacement procedure is represented by an arrow $\to$,
the direction of which indicates which variable is being replaced (the
tail). This is then followed by the relation to the original
variable. Explicitly,
\begin{align*}
  x^i \to x'e^i=x'^i(\{x^i\}) \qquad \mbox{or} \qquad x^i \to x'^i(\{x^i\}),
\end{align*}
where we have introduced a commonly used shorthand in the second expression
signifying simply that ``$\to$'' is to be interpreted as meaning
``$\to x'^i =$''.

\chapter{Black Hole Thermodynamics/ Mechanics}
\label{chap:BHTD}

\section{The Laws of Black Hole Mechanics}
\label{sec:fourlaws}

In 1971, Stephen Hawking discovered that there exists an upper bound on the
amount of energy that can be released through gravitational radiation --- or any
other form of energy release, for that matter --- upon the collision of two
black holes~\cite{Hawking:1971tu,Hawking:1971vc}. This result relies primarily
on a proof that through any given physical process, such as a collision or a
capture of sorts, the total event horizon area should never decrease, thereby
constraining how much energy may be extracted. Now known as Hawking's
\emph{black hole area theorem}, the result of this proof is commonly written as:
\begin{align}
  \label{eq:areatheorem}
  \mathcal{A}_3 \geqslant \mathcal{A}_1+\mathcal{A}_2
\end{align}
This theorem has an important role in the history of black hole thermodynamics
as it is responsible for much of the reasoning that was applied to this analogy,
in that it is clearly reminiscent of the second law of thermodynamics, and we
shall explore these parallels in greater depth below. Along with the area
theorem, in~\cite{Hawking:1971vc}, Hawking also famously proves that under
gravitational collapse, a body of matter will not only form a black hole, but
that the event horizon of the black hole formed will have spherical topology, be
stationary and axisymmetric.

During the summer of 1972, Bardeen, Carter and Hawking (BCH) pursued the
aforementioned analogy between the macroscopic properties of black holes and
thermodynamical systems, work which culminated in the formalisation of \emph{the
  four laws of black hole mechanics}~\cite{Bardeen:1973gs}, a homage to their
statistical counterparts. Let us now summarise these laws as such:
\begin{itemize}
\item The \emph{zeroth law} states that the surface gravity $\kappa$ of a
  stationary black hole is constant over its event
  horizon~\cite{Bardeen:1973gs,Hawking:1973qla,Carter:1973rla,Carter2009,
    Carter2010}.
\item The \emph{first law} expresses conservation of energy during physical
  processes through changes in the properties of the black hole such as its mass
  $M$, area $\mathcal{A}$, angular momentum $J$ or charge $Q$ with the following
  relation:
  \begin{align}
    \label{eq:firstlawBHM}
    \delta M = \frac{\kappa}{8\pi} \delta \mathcal{A} + \Omega \delta J + \Phi
    \delta Q, 
  \end{align}
  where $\Omega$ is the angular velocity of the black hole at its event horizon
  and $\Phi$ is the electrostatic potential. This relation was proved using a
  variational principle by BCH in the uncharged case, and Carter published the
  proof for the charged case in the conference proceedings for the summer
  school~\cite{Carter:1973rla} (repub. in~\cite{Carter2010}). We review this
  work in \cref{sec:firstlaw}.
\item The \emph{second law} is the area theorem itself. Following the current
  theme, we re-express \cref{eq:areatheorem} simply as:
  \begin{align}
    \label{eq:secondlaw}
    \delta\mathcal{A}\geqslant 0.
  \end{align}
\item Finally, the \emph{third law} states that it is impossible through any
  finite sequence of physical processes to reduce the surface gravity $\kappa$
  to zero. At the time, this law was only conjectured and argued for using
  logical arguments. It had actually been shown that processes leading to such a
  configuration did exist if one allowed for infinite divisibility of matter and
  infinite time~\cite{Christodoulou:1970wf,Bardeen:1970zz}. Werner Israel proved
  the third law a decade later~\cite{Israel:1986gqz} .
\end{itemize} 

The analogy between black hole mechanics and classical thermodynamics is
complete once some form of identification is made between the surface gravity
$\kappa$ and the temperature $T$, and between the horizon area $\mathcal{A}$ and
the entropy $S$, both up to some factor determined such that
$T\delta S = \kappa \delta \mathcal{A} / 8\pi$. The authors
of~\cite{Bardeen:1973gs} were initially reluctant to make this identification
and emphasized that while the similarities existed, these quantities should not
lead to the interpretation of the black hole as having either temperature or
entropy, and understandably so---classically, the effective temperature of a
black hole is absolute zero, as it is (or at least was thought to be) unable to
emit any radiation. In fact, they point out that black holes transcend the
second law in that one might in theory be able to add entropy to a black hole
without changing its final state by much. The concept of horizon area as entropy
was not, however, a new one.

Little over half a year prior to \emph{The four laws of black hole mechanics}
was received for publication, and only a couple of months before the
aforementioned summer school during which much of this work was conceived (but a
year after Hawking's initial paper on the area theorem was published), Jacob
Bekenstein wrote in a letter addressing this specific apparent violation of the
second law of thermodynamics~\cite{Bekenstein:1972tm}. In the letter, he
proposes a generalised form of the second law of thermodynamics in which the
quantity that is observed to never decrease is given by the sum of common
entropy, the entropy of the spacetime outside the black hole and a new
\emph{black hole entropy} proportional to the horizon area, ie.
$S_{\mathrm{bh}}=\eta \mathcal{A}$, with $\eta$ an as-yet-undetermined
proportionality constant. This choice of having entropy proportional to the
horizon area was actually motivated by Hawking's work on the area
theorem~\cite{Hawking:1971tu} as well as similar work by Christodoulou and
Ruffini around the same time~\cite{Christodoulou:1970wf,Christodoulou:1972kt} on
reversible and irreversible processes for black holes. In further
works~\cite{Bekenstein:1973ur,Bekenstein:1974ax}, Bekenstein attempts to
establish the proportionality factor on heuristic grounds and proposes a value
for $\eta=\log(2)/8\pi$. In fact, by considering a differential formula akin to
the first law (\cref{eq:firstlawBHM}), Bekenstein was actually able to postulate
a possible expression for the temperature, by considering the conjugate quantity
to entropy; however he too warns against interpreting this as a true temperature
in the thermal sense. That black holes could not seemingly have a nonzero
temperature proved to be the obstacle preventing Bekenstein's work on black hole
entropy from catching on in other circles early on~\cite{10.2307/24953849}.

In the early 1970s, a classical mechanism for stimulated emission from a black
hole had been discovered, known as superradiance. It describes the amplification
of waves in a special \emph{superradiant regime}, incident on a generic
Kerr-Newman black hole. The easiest way to see this effect is by considering a
wave packet of frequency $\omega$, axial quantum number $m$ and charge $e$,
incident on a black hole. The first law (\cref{eq:firstlawBHM}) should hold
throughout the capture of this packet and one expects differentials in the
ratios of $m:\omega$ and $e:\omega$. It therefore follows that
\begin{align}
  \left(1-\Omega \frac{m}{\omega}-\Phi\frac{e}{\omega}\right) \delta M = \frac{\kappa}{8\pi}\delta\mathcal{A}.
\end{align}
Imposing the second law (\cref{eq:secondlaw}) allows for $\delta M \leqslant 0$
for wave packets that satisfy
\begin{align}
  \label{eq:superradiant}
  \omega \leqslant \Omega m + \Phi e.
\end{align}
In other words, we expect wave packets incident in this superradiant regime to be
scattered off the black hole with a larger amplitude.

This effect can be traced back to works by Yakov
Zel'dovich~\cite{Zeldovich:1970aa,Zeldovich:1972aa}, however it was also
separately pointed out by Misner~\cite{Misner:1972kx}, apparently unaware of
Zel'dovich's work. Other notable contributions to the understanding of this
amplifications were made by Press and Teukolsky~\cite{Press:1972zz},
Starobinskii~\cite{Starobinskii:1973aa} and by
Bekenstein~\cite{Bekenstein:1973mi}. While most of the work cited above concerns
itself with the amplification of superradiant incident waves through stimulated
emission, Zel'dovich also suggested that, taking into account quantum mechanical
effects, one could in principle also expect spontaneous emission in superradiant
modes as well. Writing in his thesis, Don Page describes how, oblivious to
Zel'dovich's work, he and Larry Ford had at the time independently discovered
this effect, eventually going on to have discussions with Feynman, Thorne, Press
and Teukolsky, before being made aware of the
above~\cite{Page:1976zz} (along with a few amusing anecdotes from the time in~\cite{Page:2006my}).

What is important to note here is that while it was universally accepted that a
black hole could not radiate, a black hole emitting in superradiant modes would
technically not violate the second law of black hole mechanics, by
definition. It should come, therefore, as no surprise that when Hawking
eventually heard about this effect, as it made its way across research circles,
his interest was piqued and, after having discussions with Zel'dovich and
Starobinskii while in Moscow, began working on a field theory calculation that
might help in validifying this quantum phenomenon~\cite{Page:2006my}.

In 1974, Hawking made the groundbreaking discovery that not only did black holes
radiate in these superradiant modes, but that emission from black holes in fact
covered as much of the spectrum it
could~\cite{Hawking:1974rv,Hawking:1974sw}. Hawking himself, later writes, in
retrospect~\cite{10.2307/24953849}, about how he was initially embarrassed by
the result and therefore attempted to introduce various cutoffs in an effort to
suppress these additional modes. He eventually accepted the result, citing the
fact that the radiation was identical to thermal radiation from a body with
temperature $\kappa/2\pi$ as the smoking gun. Another reason Hawking ended up
believing in his result was that it made Bekenstein's theory of black hole
entropy consistent. This result has since been verified by several other
means~\cite{Bekenstein:1975tw,DeWitt:1975ys,Parker:1975jm,Hawking:1974sw,Wald:1975kc,Gerlach:1976ji,Hartle:1976tp,Unruh:1976db,Boulware:1975fe,Davies:1976ei,Hawking:1976de}
and is generally accepted as a correct result, despite our lack, still to this
day, of a consistent picture of quantum gravity.

Hawking temperature, in a sense, was the final piece of the puzzle and the
reason why black hole mechanics became black hole \emph{thermodynamics}. With the
expression Hawking found, we are now able to rewrite the first law as
\begin{align}
  \label{eq:firstlaw}
  \delta M = T_\mathrm{H} \delta S_\mathrm{BH} + \Omega \delta J + \Phi \delta Q,
\end{align}
where the temperature of the black hole is known as the \emph{Hawking
  temperature}
\begin{align}
  \label{eq:hawkingtemperature}
  T_\mathrm{H}=\frac{\kappa}{2\pi},
\end{align}
and the entropy of the black hole is known as the \emph{Bekenstein-Hawking
  entropy}
\begin{align}
  \label{eq:BHentropy}
  S_\mathrm{BH}=\frac{\mathcal{A}}{4}.
\end{align}
Similarly, we promote the second law to the aforementioned generalised second
law, with the black hole entropy given by a quarter the horizon area.

\section{The First Law}
\label{sec:firstlaw}

When Roy Kerr first presented the metric for a rotating black hole in
1963~\cite{Kerr:1963ud}, most commonly written in Boyer-Lindquist\footnote{Boyer
  and Lindquist were responsible for the maximal extension of the Kerr
  metric~\cite{Boyer:1966qh}. Boyer-Lindquist coordinates are coordinates best
  suited for describing ellipsoids and are related to cartesian coordinates
  through $ x = \sqrt{r^2+a^2}\sin\theta\cos\phi$,
  $y = \sqrt{r^2 + a^2}\sin\theta \sin \phi$ and $z=r\cos\theta$.} coordinates
as
\begin{gather}
  ds^2 = -\frac{f(r)}{\Sigma}(dt - a \sin^2 \theta d\phi)^2 +
  \frac{\Sigma}{f(r)}dr^2 + \Sigma r^2 d\theta^2 + \frac{\sin^2\theta}{\Sigma
  r^2}(adt - (r^2+a^2)d\phi)^2,\nn\\
  f(r) = 1- \frac{2m}{r} + \frac{a^2+e^2}{r^2}, \qquad \Sigma = 1+\frac{a^2}{r^2}\cos^2\theta,
  \label{eq:kerrAF}
\end{gather}
along with the gauge potential
\begin{align}
  B&=-\frac{e}{\Sigma r}(dt-a\sin^2\theta d\phi),
\end{align}
he provided an interpretation for the solution parameters $m$ and $a$ by
comparing its Taylor expansion to a previously known approximation of a spinning
particle, and concluded that to an observer at infinity this rotating black hole
would be equivalent to a particle of mass $M=m$ and angular momentum $J=ma$. The
charge term in $f(r)$ was later added by Newman \emph{et
  al.}~\cite{Newman:1965my} and a similar line of reasoning leads one to
conclude that a Kerr-Newman black hole with nonzero charge parameter $e$ has an
electric charge $Q = e$.

In 1969, Penrose provided a simple mechanism through which one could envision
extracting the rotational energy of a Kerr black hole~\cite{Penrose:1969pc}, and
in 1970, Christodoulou~\cite{Christodoulou:1970wf,Christodoulou:1972kt} (with
Ruffini, in 1971) showed using this picture that a black hole's mass could be
decomposed using an irreducible mass $M_\mathrm{ir}$, which represents the mass
of the remaining black hole when all its rotational and electromagnetic energy
is stripped, according to the following formula:
\begin{align}
  \label{eq:christodoulou-ruffini}
  M=\sqrt{\left(M_\mathrm{ir}+\frac{Q^2}{4M_\mathrm{ir}}\right)^2 + \frac{J^2}{4M_\mathrm{ir}^2}}.
\end{align}
Christodoulou and Ruffini point out that this is equivalent in principle to
Hawking's area theorem and one can indeed obtain \cref{eq:christodoulou-ruffini}
by treating $M_\mathrm{ir}$ as the mass of a Schwarzschild black hole with equal
area to a generic Kerr-Newman black hole. One then deduces
$M_\mathrm{ir}^2 = \mathcal{A}/16\pi$. An appealing aspect of this definition is
the resemblance it bears to the energy of a relativistic particle, when
expressed in terms of its rest mass and momentum.

In 1972, Larry Smarr pointed out, using this relation, that if one expressed
mass as a function $M=M(\mathcal{A},J,Q)$ of area, angular momentum and charge,
the exact differential
\begin{align}
  \label{eq:exactdiff}
dM = \mathcal{T}d\mathcal{A} + \Omega dJ + \Phi dQ
\end{align}
could be used to obtain the invariant quantities
$\mathcal{T} = (\partial M/\partial A)_{J,Q} = \kappa/8\pi$, referred to at the
time as the effective surface tension, $\Omega=(\partial M/\partial J)_{A,Q}$,
the angular velocity of the black hole, and
$\Phi = (\partial M/\partial Q)_{A,J}$, the electromagnetic potential. Smarr
then simply observed that with these quantities, \cref{eq:christodoulou-ruffini}
could be rewritten nicely as
\begin{align}
  \label{eq:smarrrelation}
  M = 2\mathcal{T}\mathcal{A} + 2\Omega J + \Phi Q.
\end{align}
In fact, this result follows immediately from Euler's theorem on homogeneous
functions, given that $M(\lambda\mathcal{A}, \lambda J, \lambda^\frac12 Q) =
\lambda^\frac12 M(\mathcal{A},J,Q)$.

It is in fact quite common to use the exact differential in \cref{eq:exactdiff}
to establish the first law itself. This was the method Bekenstein used
in~\cite{Bekenstein:1973ur} to hypothesise a black hole temperature, and this
will also be how we will ultimately establish a first law for accelerating black
holes, the object of this thesis. 

We began this section by pointing out that the origin of the interpretations of
mass and angular momentum as $M = m$ and $J = ma$ was a term-by-term comparison
with what might be expected in some low-energy limit. While this was a practical
approach, a formal method for identifying such conserved charges had in fact
already been developed by Arthur Komar in 1958~\cite{Komar:1959aa}. A conserved
current can be formed by contracting a killing vector $k$ with the Ricci
tensor. The corresponding conserved quantity is obtained by integrating this
current over a spacelike hypersurface $\mathcal{S}$ normal to the killing
vector. Using the fact that a killing vector satisfies
\begin{align}
  \label{eq:killing}
  \nabla_a \nabla^a k^b = -R^b_{~a}k^a,
\end{align}
a \emph{Komar integral} associated to a given killing vector is derived
as
\begin{align}
  \label{eq:komar}
  E_k\sim\int_{\partial\mathcal{S}} \nabla^a k^b ~ d\Sigma_{ab} = \int_{\partial\mathcal{S}} * \mathrm{\mathbf{d}}k,
\end{align}
where a normalisation is needed, which can be determined at a later stage, and
$d\Sigma_{ab}$ is the volume element on $\partial \mathcal{S}$, to give a
conserved charge $E_k$. One finds that for $k = k_t$ the timelike killing
vector, and for $k = k_\phi$, the rotational killing vector, this integral, when
evaluated at infinity $\partial \mathcal{S}_\infty$ and normalised, yields
$M = m$ and $J = m a$ respectively, for the Kerr-Newman metric. One can
construct a spacelike surface $S$ which ends only on the boundary and extends
through the horizons to the singularity.

Bardeen, Carter and Hawking were able to prove Smarr's relation, for the
uncharged case in a more general setting in~\cite{Bardeen:1973gs} and Carter
provided the proof for the charged case
in~\cite{Carter:1973rla,Carter2010}. Both methods use this as an intermediary
step to deriving the first law, showing how the aforementioned exact
differential expression~\eqref{eq:exactdiff} is in fact fully
self-consistent. We will now briefly review this calculation, omitting technical
steps which can certainly all be found in the original works.

Let us consider the Komar integrals for both killing vectors, evaluated over a
surface $\mathcal{S}$ which now extends from the boundary
$\partial \mathcal{S}_\infty$ and the event horizon $\partial
\mathcal{B}$. Integrating both sides of \cref{eq:killing} will allow us to
express mass and angular momentum as
\begin{align}
  M &= \frac{1}{4\pi}\int_\mathcal{S} R^a_{~b}k^b_t ~d\Sigma_a + M_\mathrm{H}, 
  &M_\mathrm{H} &\equiv -\frac{1}{4\pi}\int_{\partial \mathcal{B}}\nabla^a k_t^b ~ d\Sigma_{ab},\nn\\
  J &= -\frac{1}{8\pi}\int_\mathcal{S} R^a_{~b}k^b_\phi ~d\Sigma_a + J_\mathrm{H}, 
  &J_\mathrm{H} &\equiv \frac{1}{8\pi}\int_{\partial\mathcal{B}}\nabla^a k_\phi^b ~ d\Sigma_{ab},
\label{eq:smarrder1}
\end{align}
where $M_\mathrm{H}$ and $J_\mathrm{H}$ are the corresponding boundary integrals
evaluated at the horizon. For vacuum metrics, $M_\mathrm{H}=M=m$ and
$J_\mathrm{H}=J=ma$. The null generator of the horizon can be defined as
$l^a = k_t^a + \Omega_\mathrm{H} k_\phi^a$, where $\Omega_\mathrm{H}$ is a
scalar quantity which is obtained from the requirement that $l$ be orthogonal to
the rotational killing vector $k_\phi$. It can further be shown that
$\Omega_\mathrm{H}$, given by
\begin{align}
  \label{eq:angularvelocity}
  \Omega_\mathrm{H} = -\frac{g_{t\phi}}{g_{\phi\phi}},
\end{align}
is constant over the horizon and represents its angular velocity. Making use of
the fact that the surface gravity $\kappa$ can actually be defined in terms of
this vector as $\kappa = n_bl^a\nabla_al^b$, where $n$ is the unit normal to the
horizon, one eliminates $M_\mathrm{H}$ from \cref{eq:smarrder1}, establishing
\begin{align}
  \label{eq:smarrder2}
  \frac12 M = \int_\mathcal{S} \left(T^a_{~b} - \frac12 T \delta^a_b\right)k_t^b~d\Sigma_a + \Omega_\mathrm{H}
  J_\mathrm{H} + \frac{\kappa \mathcal{A}}{8\pi},
\end{align}
which reduces to the Smarr relation in the absence of charge and any external
matter. A further decomposition can be made by splitting the stress-energy
tensor $T^{ab} = T_\mathrm{M}^{ab} + T_\mathrm{F}^{ab}$ into its matter and
electromagnetic parts, which in turn allows us to express the total angular
momentum $J = J_\mathrm{M} + J_\mathrm{F} + J_\mathrm{H}$ in a similar
fashion. Finally, one can define the electric charge by integrating the
electromagnetic current $j^a = \nabla_b F^{ab}/4\pi$ over the same spacelike
surface, and, in analogy to the mass and the angular momentum formulae in
\cref{eq:smarrder1} we write it as
\begin{align}
  Q = -\int_\mathcal{S} j^a ~d\Sigma_a + Q_\mathrm{H}, \qquad Q_\mathrm{H} \equiv -\frac{1}{4\pi}\int_{\partial\mathcal{B}}F^{ab} ~ d\Sigma_{ab}.
\end{align}
It then follows that \cref{eq:smarrder2} can be rearranged into
\begin{align}
  \label{eq:smarrder3}
  \frac12 M &= \int_\mathcal{S} \left(T^a_{\mathrm{M}~b} - \frac12 T_\mathrm{M}
  \delta^a_b\right)k_t^b~d\Sigma_a - \Omega_\mathrm{H} J_\mathrm{M} - \frac12
\int_\mathcal{S} l^c A_c j^a~d\Sigma_a + \int_\mathcal{S} A_b j^{[b}l^{a]}~d\Sigma_a \nn\\
&\hspace{5em}+ \Omega_\mathrm{H} J + \frac{\kappa \mathcal{A}}{8\pi} + \frac12
\Phi_\mathrm{H} Q_\mathrm{H},
\end{align}
and the Smarr relation is now recovered when the external matter fields and
source currents in the first line are switched off.

The formula in \cref{eq:smarrder3} describes the total mass $M$ of the system,
which is a time-conserved quantity. It is possible, however, to look at two
neighbouring configurations with slightly different $M$, and consider the
corresponding variation $\delta M$, similar to the way in which one varies the
action. It is by performing this variation that one obtains the first law in its
most general form for a rotating charged black hole surrounded by electric and
matter fields:
\begin{align}
  \label{eq:firstlawder1}
  \delta M &= \int \Omega\delta dJ_\mathrm{M} + \int \overline{\Theta}\delta dS +
  \int \overline{\mu}^{(i)} \delta dN_{(i)} +\int \Phi_\mathcal{S}\delta dQ \nn\\
  &\hspace{12em}+ \Phi_\mathrm{H} \delta
  Q_\mathrm{H} + \frac{\kappa}{8\pi}\delta\mathcal{A} + \Omega_\mathrm{H}(\delta
  J_\mathrm{H}+\delta J_\mathrm{F}),
\end{align}
where $\Omega$ is the angular velocity of the fluid, $\overline{\Theta}$ is its
``effective'' or ``red-shifted'' temperature, $\overline{\mu}^{(i)}$ the
effective chemical potential corresponding to each type of particle,
$\Phi_\mathcal{S}$ denotes the electromagnetic potential accross the surface
$\mathcal{S}$, and finally the notation ``$\int\delta d$'' signifies a change in
fluid angular momentum ($J_\mathrm{M}$), entropy ($S$) or particle number
($N_{(i)}$) crossing the surface $\mathcal{S}$. This elegant result reduces to
the relation initially presented in \cref{eq:firstlawBHM} when all external
fields and sources are turned off.

This derivation shows why the first law holds, at least for asymtotically-flat
rotating black holes, from first principles. Starting with the definition of
mass as a Komar integral, we are able to show why variations of this quantity
are related to all the other variations present in the first law in the way that
they are. Previously, it had been known that one could express variations in the
mass in this way, as the exact differential~\eqref{eq:exactdiff}, however this
was seen as more of an observation based on final expressions obtained using
independent definitions for mass, area, surface gravity and other
quantities. This elegant derivation shows not merely how all of these
expressions are related, but how their \emph{definitions} are related.

\section{The Euclidean Approach to Black Hole Thermodynamics}
\label{sec:euclidean}

By the end of 1975, Hawking had already devised a way, with Jim Hartle, to
reproduce his black hole radiation calculation using the path-integral
formulation of quantum field theory~\cite{Hartle:1976tp}. In this approach, the
central object is the partition function, which has the form
\begin{align}
  \label{eq:partitionfunction1}
  Z = \int \mathcal{D}g \mathcal{D} \Phi~e^{iS[g,\Phi]},
\end{align}
where we are integrating over the space of all possible metrics, including those
which are topologically distinct spacetimes such as black hole solutions, and
all field configurations. It was then shown, by Gibbons and Hawking in
1976~\cite{Gibbons:1976ue}, that, given a properly regularised gravitational
action, it was possible to recover all the thermodynamic behaviour of black hole
mechanics from the partition function, in analogy to regular euclidean field
theory.

The action in \cref{eq:partitionfunction1} for a generic gravitational solution
is given by
\begin{align}
  S = -\frac{1}{16\pi}\int_Y d^4x \sqrt{-g} R - \frac{1}{8\pi}\int_{\partial
  Y}d^3x\sqrt{-h}K + S_\mathrm{C} + S_\mathrm{M}.
\end{align}
It is computed within a region of spacetime $Y$ and is made up of the
Einstein-Hilbert, Gibbons-Hawking, counter and matter terms respectively. $R$ is
the Ricci scalar for the bulk metric $g_{ab}$, $K$ is the extrinsic curvature of the
boundary $\partial Y$ and $h_{ab}$ is its induced three-dimensional metric. The
counterterm $S_\mathrm{C}$ is determined such that $S$ vanishes in flat space.

An issue that arises when computing the action of a black hole metric is that
singularities must be avoided. While it is known that crafty coordinate choices
allow one to patch over horizons, curvature singularities are intrinsic to the
geometry and may not be removed. It is possible, however, to construct a smooth
patch which avoids the singularity all together, by complexifying the timelike
coordinate. The subsequent spacetime will only be smooth provided the new
coordinate $\tau=it$ is made to have a periodicity $\beta = 2\pi/\kappa$, where
$\kappa$ is the surface gravity of the horizon. Let us illustrate this with the
Schwarzschild metric,
\begin{align}
  \label{eq:schwarzschildAF}
  ds^2 = -f(r)dt^2 + \frac{dr^2}{f(r)} + r^2 d\Omega^2, \qquad f(r) = 1-\frac{2m}{r}.
\end{align}
In the vicinity of the horizon $r_\mathrm{h} = 2m$, the Euclidean Schwarzschild
metric will take the form given by
\begin{align}
  \label{eq:schwarzschildAFE}
  ds^2 = f'(r_\mathrm{h})(r-r_\mathrm{h})d\tau^2+
  \frac{dr^2}{f'(r_\mathrm{h})(r-r_\mathrm{h})}+r_h^2 d\Omega^2.  
\end{align}
Introducing temporary coordinates $\rho = 2\sqrt{r-r_\mathrm{h}}$ and
$\varphi = f'(r_\mathrm{h})\tau/2$, it becomes clear that with
$\beta = f'(r_\mathrm{h})/4\pi = 8\pi m$, these coordinates describe a spacetime
of topology $S^1 \times S^2$. More precisely, each point in the $\tau-r$
subspace corresponds to a 2-sphere of corresponding radius, and the subspace
itself has an $S^1$ symmetry centred around the point $r = r_\mathrm{h}$. One
could then compute the action on a region bounded by the surface
$r = r_\mathrm{b}$.

In euclidean field theory, the periodicity of imaginary time ends up
corresponding to the temperature of the system through $T=\beta^{-1}$. This is
then used to express the partition function of such a field theory. In the grand
canonical ensemble, such a system will have thermodynamically conserved
quantities $Q_i$, and respective conjugate potentials $P_i$. The partition
function can be expressed as
\begin{align}
  \label{eq:partitionfunction2}
  Z = \Tr e^{-\beta\left(H - \sum_iP_i Q_i\right)} = \Tr e^{-\beta F},
\end{align}
where $F$ is the grand canonical free energy potential, or ``grand potential''
for short --- in the current text, however, this quantity will loosely be
referred to as the \emph{free energy potential}, and will be defined according
to the ensemble at hand.

Returning to the gravitational partition function, it follows from the
path-integral approach that the integral in \cref{eq:partitionfunction1} will
receive its most dominant contributions from the on-shell metric, that which
satisfies Einstein's equations. This allows us to use the following
approximation for the partition function, using the euclidean action,
\begin{align}
  \label{eq:partitionfunction3}
  Z  \approx e^{-S_\mathrm{E}[g,\Phi]}.
\end{align}
In analogy with the non-gravitational situation, we derive the free energy
from the partition function using \cref{eq:partitionfunction2}, and write
\begin{align}
  \label{eq:freeenergy1}
  F = -\frac{\log Z}{\beta}=\frac{S_\mathrm{E}}{\beta}.
\end{align}
Finally, one recovers an expression for the mass of the black hole by reverse
Legendre transforming the free energy:
\begin{align}
  \label{eq:massfromaction}
  M &= F + TS + \sum_iP_i Q_i\nn\\
    &= 2TS + 2\Omega J + \Phi Q
\end{align}
for the Kerr-Newman black hole. Historically, this was used to affirm the
expression for entropy using the Smarr formula from \cref{eq:smarrrelation},
however it could equally be used to verify the converse.


\section{Incorporating $\Lambda$}
\label{sec:BHTDlambda}


\placehold{Generic intro to AdS and interest in it since Maldacena. remember AdS acronym}

Early research concerning gravitational solutions which included a cosmological
constant was seen as purely academic at the time.
\begin{itemize}
\item astronomical discovery
\item diversifies landscape of solutions
\item Cosmological motivations, research into inflation, bubbles?
\item AdS/CFT and applications for studying qfts as well as condensed matter systems.
\end{itemize}
\begin{align}
  \label{eq:lambda}
  \Lambda = -\frac{(D-1)(D-2)}{2\ell^2} = -\frac{3}{\ell^2}
\end{align}

\subsection{Thermodynamics of the Kerr-AdS Black Hole}

The obvious starting point for introducing a cosmological constant is to review
how this fits into the thermodynamic framework for studying black holes
presented above. The thermodynamics of black holes in the context of a
cosmological constant were actually first discussed in a paper by Gibbons and
Hawking in 1976~\cite{Gibbons:1977mu}, in which they discuss hawking radiation
and black hole temperature in a de Sitter ($\Lambda > 0)$ background, a delicate
topic since de Sitter space also has a cosmological horizon. The original
motivation of the paper was in fact to better define the hawking radiation
associated to cosmological horizons. In 1982, Hawking and Page presented a
thermodynamic description of the Schwarzschild-AdS
spacetime~\cite{Hawking:1982dh}. By computing the action using a regularisation
scheme similar to the method presented in \cref{sec:euclidean}, adapted to an
asymptotically-AdS spacetime, HP derive expressions for the temperature and
entropy of the black hole in this background and in fact used these quantities
to show that there existed a phase transition for black holes in an anti-de
Sitter background, which we will review below.

While the Kerr-AdS metric has been known since the late
1960s~\cite{Carter:1968ks}, its thermodynamics were only discussed near the turn
of the century, when interest in asymptotically-AdS solutions boomed due to the
aforementioned AdS/CFT conjecture~\cite{Hawking:1998kw, Caldarelli:1999xj,
  Silva:2002jq, Gibbons:2004ai}. The metric, in Boyer-Lindquist coordinates, is
given by
\begin{subequations}
  \label{eq:kerrAdS}
  \begin{align}
    \label{eq:kerrAdS-metric}
    ds^2 &= -\frac{f(r)}{\Sigma}\Big(dt - a \sin^2 \theta \frac{d\phi}{\Xi}\Big)^2 +
           \frac{\Sigma}{f(r)}dr^2 + \frac{\Sigma r^2}{g(\theta)} d\theta^2 \nn\\
         &\hspace{10em}+ \frac{g(\theta)\sin^2\theta}{\Sigma r^2}\Big(adt -
           (r^2+a^2)\frac{d\phi}{\Xi}\Big)^2,
  \end{align}
  where
  \begin{gather}
    f(r) = \Big(1 + \frac{a^2}{r^2}\Big) \Big(1 + \frac{r^2}{\ell^2}\Big) -
    \frac{2m}{r}+\frac{e^2}{r^2}, \qquad g(\theta) = 1 -
    \frac{a^2}{\ell^2}\cos^2\theta, \nn\\ 
    \Sigma = 1+\frac{a^2}{r^2}\cos^2\theta,\qquad \Xi = 1-\frac{a^2}{\ell^2}.
    \label{eq:kerrAdS-fn}
  \end{gather}
\end{subequations}
The corresponding gauge potential is given by
\begin{align}
  \label{eq:kerrAdS-gauge}
  B&=-\frac{e}{\Sigma r}\Big(dt-a\sin^2\theta \frac{d\phi}{\Xi}\Big).
\end{align}

Hawking, Hunter and Taylor-Robinson~\cite{Hawking:1998kw} were the first to
present a set of thermodynamic quantities for the uncharged rotating black hole
in AdS. By computing the thermodynamics of rotating bulk spacetimes such as the
Kerr-AdS metric, they were able to reconcile the bulk partition function with
the partition function of a scalar field coupled to a
three-dimensional\footnote{They also perform this computation in a number of
  other dimensions; we single out this case for its relevance.} rotating
Einstein universe, which forms the boundary of Kerr-AdS. 

Computing the temperature from the metric \cref{eq:kerrAdS} in its Euclidean
form (for rotation one must simultaneously take $\tau=it, \alpha=-ia$), and its
entropy as a quarter the horizon area, they find:
\begin{align}
  \label{eq:kerrAdSTS}
  T&=\frac{r_+^2f'(r_+)}{4\pi(r_+^2+a^2)} = \frac{r_+}{4\pi(r_+^2+a^2)} \left[1 +
  \frac{3r_+^2}{\ell^2} + \frac{a^2}{\ell^2} - \frac{a^2}{r_+^2}\right],\nn\\
  S&=\frac{\pi(r_+^2+a^2)}{\Xi},
\end{align}
where $r_+$ denotes the location of the outer event horizon. They compute the
mass $M$ and angular momentum $J$ using the timelike and angular killing vectors
with corresponding Komar integrals, noting that a background $m=0$ subtraction
is needed for regularisation (see Magnon~\cite{Magnon:1985sc}), and find:
\begin{align}
  \label{eq:kerrAdSMJ-HTT}
  M=\frac{m}{\Xi}, \qquad J=\frac{am}{\Xi^2}.
\end{align}
Finally, the thermodynamic conjugate to the angular momentum, the angular
velocity $\Omega$ is given as
\begin{align}
  \label{eq:kerrAdSO-HTT}
  \Omega_\mathrm{H} = -\frac{g_{t\phi}}{g_{\phi\phi}}\Big|_{r_+} = \frac{a\Xi}{r_+^2+a^2},
\end{align}
and with all of these quantities, they were able to show that the partition
function for this spacetime corresponded to the partition of a scalar field
theory coupled to a three-dimensional Einstein universe rotating with angular
velocity $\Omega_\mathrm{H} + a/\ell^2$.

The authors of~\cite{Hawking:1998kw} did not, however, discuss the first law of
thermodynamics, and one can check that \cref{eq:firstlaw} does not in fact hold
with \cref{eq:kerrAdSTS,eq:kerrAdSMJ-HTT,eq:kerrAdSO-HTT}. The following year,
wanting to present a thorough thermodynamic description of Kerr-Newman-AdS black
holes, Caldarelli, Cognola and Klemm~\cite{Caldarelli:1999xj} addressed this
issue. When one computes the mass using a Komar integral, there lies an
ambiguity in the normalisation of the timelike killing vector used. In
~\cite{Hawking:1998kw}, the mass is computed with the killing vector
$\partial_t$, however the authors of~\cite{Caldarelli:1999xj} compute it using
the killing vector $\partial_t/\Xi$, while the angular momentum is still
computed with $\partial_\phi$. This leads to the following expressions for
the mass and angular momentum:
\begin{align}
  \label{eq:kerrAdSMJ-CCK}
  M=\frac{m}{\Xi^2} \qquad J=\frac{am}{\Xi^2}.
\end{align}
The justification for this choice of normalisation is that the resulting
conserved quantities above agree with expressions obtained using a Hamiltonian
approach that had been presented by Henneaux and
Teitelboim~\cite{Henneaux:1985tv} in 1985 for the uncharged case and by
Kostelecky and Perry~\cite{Kostelecky:1995ei} a decade later for the charged
case. They also show that the same expressions can be obtained using the
Brown-York method~\cite{Brown:1992br}.

Caldarelli \emph{et al.} then derived a Christodoulou-Ruffini-like formula for the
mass, expressing it as a function $M=M(S, J, Q)$. They then reverse engineer the
first law to verify the quantities conjugate to the entropy, angular momentum
and charge by computing the relevant partial derivatives. The temperature is
found to agree with the charged version of \cref{eq:kerrAdSTS}, and the
electrostatic potential is found to agree with
\begin{align}
  \label{eq:kerrAdSP-CCK}
  \Phi = B_\mu\chi^\mu\Big|^\infty_{r_+}=\frac{er_+}{r_+^2+a^2},
\end{align}
where $B$ is the electric gauge potential from \cref{eq:kerrAdS-gauge} and
$\chi=\partial_t+\Omega_\mathrm{H}\partial_\phi$ is the null generator of the
horizon. Most significantly, however, they found that $\Omega = (\partial
M/\partial J)_{S,Q}\neq \Omega_\mathrm{H}$ but rather that
\begin{align}
  \label{eq:kerrAdSO-CCK}
  \Omega = \frac{a(1+r_+^2/\ell^2)}{r_+^2+a^2}=\Omega_\mathrm{H}+\frac{a}{\ell^2},
\end{align}
This happens to coincide precisely with the difference between the angular
velocity at the horizon~\eqref{eq:kerrAdSO-HTT} and the angular velocity at
infinity. It therefore makes sense that the physical quantity relevant to
thermodynamics be the agnostic quantity $\Omega =
\Omega_\mathrm{H}-\Omega_\infty$ measuring the angular velocity of the black
hole relative to the boundary.

Finally, we would expect to be able to establish the Euler scaling relation that
we encountered earlier as Smarr's formula in \cref{eq:smarrrelation}. The
statement as it stands does not hold, despite the new quantities satisfying the
first law. Instead, the authors of~\cite{Caldarelli:1999xj} show that one may
treat the cosmological constant $\Lambda$ itself as a thermodynamical variable,
complete with a conjugate quantity
$\Theta = (\partial M/ \partial \Lambda)_{S,J,Q}$, and we have the first law
\begin{align}
  \delta M = T\delta S + \Omega \delta J + \Phi \delta Q + \Theta \delta \Lambda.
\end{align}
This also allows us to rewrite the Christodoulou-Ruffini formula as
\begin{align}
  \label{eq:smarrAdS1}
  \frac12 M= TS + \Omega J + \frac12 \Phi Q - \Theta \Lambda,
\end{align}
which agrees with mass being a homogeneous function of $S$, $J$, $Q$ and now
$\Lambda$, after Euler's theorem is applied. We leave the discussion of
$\Lambda$ as a varying quantity to the next section.

Finally, both~\cite{Hawking:1998kw,Caldarelli:1999xj} find that the relation
between their respective free energy potentials agree with the euclidean action
according to \cref{eq:freeenergy1}:
\begin{align}
  F = M- TS - \Omega J-\Phi Q = \frac{S_\mathrm{E}}{\beta}.
\end{align}
The fact that this relation is satisfied when using either
\cref{eq:kerrAdSMJ-HTT,eq:kerrAdSO-HTT}, or
\cref{eq:kerrAdSMJ-CCK,eq:kerrAdSO-CCK} is merely a consequence of the relation
\begin{align*}
  \frac{m}{\Xi^2}+\Omega_\infty J = \frac{m}{\Xi}.
\end{align*}

Silva~\cite{Silva:2002jq} then showed that the entropy obtained from this
thermodynamic prescription agreed with the Cardy
entropy~\cite{Cardy:1986ie,Bloete:1986qm} by computing central charges of a
sub-algebra deduced from the metric \cref{eq:kerrAdS}; Gibbons, Perry and
Pope~\cite{Gibbons:2004ai} extend this description to higher dimensional black
holes, clarifying similar subtleties as the one presented above. They also point
out that the mass as expressed in \cref{eq:kerrAdSMJ-CCK} agrees with masses
computed according to Abbott-Deser~\cite{Abbott:1981ff} and
Ashtekar-Das-Magnon~\cite{Ashtekar:1984zz,Ashtekar:1999jx}.

\subsection{Pressure and Volume of Black Hole Spacetimes}
\label{sec:press-volume-black}

The scaling relation~\eqref{eq:smarrAdS1} suggests, as we mentioned above, that
the cosmological constant should be treated as a thermodynamical variable. After
the work above by Caldarelli \emph{et al.}~\cite{Caldarelli:1999xj}, this
concept was revisited a few times
in~\cite{Wang:2006eb,Sekiwa:2006qj,Wang:2006bn,LarranagaRubio:2007fly}, leading
to a 2009 paper by Kastor, Ray and Traschen~\cite{Kastor:2009wy} in which some
of the computations above were put on much firmer mathematical ground, and a
new thermodynamic interpretation was presented, in an aim to reconnect with
traditional statistical mechanics.

While the entire discussion concerning having a dynamical cosmological constant
may be somewhat controversial to some, the argument can be made that whether
physical mechanisms allowing for $\Lambda$ to change exist or not, does not
preclude us from considering neighbouring configurations in parametric
(incl. $\Lambda$) space. Nonetheless, the idea of doing so was initially
proposed in the mid 1980s, where, in a series of papers~\cite{Henneaux:1984ji,
  Teitelboim:1985dp, Henneaux:1985tv, Brown:1987dd, Brown:1988kg,
  Henneaux:1989zc}, Brown, Henneaux and Teitelboim introduce the cosmological
constant as an integration constant from a theory which has a 3-form gauge
potential coupled to the gravitational field. In this theory, there is a bubble
radiation process which is reduces the cosmological
constant. In~\cite{Kastor:2009wy}, it is argued that a change in the
cosmological constant in the bulk is equivalent via the AdS/CFT correspondence
to a change in the number of colors in the non-abelian field theory on the
boundary. Finally, in recent work, Gregory, Kastor and
Traschen~\cite{Gregory:2017sor,Gregory:2018ghc} studied the thermodynamics of a
black hole system in a background cosmology undergoing slow-roll inflation and
confirmed that it changes according to a first law with variable $\Lambda$.

In order to prove the Smarr relation, the authors of~\cite{Kastor:2009wy} were
able to provide a derivation similar to the one we presented in
\cref{sec:firstlaw}, making use of the killing potential, first introduced
in~\cite{Bazanski:1990qd,Kastor:2008xb} to construct properly defined Komar
integrals in asymptotically $\Lambda\neq 0$ geometries. The outcome of this
derivation is that they obtained definition of the conjugate quantity
$\Theta = (\partial M/\partial \Lambda)_{S,J,Q}$ that we introduced in
\cref{eq:smarrAdS1}. The cosmological constant can be interpreted as a pressure
\begin{align}
  P = -\frac{\Lambda}{8\pi}
\end{align}
exerted on the spacetime, and in classical thermodynamics, the conjugate of a
pressure is the system's volume. Indeed, $\Theta$ has units of volume and we
introduce the following quantity:
\begin{align}
  \label{eq:thermovol}
  V=-8\pi\Theta,
\end{align}
known as the \emph{thermodynamic volume} of a black hole. The definition
obtained in terms of killing potentials that was found in~\cite{Kastor:2009wy}
reveals that this quantity \eqref{eq:thermovol} may be interpreted as the volume
excluded from the full spacetime by the event horizon.

A consequence of the interpretation of the cosmological constant as pressure is
that the first law for an uncharged non-rotating black hole now takes the form:
\begin{align}
  \delta M = T\delta S + V\delta P.
\end{align}
Classical thermodynamics teaches us that the mass of the black hole should
therefore not be thought of as the internal energy, but as the
\emph{enthalpy}~\cite{Kastor:2009wy}. The enthalpy is related to the internal
energy via the Legendre transform $U(S,V) = H(S,P) - PV$. It turns out that
because $H$ is linear with respect to $P$ in black hole systems, this
transformation is generally non-invertible (see~\cite{Baldiotti:2017ywq} for a
recent more formal treatment and discussion of the relation between the enthalpy
and internal energy in black hole
thermodynamics). Dolan~\cite{Dolan:2010ha,Dolan:2011xt} has since investigated
the thermodynamical consequences of this interpretation in terms of pressure,
volume and enthalpy. In particular, he notes that the heat capacities, whose
sign determines local thermodynamic stability, must be computed at constant
pressure rather than constant volume, and the free energy which is computed from
the action coincides with the Gibbs free energy
\begin{align}
  G(T,\Omega,\Phi,P) &= U(S,J,Q,V) - TS - \Omega J - \Phi Q + PV\nn\\
                     &= H(S,J,Q,P) - TS -\Omega J -\Phi Q\nn\\
                     &= M - TS - \Omega J - \Phi Q = \frac{S_\mathrm{E}}{\beta}.
\end{align}

More recently, Cveti\v c \emph{et al.}~\cite{Cvetic:2010jb} computed the
thermodynamic volume in a number of asymptotically-AdS black hole spacetimes,
comparing it to the ``na\" ive'' geometric volume, as they refer to it, given by
the integral
\begin{align}
  V' = \int_{r_0}^{r_+}dr\int d\Omega \sqrt{-g},
\end{align}
from the singularity at $r_0$ to the outer horizon at $r_+$. In general, they
find that these quantities agree for static geometries, however differ by a term
proportional to the angular momentum, when it is nonzero. This led them to
conjecture the \emph{reverse isoperimetric inequality}.

The traditional isoperimetric inequality is the general statement of Euclidean
geometry which says that the $D$-dimensional volume enclosed within a
$(D-1)$-dimensional closed surface is maximised when that surface is spherical,
or, if $V'$ is the volume enclosed and $\mathcal{A}$ is the area of
the surface that encloses it, then
\begin{align}\label{eq:euclidiso}
\left(\frac{D V'}{\omega_{D-1}}\right)^{\frac{1}{D}} \leqslant
  \left(\frac{\mathcal{A}}{\omega_{D-1}}\right)^{\frac{1}{D-1}},
\end{align}
where $\omega_{D-1}$ is the area of a unit $D$-dimensional sphere. Cveti\v c
\emph{et al.} examined the applicability of this relation to black hole
spacetimes using the geometric volume $V'$ and found the relation to be
satisfied for uncharged black holes, however charged black holes were found to
violate this statement. When the volume they used was the thermodynamic volume
$V$, they found that \emph{all} black hole spacetimes violated the inequality,
with the Schwarzschild-AdS geometry saturating the bound. This led to the
conjecture that all black holes satisfy the reverse isoperimetric inequality,
which is usually re-expressed as the ratio (increasing the number of
  dimensions by one relative to \cref{eq:euclidiso})
\begin{align}
  \mathcal{R} =
  \left(\frac{(D-1)V}{\omega_{D-2}}\right)^{\frac{1}{D-1}}
  \left(\frac{\omega_{D-2}}{\mathcal{A}}\right)^{\frac{1}{D-2}} 
  \geqslant 1,
\end{align}
which is greater than unity when the conjecture is valid.

Physically, this conjecture is often interpreted to state that for a given
thermodynamic volume, the black hole entropy is maximised in a Schwarzschild
geometry. The inequality has been shown to hold for a plethora of asymptotically
AdS and dS geometries~\cite{Cvetic:2010jb,Dolan:2013ft}, though notable
exceptions have also been
found~\cite{Hennigar:2014cfa,Hennigar:2015cja,Brenna:2015pqa}.

\subsection{Thermodynamic Stability and Phase Transitions}
\label{sec:hawkingpage}

The laws of thermodynamics require the existence of a state of equilibrium. In
flat space, one could envisage a system in a box which contains a black hole at
equilibrium with a thermal bath of radiation. If this radiation bath is held at
a fixed temperature, the equilibrium would be unstable: if a small excess of
mass fell into the black hole, its temperature would decrease and would not be
able to sustain the rate of radiation necessary to recover equilibrium and the
bath would eventually collapse into the black hole; if it were to radiate too
much at equilibrium, its temperature would rise and the rate of radiation would
exceed the rate of absoption and the black hole would evaporate. In de Sitter
(dS) space, the presence of a second, cosmological, horizon implies that
equilibrium may only be truly achieved in the special Nariai
limit~\cite{Nariai:1950aa,Nariai:1951aa} of asymptotically dS black holes where
the two horizons coincide.

On the other hand, the negative curvature inherent to anti-de Sitter (AdS)
geometries allows the entire spacetime to act as the box we used above. Only
massless states are able to escape to infinity and one can have boundary
conditions such than the incoming and outgoing states cancel to maintain a
thermal bath. In 1982, Hawking and Page~\cite{Hawking:1982dh} studied the
physical implications of such a system and made several observations.

Parametrising the cosmological constant as $\Lambda = -3/\ell^2$, the
Schwarzschild-AdS line element is
\begin{gather}
  ds^2 = -f(r)dt^2 + \frac{dr^2}{f(r)} + r^2 d\Omega_{\mathrm{II}}^2, \qquad
  f(r) = 1-\frac{2M}{r}+\frac{r^2}{\ell^2},
  \label{eq:schwAdS}
\end{gather}
where $d\Omega_\mathrm{II}^2$ is the volume element on the two-sphere, and $M$
is the mass of the black hole. If we then identify the temperature as the period
of imaginary time required to regularise the euclidean section and the entropy
as a quarter of the area, we find
\begin{align}
  T = \frac{f'(r_+)}{4\pi} = \frac{1}{4\pi
    r_+}\left(1+\frac{3r_+^2}{\ell^2}\right), \qquad S = \pi r_+^2,
\end{align}
where $r_+$ is the location of the event horizon and we have used $f(r_+)=0$ to
cast the mass in terms of this parameter, though it should be noted that mass
monotonically increases with respect to $r_+$.

Local stability in thermodynamics requires positivity of the specific heat,
which is given by
\begin{align}
  C_P = T \left(\frac{\partial S}{\partial T}\right)_P =
  -\frac{2\pi r_+^2(1+3r_+^2/\ell^2)}{1-3r_+^2/\ell^2} .
\end{align}
We notice therefore that black holes larger than $r_+ = \ell/\sqrt{3}$ may be in
a locally stable equilibrium, whereas for smaller black holes, equilibrium is
unstable. This stable equilibrium is achieved for larger black holes as gaining
mass increases temperature, thereby providing a higher rate of radiation for the
horizon which may then lead back to equilibrium. This turning point actually
corresponds to a minimal black hole temperature
$T_\mathrm{c} = \sqrt{3}/2\pi\ell$.

Another criteria for stability of a thermodynamical system is that its
configuration be that of lowest free energy for the fixed intensive
quantities. If more than one configuration is possible, then those of higher
free energy will have a tendancy to decay to the state of lowest free
energy. The previous section tells us that the quantity we are interested in is
the Gibbs free energy potential given by
\begin{align}
  G = M - TS = \frac{r_+}{4}\left(1-\frac{r_+^2}{\ell^2}\right),
\end{align}
for the Schwarzschild-AdS black hole. There is, however, another possible
thermal configuration that we may consider which is vacuum AdS filled with a
thermal bath, which has negligible free energy, however is not able to sustain
itself against gravitational collapse beyond some temperature $T_\mathrm{u} \sim
s^{-\frac14} \ell^{-\frac12}$ where $s$ is the effective number of spin states
of the radiation~\cite{Hawking:1982dh}. 

\Cref{fig:hawkingpage} illustrates the situation we have described. For
temperatures lower than $T_\mathrm{c}$, the only possible configuration is
vacuum radiation. Above this temperature, there are two branches corresponding
to the larger stable black holes (lower branch) and the smaller unstable black
holes (upper branch). We see that for the lower branch, there is a temperature
$T_\mathrm{HP} = 1/\pi\ell$ above which large black holes on the stable branch
have negative free energy, lower than the corresponding radiation in pure
AdS. There is therefore a first order phase transition at $T_\mathrm{HP}$, known
as the \emph{Hawking-Page phase transition}. Between
$T_\mathrm{c}<T<T_\mathrm{HP}$, there are three configurations possible, however
the vacuum radiation is still the most thermodynamically favoured and one would
expect black holes at this temperature to decay to radiation. For comparison, an
asymptotically flat black hole always has positive free energy and is therefore
never stable against decay to radiation. Via the AdS/CFT correspondence, it has
been shown that in the dual boundary conformal field theory, this phase
transition may be interpreted as a confinement/deconfinement phase
transition~\cite{Witten:1998zw}.

\begin{figure}
  \centering
  \includegraphics{hawkingpage}
  \caption{\label{fig:hawkingpage}\emph{The Hawking-Page phase transition}. The
    black line represents the Gibbs free energy of Schwarzschild-AdS black holes
    at different temperatures. The critical temperature $T_\mathrm{c}$ separates
    the upper and lower branches of this plot, corresponding to unstable smaller
    black holes and stable larger black holes respectively. In blue, we have
    vacuum radiation, which is unstable for temperatures higher than
    $T_\mathrm{u}$ (which has been arbitrarily chosen here). The red dashed line
    corresponds to asymptotically flat black holes and is plotted for
    comparison. These curves have been produced for a
    geometry which has $\ell=1$ in planck units.}
\end{figure}

This analysis may be extended to charged black holes. This was done by Chamblin
\emph{et al.}~\cite{Chamblin:1999tk} in the early days of holography. The
inclusion of charge allows us to consider either the canonical ensemble, where
we fix charge itself or the grand canonical ensemble, where we fix the
electrostatic potential $\Phi$ and allow charge to vary. In the latter case, the free
energy potential we will be interested is given by
\begin{align}
  G' = M - TS - \Phi Q.
\end{align}

In the canonical ensemble we may no longer compare black hole solutions at fixed
charge to pure AdS as the vacuum on-shell field equations do not allow for a
charged radiation bath in AdS. Referring to \cref{fig:charge-fe}, we observe two
different regimes. For $Q<Q_\mathrm{c}$, there are three branches; a first
branch of stable black holes with $M<M_1$, a second intermediary branch of
unstable black holes with $M_1<M<M_2$, and a third branch $M>M_2$. In the free
energy diagram, the first and third branches intersect for some temperature
$T_*$ at which there is a first order phase transition between large and small
black holes. These curves display what has been referred to as a ``swallow
tail'' behaviour, referring to the region enclosed by the three branches. As the
charge increases towards $Q_\mathrm{c}$, this region shrinks and beyond it, in
the second regime, the intersection disappears and the phase transition becomes
continuous. It has been noted that this characteristic is very reminiscent of
the liquid/gas phase transition of a Van-der-Waals fluid.

In the grand canonical ensemble we may still compare our solutions to a
background of vacuum radiation in AdS with a correspondingly fixed electrostatic
potential. From \cref{fig:grand}, we see there is a critical value for this
potential which we denote $\Phi_\mathrm{c}$. Configurations with a potential
smaller than this are similar to the uncharged scenario, with the existence of a
Hawking-Page-like phase transition for the lower, stable, of two branches, and
no black hole solutions existing below a certain temperature. For larger
potentials, the free energy is always negative and the black hole solution is
always preferred.

\begin{figure}
  \centering
  \begin{subfigure}[b]{0.45\textwidth}
    \includegraphics[width=\textwidth]{chargeplot}
    \caption{\label{fig:charge-fe}}
  \end{subfigure}\qquad
  \begin{subfigure}[b]{0.45\textwidth}
    \includegraphics[width=\textwidth]{grand}
    \caption{\label{fig:grand}}
  \end{subfigure}
  \caption{\label{fig:chargeplots}The free energy diagrams for the canonical
    (left) and grand canonical (right) ensembles are reproduced. For the
    canonical ensemble, three curves at different charges ($e = 0.05\ell$,
    $e=0.12\ell$ and $e = 0.2\ell$) are displayed over the uncharged case. The
    two intermdiary curves display the characteristic ``swallow tail''
    behaviour. For the grand canonical ensemble, we have again reproduced three
    curves at fixed potential ($\Phi = 0.7$, $\Phi = 1.0$ and $\Phi = 1.2$) over the
    uncharged plot. These figures have all been reproduced for $\ell = 1$.}
\end{figure}

\begin{center}
  * * *
\end{center}

The framework of black hole thermodynamics is undoubtedly a facinating one. The
ability to breathe life back into as old a subject as thermodynamics by applying
it to modern and exotic solutions of gravitational physics is certainly
exciting. The rich phase structure black holes possess presented in the last
section are an example of the utility of thermodynamics.

\placehold{examples of applications of black hole thermo:
  geometrothermodynamics, black hole chemistry}
 
\chapter{Acceleration and the C-metric}
\label{chap:cmet}

\section{Origins of the C-metric}
\label{sec:cmet-history}

The first non-trivial solution to the Einstein field equations of General
Relativity appeared in January 1916, little over a month after their initial
publication in late November 1915~\cite{Einstein:1915aa,Einstein:1916aa}. It was
Karl Schwarzschild who initially discovered the spacetime and presented it as
the most general solution to describe a spherically-symmetric solution to the
vacuum field equations~\cite{Schwarzschild:1916ab, Schwarzschild:1916aa,
  Schwarzschild:1916ea}. This solution went largely misunderstood for the next
few decades as various attempts were made to understand the singularities the
metric contains. It wasn't until 1958 that David Finkelstein, using a set of
coordinates earlier discovered by Arthur Eddington that smoothly patched over the
coordinate singularity, provided the interpretation of this singularity as a
surface which could only be traversed in one
direction~\cite{Finkelstein:1958aa}. The spacetime eventually went on to become
known as the Schwarszchild black hole, however it is easy to forget how much
time elapsed between the discovery of the solution and for a widely accepted
interpretation to be presented.

Within the first few years of the field equations being known, other solutions
were found. These were mostly mathematical solutions whose physical meanings
were unknown, much like the Schwarzschild metric. Of particular interest to this
thesis is a metric belonging to a larger class of solutions discovered by
Levi-Civita in 1918~\cite{Levi-Civita:1918aa}.

This metric would then be rediscovered in the early 1960s, in a wave of research
presumably inspired by the aforementioned understanding of black holes, by
Newman and Tamburino~\cite{Newman:1961aa}, Robinson and
Trautman~\cite{Robinson:1962zz} and then again by Ehlers and
Kundt~\cite{Ehlers:1962aa}. In particular, Ehlers and Kundt explicitly take the
initial body of work laid out by Levi-Civita, cited above, and, in their own
words, ``follow his line of thought'' and ``simplified and completed his
derivations''. They continue on to present an invariant classification of
degenerate static vacuum fields. Without delving into the details, it is this
classification of ``A'', ``B'' and ``C''-type metrics which gives its name, to
this day, to the \emph{C-metric}, the axisymmetric vacuum solution describing an
accelerating black hole.

At that point in time, the C-metric was not yet fully understood, but it was
seen as a metric that bridges the gap between the Schwarzschild metric and its
charged counterpart (Reissner-Nordstr\"om), and the Weyl and Robinson-Trautman
solutions~\cite{Kinnersley:1970zw}.

Eventually, in 1970, Kinnersley and Walker (KW)~\cite{Kinnersley:1970zw} picked
up the C-metric and finally provided the interpretation of this solution as an
\emph{accelerated} black hole. Let us now review how KW obtained their version
of the C-metric. The Levi-Civita solution is given in the form
\begin{align}
  \label{eq:levi-civita-metric}
  ds^2 = \frac{1}{(x+y)^2} \left(-F(y)dt^2 + \frac{dy^2}{F(y)} + \frac{dx^2}{G(x)}
  + G(x) dz^2\right)
\end{align}
where
\begin{align}
  G(x) = a_0 + a_1 x + a_2 x^2 + a_3 x^3 \qquad \mbox{and} \qquad F(y) = -G(-y),
\end{align}
with $a_0\ldots a_3$ some a set of parametrising constants. The metric admits
a conformal transformation given by
\begin{gather}
  \label{eq:cmet-transf}
  x\to \eta\beta x + \varepsilon, \quad y \to \eta\beta y - \varepsilon, \quad t \to \eta t \quad \mbox{and} \quad
  z\to \eta z
\end{gather}
as long as $G(x) \to q G(x)$, which imposes the following relations between the
coefficients of $G$ before (denoted by a tilde $\tilde{}~$) and after the
transformation:
\begin{subequations}
  \label{eq:cmet-coef}
  \begin{align}
    \beta a_0 &= \ta_0 + \ta_1 \varepsilon+ \ta_ 2 \delta^2 + \ta_3 \varepsilon^3, \label{eq:cmet-coef1}\\
    a_1 &= \eta \ta_1 + 2\eta\ta_2 \varepsilon + 3 \eta\ta_3 \varepsilon^2, \label{eq:cmet-coef2}\\
    a_2 &= \eta^2\beta\ta_2 + 3 \eta^2\beta\ta_3\varepsilon, \label{eq:cmet-coef3}\\
    a_3 &= \eta^3\beta^2\ta_3.\label{eq:cmet-coef4}
  \end{align}
\end{subequations}
This symmetry allows us to remove two degrees of freedom from the initial
solution, and it is this freedom of parametrisation which has led to some
confusion as to its interpretation. The final setup that KW present involves
setting $a_1 = 0$, $a_0 = -a_2 = 1$ and then labelling $\beta = A^2$ and
$a_3 = -2 m A$. The final form (including a charged term, introduced later) of
the C-metric, as written by Kinnersley and Walker is therefore given by
\begin{align}
  \label{eq:kinnersley-walker-metric}
  ds^2 = \frac{1}{A^2(x+y)^2} \left(-F(y)dt^2 + \frac{dy^2}{F(y)} +
  \frac{dx^2}{G(x)} + G(x) dz^2\right)
\end{align}
where
\begin{align}
  G(x) = 1-x^2-2mAx^3-e^2 A^2 x^4\qquad \mbox{and} \qquad F(y) = -G(-y).
\end{align}
This paper was published around the time Kinnersley published his doctoral
thesis, in which he also shows that the charged C-metric is simply obtained by
adding in the quartic term above, along with the gauge potential $B = -eAydt$.
For now the parameters $m$ and $A$ can just be interpreted as the two remaining
degrees of freedom for this solution, however it will become apparent that these
can be thought of as (being related to) the mass and the acceleration of a
massive particle/black hole respectively. The Schwarzschild and
Reissner-Nordstr\"om spacetimes are then recovered by first changing $y = 1/Ar$
and $t\to At$, taking the limit as $A \to 0$ and then identifying
$x = \cos\theta$ and $z = \phi$.

In order to understand the accelerating nature of this solution, it will be
useful to first briefly review Rindler coordinates, which are simply set of
coordinates well-suited to uniformly accelerating trajectories. To see what this
means, let's consider the path of an uniformly accelerating object in flat
1+1-dimensional space. The accelerations as measured locally by the object,
$\alpha$, and as measured in a static lab frame, $a$, are related by
\begin{equation}
  \alpha = \gamma^3a,
\end{equation}
where $\gamma = (1-v^2)^{-1/2}$ is the usual Lorentz factor, with $v$ the
instantaneous velocity of the object as measured by the static observer. Solving
this equation for its position, denoted by $x$, reveals that the path taken by
an accelerating observer satisfies
\begin{equation}
  -t^2+x^2=\frac{1}{\alpha^2}.
\end{equation}
This is known as hyperbolic motion, and we have the following parametrisation:
\begin{align}
  \label{eq:trajectory}
  x &= \frac{1}{\alpha} \cosh \alpha \tau,\nn\\
  t &= \frac{1}{\alpha} \sinh \alpha \tau.
\end{align}
We may use this parametrisation to obtain a set of coordinates which is centred
on the idea of accelerated objects, in other words, coordinates for which paths
at constant coordinate are these hyperbolic trajectories. Such a transformation
is given by:
\begin{align}
  x = \frac{\xi}{\alpha} \cosh \alpha \tau,\qquad t = \frac{\xi}{\alpha} \sinh
  \alpha \tau.
\end{align}
This transformation gives rise to what is known as the \emph{Rindler metric},
\begin{align}
   ds^2 = -\xi^2 d\tau^2 + \frac{d\xi^2}{\alpha^2}.
\end{align}
These coordinates cover the region $t^2<x^2$. This is due to the presence of an
\emph{acceleration horizon} at $t = \pm x$, or $\xi = 0$. The analytic
continuation over the horizon is simply done by reverting back to the original
cartesian coordinates.

Let us now return to the C-metric and its interpretation. To see the
accelerating nature of the metric, we will work from its flat-space limit, which
we obtain from \cref{eq:kinnersley-walker-metric} by setting $m=e=0$. The metric
is now
\begin{align}
  \label{eq:flatlimit}
  ds^2=\frac{1}{A^2(x+y)^2}\left(-(y^2-1)A^2dt^2+\frac{dy^2}{y^2-1} +
  \frac{dx^2}{1-x^2} + (1-x^2)dz^2\right).
\end{align}
Using the following nontrivial coordinate transformation:
\begin{align}
  \xi = \frac{\sqrt{y^2-1}}{x + y}, \qquad \rho =
  \frac{1}{A}\frac{\sqrt{1-x^2}}{x+y}, \qquad t=A\tau, \qquad z=\varphi,
\end{align}
we are able to recover the Rindler metric in 3+1 dimsensional cylindrical
coordinates:
\begin{align}
  \label{eq:rindler}
  ds^2 = -\xi^2 d\tau^2 + \frac{d\xi^2}{A^2} + d\rho^2 + \rho^2 d\varphi^2.
\end{align}
This reinforces the interpretation of $A$ as the acceleration parameter. The
acceleration horizon, $\xi = 0$ is therefore located, in the original
coordinates, at y = 1. As mentioned above, it is possible too to write the
C-metric in a pseudo-spherical coordinate system with a radial coordinate $r$,
related to y through the substitution $y = 1/Ar$, and we find that the
acceleration horizon is located at $r = 1/A$. Intuitively, therefore, it makes
sense that the acceleration horizon be located farther away with vanishing
acceleration. Similarly, we also have that the origin of this spherical system,
$r = 0$, which corresponds to $\xi = 1$, follows the trajectory
\eqref{eq:trajectory} that our initial accelerating object did.


\section{The Pleba\'nski-Demia\'nski Metric}
\label{sec:PD}

The Pleba\'nski-Demia\'nski (PD) family of solutions~\cite{Plebanski:1976gy} was
published in 1976, and was described as a new class of stationary and
axisymmetric solutions to the Einstein-Maxwell-$\Lambda$ field equations. In
their paper, PD show how, through a series of different coordinate
transformations, one could obtain from their metric most of the known black hole
solutions, including, as we shall see below, the accelerating C-metric as well
as the rotating Kerr metric. Additionally, the solution also accounted for
electromagnetic fields, a cosmological constant $\Lambda$ and even contains a
parameter which in certain limits can be identified as the NUT charge. In the
sections below, we will demonstrate how one recovers the C-metric from this
solution, how the Kerr metric is also obtained from this parent solution, and
finally, we shall present modern modifications to the metric that have vastly
simplified calculations involving the generalised (rotating, charged)
C-metric. Before that however, let us give the the Pleba\'nski-Demia\'nski
metric\footnote{Compared to the original text, we have taken $p$, $q$ and
  $\sigma$ to have opposite signs, for consistency.}:
\begin{align}
  \label{eq:PD}
  ds^2 &= \frac{1}{(p + q)^2} \left( - \frac{Q(q)}{1+(pq)^2}(d\tau + p^2
         d\sigma)^2 + \frac{1+(pq)^2}{Q(q)}dq^2 \right.\nn\\
  &\hspace{10em}\left.+ \frac{1+(pq)^2}{P(p)}dp^2 +
         \frac{P(p)}{1+(pq)^2}(d\sigma - q^2 d\tau)^2 \right), 
\end{align}
where
\begin{align}
  Q &= -\frac{\Lambda}{6} + g^2 - \gamma - 2nq + \epsilon q^2 -
      2mq^3 + \left(-\frac{\Lambda}{6}+e^2+\gamma\right)q^4,\nn\\
  P &= -\frac{\Lambda}{6} - g^2 + \gamma - 2np - \epsilon p^2 -
      2mp^3 + \left(-\frac{\Lambda}{6}-e^2-\gamma\right)p^4,
\end{align}
where $m$, $n$, $e$, $g$, $\epsilon$ and $\gamma$ are real parameters of the
solution. Additionally, it is worth drawing attention to the presence of a
cosmological constant term, as this term can be carried through the derivation of
the C-metric, yielding an accelerating black hole with a nonzero
$\Lambda$. While unfamiliar, the naming choice for these coordinates facilitates
an agnostic treatment of this solution. To simplify the derivations below, we
perform the parametric shift $\gamma\to \gamma + g^2 +\frac{\Lambda}{6}$. The
metric functions are now given by
\begin{align}
  \label{eq:PDfn}
  Q &= -\frac{\Lambda}{3} - \gamma - 2nq + \epsilon q^2 -
      2mq^3 + \left(\gamma + e^2 + g^2\right)q^4,\nn\\
  P &=  \gamma - 2np - \epsilon p^2 -
      2mp^3 - \left(\gamma+e^2+g^2+\frac{\Lambda}{3}\right)p^4.
\end{align}
Since the work presented in this thesis focusses on asymptotically AdS
spacetimes, we shall write
\begin{align*}
  \Lambda=-\frac{3}{\ell^2}.
\end{align*}

\subsection{From \PD{} to the C-metric}
\label{sec:PDtoC}

In order to show how the \PD{} metric contracts to the C-metric, we must first
perform the following rescaling of our coordinates:
\begin{align}
  p\to\sqrt{aA}p, \qquad q\to\sqrt{aA} q, \qquad
  \tau\to\sqrt{\frac{a}{A^3}}\tau, \qquad \sigma\to\sqrt{\frac{a}{A^3}}\sigma,
\end{align}
where we have introduced new nonzero parameters $a$ and $A$ in anticipation of
what follows. To compensate for this rescaling, we may simplify the metric with
the following parameter/function rescalings~\cite{Plebanski:1976gy,Pravda:2002kj}:
\begin{gather}
  m\to \left(\frac{A}{a}\right)^\frac32 m, \qquad n\to
  \left(\frac{A}{a}\right)^\frac12 n, \qquad e\to\frac{A}{a}e, \qquad
  g\to\frac{A}{a}g,\nn\\
  \epsilon\to \frac{A}{a}\epsilon, \qquad \gamma\to A^2\gamma, \qquad P\to A^2P,
  \qquad Q\to A^2 Q.
\end{gather}
The resulting metric is given by
\begin{align}
  \label{eq:PDrescaled}
  ds^2 &= \frac{1}{A^2(p + q)^2} \left( - \frac{Q(q)}{1+(aApq)^2}(d\tau + aAp^2
         d\sigma)^2 + \frac{1+(aApq)^2}{Q(q)}dq^2 \right.\nn\\
  &\hspace{8em}\left. + \frac{1+(aApq)^2}{P(p)}dp^2 +
         \frac{P(p)}{1+(aApq)^2}(d\sigma - aAq^2 d\tau)^2 \right),
\end{align}
with
\begin{align}
  \label{eq:PDfnrescaled}
  Q &= \frac{1}{A^2\ell^2} - \gamma - \frac{2nq}{A} + \epsilon q^2 -
      2mAq^3 + A^2\left(\gamma a^2 + e^2 + g^2\right)q^4,\nn\\
  P &=  \gamma - \frac{2np}{A} - \epsilon p^2 -
      2mAp^3 + \left(\frac{a^2}{\ell^2} - A^2(\gamma a^2+e^2+g^2)\right)p^4.
\end{align}
Now, according to~\cite{Plebanski:1976gy}, the curvature invariants of this
spacetime do not depend on the parameters $\epsilon$, $n$ and $\gamma$. This
tells us that these parameters are simple gauge choices, and we may fix them
without affecting the geometry. To recover the C-metric, we first take the limit
$a\to 0$, after which we are free to set $\epsilon=\gamma=1$ and $n=0$. The
C-metric, as written in \cref{eq:kinnersley-walker-metric}, is then simply
obtained by identifying $\tau=t$, $q=y$, $p=x$ and $\sigma=z$. The metric
\eqref{eq:PDrescaled} is actually known as the spinning C-metric (SC-metric),
as further
work~\cite{Farhoosh:1980zz,Bicak:1999sa,Letelier:1998rx,Pravda:2002kj} showed
that while not analytically pleasant, this metric displays characteristics of a
rotating black hole while preserving its acceleratng nature.

\subsection{From \PD{} to the Kerr-Newman Metric}
\label{sec:PDtoKN}

The rotating nature of the metric~\eqref{eq:PDrescaled} becomes apparent when we
switch to Boyer-Lindquist-type, or pseudo-spherical, coordinates. The following
transformation:
\begin{gather}
  \tau\to A\left(\tau-\frac{a\sigma}{\Xi}\right), \qquad q \to \frac{1}{Ar},
  \qquad \sigma \to \frac{\sigma}{\Xi}\\
  \mbox{with} \quad Q\to
  \frac{Q}{A^2r^4}, \quad \mbox{and} \quad
  \Xi=1-\frac{a^2}{\ell^2}+\mathcal{O}(A),
  \label{eq:PDtoBL}
\end{gather}
where we have included the possibility for a term in $\Xi$ which may depend on
acceleration but must vanish in its absence, produces the following metric:
\begin{align}
  ds^2 &= \frac{1}{(1+Arp)^2}\left( -\frac{Q(r)}{r^2+a^2p^2}\Big(d\tau -
         a(1-p^2) \frac{d\sigma}{\Xi} \Big)^2 + \frac{r^2+a^2p^2}{Q(r)}dr^2
         \right.\nn\\
  &\hspace{8em} \left. + \frac{r^2+a^2p^2}{P(p)}dp^2
  + \frac{P(p)}{r^2+a^2p^2} \Big(ad\tau-(r^2+a^2)
    \frac{d\sigma}{\Xi}\Big)^2\right),
\end{align}
with
\begin{align}
  \label{eq:PDfnrescaledKN}
  Q &= (\gamma a^2+e^2 + g^2) - 2mr + \epsilon r^2 - 2nr^3 + r^4\left(\frac{1}{\ell^2} 
      - A^2\gamma\right),\nn\\ 
  P &=  \gamma - \frac{2np}{A} - \epsilon p^2 -
      2mAp^3 + \left(\frac{a^2}{\ell^2} - A^2(\gamma a^2+e^2+g^2)\right)p^4.
\end{align}
The physical parameters of this solution are $m$, $n$, $e$ and $g$, with
$\gamma$ and $\epsilon$ absent from curvature invariants. Unfortunately, this
metric does not present a convenient way of writing $p=p(\theta)$ unless we set
$A=0$. The reason for this is that ordinarily, one would write $p=\cos{\theta}$,
however for terms to cancel out neatly, we would expect to be able to factorise
$P$ such that a $\sin^2\theta$ piece could be pulled out. For $A=0$, we are able
to do this, provided we are able to set $n=0$ and $\gamma=1$ as before, and this
time $\epsilon=1 + a^2/\ell^2$, which gives the Kerr-Newman-AdS metric. In fact,
in this limit, the parameter $n$ was identified in~\cite{Plebanski:1976gy} as
the NUT charge.

\subsection{The Factorised C-metric}
\label{sec:factorised}

The issue that prevented us in the previous section from expressing the C-metric
in pseudo-spherical coordinates was that we are unable to factorise the metric
functions for the transformation to be convincing. This was an ugly, but
well-known, symptom in C-metric calculations that results from the high order
polynomials which make up its components. Most computations with this metric
will generically revolve around the coordinate ranges, which are dictated by the
metric functions and their root configuration. An immediate example is the range
of the azimuthal coordinate --- covered in \cref{sec:conical} --- which, if a
regularity condition at one of the poles is imposed, depends on one of the roots
of $P$($\equiv G$).

\subsubsection{Without rotation}
Inspired by~\cite{Dowker:2001dg}, in 2003, Hong and Teo (HT) realised that they
could use the symmetry in \cref{eq:cmet-transf,eq:cmet-coef} to re-express the
metric in a way that simplified calculations~\cite{Hong:2003gx}. HT realised
that by re-tuning this set of function coefficients --- those that Kinnersley
and Walker picked to provide the original interpretation, they could express
these polynomials in a factorised form with simple roots. In the uncharged case,
and using the same notation, this was achieved by setting $a_0 = -a_2 = 1$ as
before and $a_1 = -a_3 = 2 m A$. The C-metric in Hong-Teo form is then still
given by \cref{eq:kinnersley-walker-metric}, however the metric functions now
factorise nicely:
\begin{align}
  \label{eq:HTFandG}
  G(x) = (1-x^2)(1+2mA x)\qquad \mbox{and} \qquad F(y) = -G(-y).
\end{align}
It is important to stress that the constants $m$ and $A$ used here are not the
same as those in the KW form of the metric. The relation between the two metrics
and their parameters can be found by simply applying the coordinate
transformation~\eqref{eq:cmet-transf} between the two parameter spaces. For the
original metric, we had $\ta_0 = -\ta_2 = 1$, $\ta_1=0$ and
$\ta_3=-2\tm\tA$. To preserve the metric
\eqref{eq:kinnersley-walker-metric}, we see that the parameter $\beta$ of the
transformation must be
$\beta=A^2/\tA^2$. \Cref{eq:cmet-coef1,eq:cmet-coef2,eq:cmet-coef3} provide an
initial set of relations:
\begin{subequations}
  \label{eq:cmet-KW-to-HT}
  \begin{align}
    \beta &= 1- \varepsilon^2 - 2\tm \tA = \frac{A^2}{\tA^2}, \\
    mA&= -\eta \varepsilon (1+3\tm\tA \varepsilon),\\
    \eta &= \beta^{-\frac12}(1 + 6\tm\tA \varepsilon)^{-\frac12},
  \end{align}
  and \cref{eq:cmet-coef4} imposes the following condition on $\varepsilon$:
  \begin{align}
    \tm\tA + \varepsilon + 8\tm\tA \varepsilon^2 + 16\tm^2\tA^2 \varepsilon^3 = 0. \label{eq:KWtoHT4}
  \end{align}
\end{subequations}
As we show in \cref{sec:parameters}, there is an upper bound on $m A$ for this
spacetime. In the Kinnersley-Walker parametrisation of the C-metric,
$0\leqslant\tm\tA\leqslant 1/(3\sqrt{3})$ and in this regime, the above
constraint~\eqref{eq:KWtoHT4} is only satisfied for one value of $\varepsilon$,
which can in turn be used to resolve the remainder of the transformation. As it
is unpleasantly apparent, there is a sense in which the complicated root
structure of the KW parametrisation has been shifted and swept into the
parameters themselves.

\subsubsection{With rotation}
Following on from their work on the non-rotating C-metric, a year later, Hong
and Teo published a similarly factorised version of the C-metric which contained
a rotation parameter~\cite{Hong:2004dm}. Unlike the spinning C-metric, however,
due to the factorised nature of this new metric, not only is it more pleasant to
work with, but it may also be fully written in Boyer-Lindquist-type coordinates,
in such a way that either the non-rotating (charged, AdS) C-metric or the
Kerr-Newman-AdS metric may be obtained simply by turning off their respective
parameters.

In order to obtain such a metric, the starting point is different to the
non-rotating case. Rather than search for a convenient coordinate
transformation, HT utilised a top-down approach starting from the \PD{}
metric~\eqref{eq:PD}. In fact, the derivation requires making the same rescaling
as we did earlier, therefore we will pick up from \cref{eq:PDrescaled}, the
metric, and \cref{eq:PDfnrescaled}, its functions. For convenience, the latter
are given by
\begin{align}
  Q &= \frac{1}{A^2\ell^2} - \gamma - \frac{2nq}{A} + \epsilon q^2 -
      2mAq^3 + A^2\left(\gamma a^2 + e^2 + g^2\right)q^4,\nn\\
  P &=  \gamma - \frac{2np}{A} - \epsilon p^2 -
      2mAp^3 + \left(\frac{a^2}{\ell^2} - A^2(\gamma a^2+e^2+g^2)\right)p^4.
\end{align}

The gauge freedom in this solution is such that we are free to pick $\gamma$ and
$\epsilon$ without affecting the physical geometry of the spacetime. Indeed,
earlier this was used to recover the non-rotating C-metric in the KW
parametrisation. Additionally, the parameter $n$ can be related to the NUT
charge $l$~\cite{Griffiths:2006tk,Griffiths:2005qp}. The factorised metric is
obtained by the requirement that the NUT charge vanish, which sets $n =
-mA^2$. We then use the aforementioned gauge freedom to set
\begin{align*}
  \gamma = 1, \qquad \epsilon = 1+\frac{a^2}{\ell^2}-A^2(a^2+e^2+g^2).
\end{align*}
With these choices, we have the following factorised metric functions,
all the while preserving the form of the rescaled \PD{} metric~\eqref{eq:PDrescaled}:
\begin{align}
  \label{eq:PDfnfactorised}
  Q &= \frac{1}{A^2\ell^2} (1 + a^2A^2q^2) + (q^2 - 1)(1 - 2mAq + A^2(a^2 + e^2
      + g^2) q^2),\nn\\
  P &= (1 - p^2) \left(1 + 2mAp + \left(A^2 (a^2 + e^2 + g^2) -
      \frac{a^2}{\ell^2}\right) p^2\right).
\end{align}
The non-rotating C-metric that HT first presented is recovered here for
$\Lambda = a = 0$. The factorised nature of this solution allows us to write it
in Boyer-Lindquist coordinates with the following similar transformation to the
one we used to derive the Kerr-AdS metric~\eqref{eq:PDtoBL}:
\begin{gather}
  \tau\to A\left(t-\frac{a\phi}{K}\right), \qquad q \to \frac{1}{Ar}, \qquad p
  \to \cos\theta, \qquad \sigma \to \frac{\phi}{K}.
\end{gather}
The resulting metric is what we shall henceforth refer to as the
\emph{generalised C-metric}, which we will use in future chapters. It is given
by
\begin{subequations}
  \label{eq:PDfactrth}
  \begin{align}
    \label{eq:PDfactrth-metric}
    ds^2 &= \frac{1}{\Omega^2}\bigg\{ -\frac{f(r)}{\Sigma}\Big[dt - a\sin^2\theta
           \frac{d\phi}{K} \Big]^2 + \frac{\Sigma}{f(r)}dr^2 + \frac{\Sigma
           r^2}{g(\theta)}d\theta^2 \nn\\
         &\hspace{12em} + \frac{g(\theta) \sin^2\theta}{\Sigma r^2} \Big[adt-(r^2+a^2)
           \frac{d\phi}{K}\Big]^2\bigg\},
  \end{align}
  and the metric functions are 
  \begin{gather}
    f(r) = (1 - A^2r^2) \bigg[1 - \frac{2m}{r} + \frac{a^2 + e^2+g^2}{r^2}\bigg] +
    \frac{r^2 + a^2}{\ell^2}, \nn\\
    g(\theta) = 1 + 2mA\cos\theta + \bigg[A^2 (a^2 + e^2+g^2) - \frac{a^2}{\ell^2}\bigg]
    \cos^2\theta, \nn\\ 
    \Sigma=1+\frac{a^2}{r^2}\cos^2\theta, \qquad \Omega = 1+Ar\cos\theta.
    \label{eq:PDfactrth-fn}
  \end{gather}
\end{subequations}
The parameter $K$ that we have introduced allows us to define $\phi$ such that
it has a $2\pi$-periodicity. We will discuss its role extensively in further
sections; let it therefore simply be said for now that it affects the
distribution of conical defects in the spacetime.


\section{Reviewing the C-metric and its Features}
\label{sec:cmet-review}

So far, we have seen how the C-metric and its interpretation were developed and
understood better. We will now dive in a little deeper and present some subtle
aspects of the solution, such as horizon structure, coordinate and parameter
ranges as well as clarifying this idea of having a conical defect along one
axis. We will begin with the non-rotating form for simplicity, before covering
the generalised C-metric.

\subsection{The Non-Rotating C-metric}
We begin with the asymptotically AdS form of the C-metric. In
asymptotically flat space, the C-metric describes a configuration of two black
holes accelerating in opposite directions. Each black hole has unequal conical
deficits extending from the north and south poles of each event horizon to
either the boundary or an acceleration horizon that separates the two. The
introduction of a cosmological constant, as we shall see, alters the picture if
it is large enough. In that case, the solution describes only one such black
hole without an acceleration horizon, and both deficits extend out to the
boundary. Although the C-metric is well-known among relativists,
there are features of the specific form we will be using that are worth
highlighting, discussing how they depend on the parameters of the solution.

For convenience we rewrite the non-rotating charged AdS
C-metric~\cite{Hong:2003gx,Hong:2004dm}:
\begin{subequations}
  \label{eq:NRyx}
  \begin{align}
    \label{eq:NRyx-metric}
    ds^2 = \frac{1}{A^2(x+y)^2} \left(-F(y)dt^2 + \frac{dy^2}{F(y)} +
    \frac{dx^2}{G(x)} + G(x) dz^2\right),
  \end{align}
  and its metric functions
  \begin{align}
    \label{eq:NRyx-fn}
    F(y)&=\frac{1}{A^2 \ell^2} + (y^2 - 1)(1-2mAy+e^2A^2y^2),\nn\\
    G(x)&=(1-x^2)(1+2mAx+e^2A^2x^2).
  \end{align}
\end{subequations}
The factorised metric functions allow us to write this metric also in
pseudo-spherical coordinates given by
\begin{gather}
  t \to At, \qquad y \to \frac{1}{Ar}, \qquad x \to \cos\theta, \qquad z \to
  \frac{\phi}{K}.
\end{gather}
The resulting metric is:
\begin{align}
\label{eq:NRrth-metric}
ds^2=\frac{1}{\Omega^2}\left[ 
f(r) dt^2
-\frac{dr^2}{f(r)} - r^2 \Big( \frac{d\theta^2}{g(\theta)} 
+ g(\theta)\sin^2\theta \frac{d\phi^2}{K^2}\Big)\right].
\end{align}
The conformal factor
\begin{align}\label{eq:conf}
\Omega=1+Ar\cos\theta
\end{align}
sets the location of the boundary at $r_\bd=-1/A\cos\theta$. The other metric
functions are given as
\begin{align}
f(r)&=(1-A^2r^2)\Bigl(1-\frac{2 m}{r}+\frac{e^2}{r^2}\Bigr)
+\frac{r^2}{\ell^2},\nn\\
g(\theta)&=1+2mA \cos\theta+ e^2A^2 \cos^2\theta.
\label{eq:NRrth-metric}
\end{align}
The remaining parameters, $e$, $m$, $A\geqslant 0$ are related to the charge,
mass and acceleration of the black hole. In the following sections we will
discuss these coordinates and their ranges and review various constraints we
must impose on these parameters. We will then also briefly discuss the existence
of conical defects in the spacetime before tackling the rotating solution.

\subsection{Coordinate Ranges and Parametric Restrictions}
\label{sec:parameters}

While easier to interpret, Boyer-Lindquist coordinates are not the best suited
for the parametric analysis we will present. This is due to the location of the
boundary which is shifted from its usual location at $r=\infty$. The
interpretation is that for certain values of $\theta$ the boundary is closer
than infinity, and that for others it is in fact \emph{beyond} infinity, in the
sense that we must glue together the regions $0<r<\infty$ and
$-\infty < r < r_\mathrm{bd}$. Rather, it is more appropriate to be using the
$(t,y,x,z)$ coordinate system.

The premise we will be basing our analysis on is that we will require the
signature of the metric be preserved over the coordinate ranges allowed. Since
the metric depends on neither $t$ nor $z$, it follows that we simply require
these two coordinates be real. We have also encountered the fact that $z$
behaves like a special angular coordinate, and therefore should have some
periodicity $\Delta z$ attributed to it. This will be the subject of the next
section. The Kretschmann scalar
\begin{align}
  R_{\alpha\beta\gamma\delta}R^{\alpha\beta\gamma\delta} = \frac{24}{\ell^4} +
  (x+y)^6\left(48A^6m^2+\mathcal{O}(e^2)\right)
\end{align}
reveals curvature singularities at $|x|,|y|\to\infty$. We have also seen that
the Hong-Teo~\cite{Hong:2003gx} factorisation allows us to map
$x=\cos\theta$. It therefore makes sense that $x \in [-1,1]$, and we must
require that $G(x)$ be positive over this range. This condition will be
satisfied provided
\begin{align}
  \label{eq:conditionsNR}
  e^2A^2 >
  \begin{cases}
    2mA - 1 \qquad &\mbox{if} \quad mA \leqslant 1,\\
    m^2A^2 \qquad &\mbox{if} \quad mA > 1.
  \end{cases}
\end{align}
In the absence of charge, this reduces to the simple requirement that
$2mA\leqslant 1$. It is for this reason that one may view this condition as an
upper bound on acceleration, however this is not so for larger values of charge
$e$. The set of parameters excluded by \cref{eq:conditionsNR} is reproduced as
blue hatching in \cref{fig:param-regionsNR}. By this condition, configurations
with larger values of $A$ would need to be charged.

\begin{figure}
  \centering
  \includegraphics{eA-m1p5a0p0-large}
  \caption{\label{fig:param-regionsNR}Parametric space for the non-rotating C-metric
    at $m=1.5\ell$. The blue hatched region corresponds to solutions excluded by
    \cref{eq:conditionsNR}. The region above the solid black line is excluded
    via cosmic censorship. The dashed black line corresponds to
    $A_\mathrm{crit}$, and separates solutions which have an acceleration
    horizon to the left, and those that don't to the right. The red lines
    delimit parametric regions corresponding to spacetimes with additional
    horizons that intersect the boundary.}
\end{figure}

We have also seen that $y$ behaves like an inverse radial coordinate, and the
zeros of $F(y)$ therefore correspond to horizons. From the
metric~\eqref{eq:NRyx-metric}, the boundary is located at $y_\mathrm{bd} =
-x$. This distinguishes two possible regions for $y$, however, for the positive
parameter configurations we are restricting ourselves to, the region $y<-x$ will
always have a naked singularity at $y \to -\infty$. With charge, generic
configurations will always have 0, 2, or 4 distinct horizons. Physically, we
have a pair of inner and outer horizons similar to those that characterise the
regular Reissner-Nordstr\"om (RN) solution which typically approach one another
and vanish with larger charge. We also have an acceleration horizon inherent to
these accelerating spacetimes. When an acceleration horizon is present, there is
a second ``outer'' acceleration horizon, and both of these intersect with the
boundary. Pairs of horizons divide the spacetime into regions which share the
same signature; a region (i) within the innermost RN horizon which is excluded
through cosmic censorship, a region (ii) between the outer RN horizon and the
regular acceleration horizon, which we are most interested in, and finally a
region (iii) beyond the outermost acceleration horizon which is of little
interest to the work we shall be presenting. In \cref{fig:param-regionsNR}, each
line (other than the blue line) delimits configurations with different numbers
of horizons. First, the dashed line corresponds to values of $A$ beyond which
there is an acceleration horizon, the solid black line corresponds to the
extremal limit above which we have a naked singularity in region (ii). The space
between the red lines also has four horizons; another pair of inner and outer
horizons forms in region (ii), however they too intersect the boundary. This
pair further subdivides region (ii), however neither region is of interest as in
one we have a curvature singularity visible to the boundary and the other is
similar to region (iii). At the uppermost red line in
\cref{fig:param-regionsNR}, the new intermediary region is ``absorbed'' by the
two horizons that bound it, and we are left with two horizons. These nuances
were explored in~\cite{Hubeny:2009kz}, where many of these horizons were sorted
into two categories of horizons: black \emph{funnels} --- these are the bulk
duals of black holes strongly coupled to a field theory plasma, and black
\emph{droplets}, which are dual to weakly coupled black holes.

\subsection{The Conical Defect}
\label{sec:conical}

A conical deficit $\delta = 2\pi - \Delta \varphi$ (here $\varphi$ is a generic
azimuthal coordinate) is associated to the presence of a cosmic string with
tension $\mu = \delta/8\pi$. One may introduce such defects to familiar
spacetimes. For example, one may write the Schwarzschild metric, only this time
with $g^{\mathrm{Sch}}_{\phi\phi}= r^2\sin^2\theta K^{-2}$. For $K>1$, the
result is a black hole with a string running through its
core~\cite{Aryal:1986sz}. The deficit along both the $\theta = 0$ and
$\theta= \pi$ axes is the same, and the tension of the string is
$\mu=\frac14(1-K^{-1})$. The C-metric has unequal deficits, and the resulting
string tension imbalance is what physically drives the acceleration.

The conical defect inherent to the C-metric is controlled through the
periodicity of $z$. For simplicity, we pick a new coordinate $\phi = K z$ such
that its periodicity $\Delta\phi=2\pi$, and the choice of $K$ replaces the
choice of $\Delta z$. We choose to do this (a) for familiarity and (b) so that
this apparent degree of freedom is explicit in computations. The angular part of
the metric in \cref{eq:NRrth-metric} near the poles is
\begin{align}
ds^2 \sim \frac{1}{\Omega^2}\frac{r^2}{g(\theta_{\pm})}\bigg[d\rho^2 +
  \frac{g^2(\theta_{\pm}) \rho^{2}}{K^2}d\phi^2\bigg]. 
\end{align}
where $\theta = \theta_{\pm} \pm \rho$ with $\theta_+=0$ and $\theta_-=\pi$. The
conical deficits along each axis are given by 
\begin{align}
  \label{eq:deficitsNR}
  \delta_\pm=2\pi\bigg[1-\frac{g(\theta_\pm)}{K}\bigg] = 2\pi\bigg[1-\frac{1\pm
  2mA + e^2A^2}{K}\bigg]. 
\end{align} 
The tensions of cosmic strings connecting the event horizon to the boundary (or
acceleration horizon, if there is one) are related to the deficits, and given by
$\mu_\pm=\delta_\pm/8\pi$. It is now evident how our choice of $K$ will impact
the geometry of the spacetime; specifically, along with $A$, it will regulate
the distribution of tensions along either axis. It is also worth mentioning that
a negative deficit (corresponding to an excess) is possible, however this would
be sourced by a negative energy object. We can remove one of the deficits for
the C-metric by defining $K=L_\pm$ where
\begin{align}
L_\pm=1\pm 2mA + e^2A^2.
\end{align}
Since having $K=L_-$ induces an excess at the other pole, it is generally the
custom to have $K=L_+$, regularising the north pole, and only having a string at
the south pole. In \cref{fig:embed3D} we illustrate how changing $K$ affects an
embedding of the horizon in $\mathbb{E}^3$. 

\begin{figure}
  \centering
  \begin{subfigure}[b]{0.2\textwidth}
    \includegraphics[width=\textwidth]{embed3D-smooth}
    \caption{\label{fig:embed3D-smooth}}
  \end{subfigure}\qquad
  \begin{subfigure}[b]{0.2\textwidth}
    \includegraphics[width=\textwidth]{embed3D-Kbig}
    \caption{\label{fig:embed3D-Kbig}}
  \end{subfigure}
  \caption{\label{fig:embed3D}Embeddings of the non-rotating C-metric in
    $\mathbb{E}^3$, for the black hole with $A=0.01\ell$, $m=9\ell$ and (a) $K=
    L_+$, (b) $K= 1.2 L_+$.}
\end{figure}

\subsection{The Rotating C-metric}
\label{sec:rotating-c-metric}
It will be of interest to extend the observations made in the previous section
to the rotating C-metric. The approaches used in this section will be very
similar to what is presented above, and the results obtained below follow
accordingly. The metric we will be using is the solution we derived earlier from
the Pleba\'nski-Demia\'nski metric~\eqref{eq:PDfactrth},
\begin{subequations}
  \label{eq:cmetrth}
  \begin{align}
    \label{eq:rth-metric}
    ds^2 &= \frac{1}{\Omega^2}\bigg\{ -\frac{f(r)}{\Sigma}\Big[dt - a\sin^2\theta
           \frac{d\phi}{K} \Big]^2 + \frac{\Sigma}{f(r)}dr^2 + \frac{\Sigma
           r^2}{g(\theta)}d\theta^2 \nn\\
         &\hspace{15em} + \frac{g(\theta) \sin^2\theta}{\Sigma r^2} \Big[adt-(r^2+a^2)
           \frac{d\phi}{K}\Big]^2\bigg\},
  \end{align}
  The conformal factor $\Omega^{-2}$ is still given by \cref{eq:conf}, and we have
  the following metric functions:
  \begin{gather}
    f(r)= (1-A^2r^2)\bigg[1-\frac{2m}{r} + \frac{a^2+e^2}{r^2}\bigg] +
    \frac{r^2+a^2}{\ell^2},\nn\\  
    g(\theta) = 1+2mA\cos\theta+\bigg[A^2(a^2+e^2) - \frac{a^2}{\ell^2}\bigg]
    \cos^2\theta,\nn\\  
    \Sigma= 1+\frac{a^2}{r^2}\cos^2\theta. 
    \label{eq:rth-fn}
  \end{gather}
\end{subequations}
It is worth noting that this form of the rotating C-metric, with $A=0$, reduces
to the familiar Kerr-AdS metric written in Boyer-Lindquist coordinates ---
assuming $K$ is picked so as to regularise the poles --- which allows us to
identify $a$ as being a rotation parameter.

With the inclusion of rotation, we have the following parametric restrictions:
\begin{align}
  \label{eq:conditions}
  A^2(a^2+e^2)-\frac{a^2}{\ell^2} <
  \begin{cases}
    2mA - 1 \qquad &\mbox{if} \quad mA \leqslant 1\\
    m^2A^2 \qquad &\mbox{if} \quad mA > 1
  \end{cases}
\end{align}
once again, obtained by requiring that the metric function $g(\theta)$ be
positive over $0\leqslant\theta<\pi$. We reproduce different parametric spaces
in \cref{fig:param-regions} and indicate once again with blue hatching the
regions excluded by \cref{eq:conditions}. As previously stated, we see that for
smaller parameter values, this condition acts as an upper bound on the rotation
and/or the acceleration. We have also included lines delimiting regions which
correspond to solutions with different numbers of horizons. As in the charged
non-rotating case, horizons come in pairs, and a generic configuration will have
0, 2 or 4. The meaning of each of these lines is the same as earlier, with the
dashed line indicating an acceleration horizon and the solid black line
indicating extremal black holes, with grey hatching indicating configurations
ruled out by cosmic censorship. For completion, we have also traced the red
lines which correspond to further horizon pairs coming in, however these do not
concern us. It is worth noting that we have found configurations similar to
\cref{fig:aA-m0p5e0p3} in which the upper bound on rotation is due to
\cref{eq:conditions} and not censorship. Interestingly, for larger mass and
acceleration, these two conditions combine to exclude all possibilities, as in
\cref{fig:ae-m1p6A1p3}. 

\begin{figure}
  \centering
  \begin{subfigure}[b]{0.45\textwidth}
    \includegraphics[width=\textwidth]{aA-m0p5e0p3-small}
    \caption{\label{fig:aA-m0p5e0p3}$m=0.5\ell$, $e=0.3\ell$.}
  \end{subfigure}\qquad
  \begin{subfigure}[b]{0.45\textwidth}
    \includegraphics[width=\textwidth]{aA-m0p9e0p3-small}
    \caption{\label{fig:aA-m0p9e0p3}$m=0.9\ell$, $e=0.3\ell$.}
  \end{subfigure} \\
  \begin{subfigure}[b]{0.45\textwidth}
    \includegraphics[width=\textwidth]{eA-m1p5a0p8-small}
    \caption{\label{fig:eA-m1p5a0p8}$m=1.5\ell$, $a=0.8\ell$.}
  \end{subfigure}\qquad
  \begin{subfigure}[b]{0.45\textwidth}
    \includegraphics[width=\textwidth]{eA-m1p5a2p0-small}
    \caption{\label{fig:eA-m1p5a2p0}$m=1.5\ell$, $a=2.0\ell$.}
  \end{subfigure} \\
  \begin{subfigure}[b]{0.45\textwidth}
    \includegraphics[width=\textwidth]{ae-m0p6A0p5-small}
    \caption{\label{fig:ae-m0p6A0p5}$m=0.6\ell$, $A=0.5\ell^{-1}$.}
  \end{subfigure}\qquad
  \begin{subfigure}[b]{0.45\textwidth}
    \includegraphics[width=\textwidth]{ae-m0p7A1p1-small}
    \caption{\label{fig:ae-m1p6A1p3}$m=0.7\ell$, $A=1.1\ell^{-1}$.}
  \end{subfigure}
  \caption{\label{fig:param-regions}Allowed parametric regions for the
    C-metric. The regions marked out with blue hatching correspond to those
    forbidden by \cref{eq:conditions}, and those marked out in grey are excluded
    by cosmic censorship. The dashed lines correspond to acceleration horizons
    coming in and the red lines outline regions where an extra pair of
    boundary-intersecting horizons are formed.}
\end{figure}

Finally, we may also determine the conical deficits by looking at the region
near each pole. The string tensions for the rotating C-metric are given by
\begin{align}
\mu_\pm = \frac{\delta_\pm}{8\pi}=\frac14\bigg[1-\frac{g(\theta_\pm)}{K}\bigg] =
  \frac14\bigg[1-\frac{1\pm 2mA + A^2(a^2+e^2)-a^2/\ell^2}{K}\bigg], 
\end{align}
and the expressions needed to regularise either pole are
\begin{align}
L_\pm=1\pm 2mA + \bigg[A^2(a^2+e^2)-\frac{a^2}{\ell^2}\bigg].
\end{align}

\begin{center}
  * * *
\end{center}

This brings our analysis of the C-metric to an end. See Dias and Lemos for
analyses of the coordinate ranges/parameter restrictions on the C-metric in AdS
space~\cite{Dias:2002mi}, dS space~\cite{Dias:2003xp} and in special limiting
cases~\cite{Dias:2003up}; Krtou\v s~\cite{Krtous:2005ej} for an analysis of the
maximal extension of the charged AdS C-metric; Hubeny, Rangamani and
Marolf~\cite{Hubeny:2009kz} for a complete overview of the various spacetime
solutions contained in the non-rotating uncharged AdS C-metric for different
choices of coordinate ranges and different parameter values; and Chen, Ng and
Teo~\cite{Chen:2016rjt,Chen:2016jxv} for a recent exhaustive analysis of the
parametric space of the rotating C-metric.


\chapter{Thermodynamics of black holes with conical defects}
\label{chap:td-conical}

\todoruth[inline]{present the work forming the core of this thesis --- how to
  define TD of acc bh. Since these bh contain conical deficits, we must address
  the complication of these for TDs. We begin: by studying bhs with deficits
  (though without acc). There are 2 perspectives one could take, either $\mu$
  changes of it doesn't. Not changing is physically motivated, thinking about
  cosmic strings. Changing allows dynamical processes with capture of
  strings. Introducing acceleration complicates matters, so first hold $\mu$
  fixed. Then vary $\mu$, finding consistent TD. Finally, we describe recent
  work indicating consistency is not a sufficient criteria to obtain TD
  variables, but in fact one must be careful about definition of timelike
  killing vector}

As we have seen so far, the space of solutions which appear to obey the laws of
thermodynamics is rich. Black hole solutions with various conserved charges, may
it be electromagnetic, rotational or NUT, in universes with or without a
cosmological constant and in any number of dimensions have all been shown to
obey the first law once the proper charges and thermodynamic potentials are
correctly identified. Any thermodynamic description requires the existence of an
equilibrium. In asymptotically flat space, equilibrium is achieved by placing
the black hole in a (big) box along with some nondescript fluid surrounding
it. In AdS space, the inherent negative curvature of the background geometry
allows for the entire spacetime to act as the box containing the
equilibrium. From here, one might question whether such a thermodynamic
description might exist for the accelerating solutions we introduced in the
previous chapter. At a glance, one might be tempted to point at the acceleration
as directly preventing the existence of an equilibrium; after all, it must be
driven by some external force. This is a valid concern, however, as we will
show, we can recover the concept of equlibrium for slowly accelerating black
holes in AdS space, by a similar thought process as that for the nonaccelerating
solution.

We have shown, in the previous chapter, how acceleration is driven by the
existence of a conical deficit corresponding to the influence of a cosmic string
attached to the horizon. It will therefore be necessary for us to first
investigate how one might formulate thermodynamics of spacetimes in the presence
of conical defects, and to simplify that task we will restrict ourselves to
nonaccelerating spacetimes.

\section{Thermodynamics with conical deficits}\label{sec:singlemu}

Let us commence with the basic geometry which describes a static black hole with
a cosmic string running through its core: a spacetime first studied by Aryal,
Ford and Vilenkin (AFV)~\cite{Aryal:1986sz}. AFV considered a conical deficit
through a Schwarzschild black hole:
\begin{equation}
ds^2 = - f(r) dt^2 + \frac{dr^2}{f(r)} + r^2 d\theta^2
+ r^2 \sin^2\theta \left (\frac{d\phi}{K}\right ) ^2
\label{afv}
\end{equation}
where $f(r) = 1 -2m/r$. They considered a first law of thermodynamics to argue
that the entropy of the black hole remained at one quarter of its area, now
containing a factor of $K$: $S = \pi r_+^2/K$. The thermodynamics of a black
hole with a string was also considered in greater thoroughness by Martinez and
York \cite{Martinez:1990sd}, although the \emph{tension} of the cosmic string,
(see \cref{sec:conical}) was held fixed. The only context in which a
\emph{varying} tension was considered was in \cite{Bonjour:1998rf}, where the
varying tension was produced by the capture of a moving cosmic string by a black
hole, and it was argued that in the collision of a black hole and cosmic string,
the black hole would retain a portion of the string thus increasing its mass.

We revisit this static system first, as a means of exploring the impact of
varying tension on black hole thermodynamics. We will consider a charged black
hole represented by the metric~\eqref{afv}, with
\begin{equation}
f(r) = 1-\frac{2m}{r} + \frac{e^{2}}{r^{2}} + \frac{r^{2}}{\ell^2}\;, \qquad
\text{and} \quad {B} = - \frac{e}{r} dt\;.
\end{equation}
The parameters $m$ and $e$ are related to the black hole's mass and charge
respectively, $B$ is the Maxwell potential, and we allow for a negative
cosmological constant via $\ell = \sqrt{-\Lambda/3}$.

In order to treat varying tension we leave the parameter $K$ in \cref{afv}
unspecified. As has already been explained, this parameter would typically
simply be unity (or a function of rotation in the Kerr-AdS case), however, by
keeping $K$ explicitly in the metric we can study conical defects through a
well-behaved system in a straightforward manner.

Examining the geometry near $\theta_+ = 0$ and $\theta_- = \pi$ reveals how the
parameter $K$ relates to the conical defect. Near the poles, the metric becomes
\begin{align}
ds_{\mathrm{II}}^2 = r^2 \left[ d\vartheta^2 + \frac{\vartheta^2}{K^2} d\phi^2\right],
\end{align}
on surfaces of constant $t$ and $r$, where
$\vartheta = \pm (\theta - \theta_\pm)$ is the `distance' to either pole. If
$K\neq 1$, there will be a conical defect along the axis of revolution, which
corresponds to a cosmic string of tension
\begin{align}
\mu = \frac{\delta}{8\pi} = \frac14 \bigg[1-\frac{1}{K}\bigg],
\label{afvtension}
\end{align}
where $\delta$ is the conical deficit. The interpretation of tension is
justified by analysing the equations of motion for an actual cosmic string
vortex in the presence of a black hole~\cite{Achucarro:1995nu}, where
\eqref{afv} was obtained as the asymptotic form of the metric outside the string
core.  A tensionless string corresponds to a regular pole, $K=1$, and in this
metric, the tension along either polar axis is equal, allowing simultaneous
regularisation of the two poles. The static black hole is inertial, as the
deficits balance each other out. This exercise provides insight into the role
$K$ plays within a metric. Different values for this parameter determine the
severity of an overall defect running through the black hole.

Now let us consider the temperature and entropy (as defined earlier) of the
black hole. We compute $T$ by demanding regularity of the Euclidean section of
the black hole \cite{Gibbons:1976ue}, giving
\begin{align}
T = \frac{f'(r_+)}{4\pi} = \frac{1}{2\pi r_+^2} \left [
m - \frac{e^2}{r_+} + \frac{r_+^3}{\ell^2}\right]
\end{align}
thus 
\begin{align}
2TS = \frac{m}{K} - \frac{e}{r_+}\left (\frac{e}{K} \right ) 
+ 2\left (\frac{3}{8\pi \ell^2} \right ) \left (\frac{4\pi}{3} r_+^3\right ) 
= M - \Phi Q + 2PV
\end{align}
gives a Smarr formula \cite{Smarr:1972kt} for the black hole, where $M=m/K$ is
the mass of the black hole, $Q = \frac{1}{4\pi}
\int \star dB=e/K$ is the charge on the black hole, 
$\Phi = e/r_+$ the potential at the horizon, and $P = 3/8\pi \ell^2$, 
$V = 4\pi r_+^3/3$ the thermodynamic pressure and volume respectively 
\cite{Teitelboim:1985dp,Kastor:2009wy,Dolan:2011xt}.

Now let us consider the effect of changing the parameters of the black hole
a small amount; the location of the horizon of the black hole will also shift
so that $f+\delta f= 0$ at $r_++\delta r_+ $:
\begin{align}
0= f'(r_+) \delta r_+ - \frac{2\delta m}{r_+} 
+ \frac{2e\delta e}{r_+^2}  - 2r_+^2 \frac{\delta \ell}{\ell^3}
\label{simplevary}
\end{align}
However, we can now replace the variation of the parameters $m$, $e$, $\ell$
with the variation of the corresponding thermodynamic charges $M$, $Q$, $P$, and
the variation of $r_+$ with that of entropy, with the important proviso that we
must allow for the variation of tension through $K$. Thus
$\delta m = K \delta M + M \delta K$ etc. and $\delta K = 4K^2\delta \mu$ from
\cref{afvtension}.  After some rearrangement, \cref{simplevary} gives our first
law of thermodynamics with varying tension:
\begin{align}
\delta M = T \delta S + V \delta P + \Phi \delta Q
-2 \lambda \delta\mu
\label{firstlawwithK}
\end{align}
where
\begin{align} 
\lambda = \left ( r_+ - m\right)
\label{TDlengthnoacc}
\end{align}
is a \emph{thermodynamic length} conjugate to the string tension.

This is an important ingredient to this formulation --- that string tension (in
this case equal along each axis) could be thought of as analogous to a
thermodynamic charge that therefore has a corresponding thermodynamic potential.
Rather than write a single $\lambda\delta\mu$ term, instead we write two such
terms, referring to the deficits emerging from each pole. Although these are
obviously equal in this case, one might envision situations where this is not
the case. Indeed, our experience with accelerating solutions is that these are
spacetimes where the conical configuration does just that; attributing a
$\lambda \delta \mu$ term per pole is therefore justified.

\subsection{A concrete example --- capture of a cosmic string}

Let us observe consider the following example to verify the first law in action:
the capture, and subsequent escape, of a cosmic string by a black hole.  This
example was first proposed in \cite{Bonjour:1998rf} in the case of a charged
vacuum black hole.  The idea is that the string is moving and gets briefly
captured by the black hole. In the capture process, the internal energy of the
black hole should remain fixed: the physical intuition is that if a cosmic
string were to pass through a spherical shell of matter, energy conservation
would demand that the spherical shell still have the same total energy
throughout the process, thus either it would become denser, or its radius would
increase.  Of course, in the case of the spherical shell, the cosmic string
would simply transit through, leaving the system. For the black hole however, we
will see this is not the case, and we have the interpretation of a segment of
string having been captured by the black hole, with the black hole increasing
its mass accordingly. This process was considered in the probe limit in
\cite{Lonsdale:1988xd,DeVilliers:1997nk}. We therefore consider the vacuum
Reissner-Nordstr\"om (RN) metric which has the following structure function:
\begin{align}
f(r) = 1  - \frac{2m}{r} + \frac{e^2}{r^2}
\end{align}
with the charge of the black hole being defined via $Q = e/K$, and the electric
potential being $\Phi = e/r_+$.

Let us suppose that the string is light, or $\mu \ll 1$, then in the first stage
where the black hole captures the string, fixing $M$ and $Q$ implies
$\delta m = 4 m \delta \mu$ and $\delta e = 4e\delta \mu$ (to first order in
$\mu$). Thus
\begin{align}
T \delta S = \frac{r_+-r_-}{4\pi r_+^2} \left [
2\pi r_+ \delta r_+ - 4\pi r_+^2 \delta\mu \right ]
= (r_+-r_-) \delta \mu = 2\lambda \delta \mu
\end{align}
as required. Interestingly, because the internal energy has been fixed, the
event horizon has to move outwards to compensate for the conical deficit. Since
the entropy contains just one factor of $K$, but two of $r_+$, the net effect is
an increase of entropy, indicating this is an irreversible thermodynamic
process. The one interesting exception being an extremal black hole.

In the second step, the string pulls off the black hole, so $\delta \mu = -\mu$,
and since the string is uncharged, $\delta Q$ must remain zero, and $e$ 
returns to its original value. However, since entropy cannot decrease, 
$M$ must increase
\begin{align}
\delta M = T\delta S + 2(r_+-m)\mu
= \frac{(r_+-r_-) \delta r_+}{2r_+} + 2(r_+-r_-)\mu 
\end{align}
In \cite{Bonjour:1998rf}, it was supposed that $m$ did not change, leading to an
increase in $M$ of $4m\mu$, which was then stated as being the mass of the
string behind the event horizon, however this is in fact only true for the
uncharged black hole. Instead, it seems more physically accurate to suppose that
$r_+$ does not decrease, as otherwise the local geodesic congruence defining the
event horizon would appear to be contracting in contradiction to the area
theorem.  In this case, $\delta M = 2 (r_+-r_-) \mu$, or the length of cosmic
string trapped between the inner and outer horizons. Even if one allows the
local horizon radius to shrink while maintaining constant entropy,
$\delta M = (r_+ - r_-)\mu$: half the former amount, but still an increase of
mass due to the capture of a length of cosmic string.

\section{Thermodynamic length for the rotating black hole}
\label{sec:therm-length-rotat}

From here, we can look into extending this property to rotating spacetimes and
see how this affects the corresponding thermodynamic expressions. However,
before proceeding into detail, it will be useful to remind ourselves of some of
the subtleties introduced when discussing the thermodynamics of a rotating black
hole in asymptotically AdS space, initially discussed by Hawking, Hunter and
Taylor-Robinson (HHT) in~\cite{Hawking:1998kw}, which we covered in
\cref{sec:BHTDlambda}. These subtleties were pointed out
in~\cite{Caldarelli:1999xj,Gibbons:2004ai}, where it was shown that with a
nonzero cosmological constant, the boundary is actually rotating with angular
velocity
\begin{align}
\Omega_\infty = \lim_{r\to\infty} -\frac{g_{t\phi}}{g_{\phi\phi}} = aK
  \frac{r^2/\ell^2}{a^2 r^2 \sin^2\theta/\ell^2- r^2 g(\theta)} = -
  \frac{a}{\ell^2} \frac{K}{\Xi}, \qquad \Xi = 1-\frac{a^2}{\ell^2}
\end{align}
implying that the angular velocity ought to be re-normalised and that
$\Omega = \Omega_\mathrm{H} - \Omega_\infty$ is the true total angular
velocity. The mass was then found to be given by $M=m/\Xi^2$ as opposed $m/\Xi$,
the expression originally given by HHT, which is obtained using the Komar method
when a normalisation of the timelike killing vector is omitted. Similarly, the
expression for thermodynamic volume,
\begin{align}
  \label{eq:volKerrAdS}
V = \frac{4\pi}{3K} \left( r_+(r_+^2+a^2) + ma^2 \right)
\end{align}
contains a second, rotation-dependent term which may also be viewed as a
normalisation.

Employing similar ideas and viewing thermodynamic potentials as having
normalising terms, we can actually show that these corrections to the
thermodynamic mass and angular velocity are required to satisfy the first law
while simultaneously obtaining these expressions for arbitrary and potentially
varying string tensions.

The first step is to vary $f(r_+)=0$ to establish an initial thermodynamic
relation. Identifying $S$ as a quarter of the horizon area and $T$ as the
temperature given by the euclideanisation procedure,
\begin{align} \label{eq:SandT-al}
S = \frac{\pi}{K}(r_+^2+a^2) \qquad
  T=\frac{1}{2\pi(r_{+}^{2}+a^{2})}
  \bigg[m-\frac{a^{2}}{r_{+}}+\frac{r_{+}^{3}}{\ell^{2}}\bigg], 
\end{align}
will lead to the following statement:
\begin{align}\label{eq:master-al}
\delta\left(\frac{m}{K}\right)=T\delta S +V_0\delta P + \Omega_0\delta J -
  2r_+\delta\mu + \frac{m\delta K}{2K^2},
\end{align}
where we have written $V_0$ and $\Omega_0$ in anticipation of correction terms,
however it is worth noting that $\Omega_0=\Omega_H$ is the angular velocity at
the horizon and that $V_0$, the first term in \cref{eq:volKerrAdS}, satisfies a
\emph{reduced} Smarr relation given by $m/K = 2(TS-PV_0+\Omega_0 J)$. These
quantities are given by the following relations:
\begin{align} \label{eq:horpotentials-al}
\Omega_0 =  \frac{a K}{r_{+}^{2}+a^{2}}, \qquad J=\frac{ma}{K^2}, \qquad V_0 =
  \frac{4\pi}{3K}r_+(r_+^2+a^2), \qquad P = \frac{3}{8\pi\ell^2}, 
\end{align}
where the expression for the angular momentum $J$ is obtained unambiguously via
the Komar method using background ($m=0$) subtraction, as
per~\cite{Magnon:1985sc}.

While \cref{eq:master-al} looks like a first law, it is necessary to remember
that $K$ parametrises the tension and can therefore not appear as a standalone
term. The aim of this derivation is precisely to find such a first law. Let us
now introduce a function $\gamma = \gamma(a,\ell)$ in our expression for the
mass. We know that $\gamma$ will need to depend on $a$ and $\ell$ from the
relation between $K$ and $\mu$. Using this ansatz for mass and then perturbing
it, we have
\begin{align}  
  M&=\frac{m}{K}\gamma(a,\ell),\\
  \qquad \delta M &= \delta \left(\frac{m}{K}\right) +\frac{m}{K}(\gamma_{a}
                    \delta a + \gamma_{\ell}\delta\ell)\nn\\
   &= (\gamma - a\gamma_{a})\delta \left(\frac{m}{K}\right) +
     \gamma_{a}K\delta J + \frac{m a \gamma_{a}}{K^{2}}\delta K +
     \frac{m}{K}\gamma_{\ell}\delta \ell.
     \label{eq:varm-al}
\end{align}
Now, if we rewrite \cref{eq:master-al} using $\Omega = \Omega_0+\Omega_1$,
$V = V_0 + V_1$ as
\begin{align}
  \label{eq:flt-al}
  &T\delta S + V\delta P + \Omega \delta J - 2\lambda \delta \mu\nn\\
  &\hspace{7em}= \delta
  \left(\frac{m}{K}\right) + \Omega_1\delta J -\frac{3V_1}{4\pi \ell^3}\delta
  \ell + 2(r_+-\lambda)\delta\mu -\frac{m}{2K^2}\delta K,
\end{align}
we may require that \cref{eq:varm-al,eq:flt-al} be equal to find constraints on
$\gamma$. Using \cref{eq:deficit-al} and $\mu=\delta/8\pi$ to express $\delta K$
in terms of $\delta \mu$, we can infer a differential equation that $\gamma$
ought to satisfy,
\begin{align}\label{eq:eq1al}
\left(1+\frac{a^2}{\ell^2}\right)\gamma-
  \left(1-\frac{a^{2}}{\ell^{2}}\right)a\gamma_{a}-1=0 
\end{align}
as well as the following expressions for the correction terms, defining them in
terms of $\gamma$:
\begin{gather}
V_1=-\frac{4\pi}{3K}\frac{m
  \ell^2}{1+a^2/\ell^2}\left(\left(1+\frac{a^2}{\ell^2}\right)
  \ell\gamma_{\ell}+2\frac{a^2}{\ell^2}a
  \gamma_{a}+\frac{a^2}{\ell^2}\right),\nn\\  
\Omega_1=K\gamma_{a}\left(1-2\frac{a^2}{\ell^2}\frac{1}{1+a^2/\ell^2}\right)-
\frac{aK}{\ell^2}\frac{1}{1+a^2/\ell^2},\nn\\ 
\lambda=r_+-\frac{m}{1+a^2/\ell^2}\left(2a\gamma_{a}+1\right). 
\end{gather}
We also require that the Smarr relation, which follows from the
scaling properties of the system, also be satisfied. Inserting the above
expressions into
\begin{align}
M = 2(TS-PV+\Omega J),
\end{align}
we obtain another differential equation for $\gamma$,
\begin{align}\label{eq:eq2al}
\left(1+\frac{a^2}{\ell^2}\right)(\gamma-\ell\gamma_{\ell})-2a\gamma_{a}-1=0.
\end{align}
It is then straightforward to solve \cref{eq:eq1al,eq:eq2al} and one obtains
\begin{gather}
\gamma = \frac{1}{\Xi}\left(1+\frac{a}{\ell}\zeta\right),\qquad \lambda = r_+ -
\frac{m}{\Xi^2}\left(1+\frac{a^2}{\ell^2}+\frac{2a}{\ell}\zeta\right), \nn\\  
\Omega_1 = \frac{a}{\ell^2}\frac{K}{\Xi}\left(1+\frac{\ell}{a}\zeta\right), \qquad V_1 =
\frac{4\pi}{3}\frac{ma^2}{K\Xi}\left(1+\frac{\ell}{a}\zeta\right), 
\end{gather}
where $\zeta$ is an integration constant. We can fix it by identifying $\Omega_1
= -\Omega_\infty$ provided $\zeta = 0$, which also assures that the angular
velocity of the boundary vanishes for $a=0$. Similarly, we obtain the correct
expression for thermodynamic volume if $\zeta = 0$. \todoopt{provide explanation with
  parity argument}

Finally, one can repeat this derivation with the inclusion of charge $Q=e/K$ and
introduce a correction term to the potential $\Phi=\Phi_0+\Phi_1$. This leads to
a similar expression for $\gamma$, with $\zeta$ now a function of charge
\begin{align}
  \gamma =
\frac{1}{\Xi}\left[1+\frac{a}{\ell}\zeta\left(\frac{e^{2}}{\ell^{2}}\frac{1}{\Xi^{2}}\right)\right].
\end{align}
Substituting this in for the following correction terms satisfies
the first law for a charged, rotating black hole with potentially varying
tensions.\todoopt{insert expressions without $\zeta$}
\begin{align}
  V_1&=-\frac{4\pi}{3}\frac{m a^2}{K(1+a^{2}/\ell^{2})}\left[2a\gamma_{a}+2e
\gamma_{e}-\left(1+\frac{\ell^2}{a^2}\right)\ell\gamma_{\ell}+1\right],
 \nn\\
  \lambda &= r_+-\frac{m}{1+a^2/\ell^2}(2a\gamma_{a}+2e\gamma_{e}+1),\nn\\
  \Phi_1 &= m\gamma_{e}, \nn\\
\Omega_1&=-\frac{aK}{\ell^2(1+a^2/\ell^2)}
\left[a\gamma_{a}\left(1-\frac{\ell^2}{a^2}\right)+2e\gamma_{e}+1\right]
=\frac{a}{\ell^2} \frac{K}{\Xi} \left[1+\frac{\ell}{a}\zeta\left(\frac{e^{2}}{l^{2}}\frac{1}{\Xi^{2}}\right)\right].
\end{align}
The same argument as above can be used to set $\zeta=0$.

\section{The thermodynamic length}

To recap, we have shown that allowing for a varying conical deficit
in black hole spacetimes, the first law of thermodynamics becomes
\begin{align}
\delta M = T \delta S + V\delta P + \Phi \delta Q +\Omega\delta J - 2\lambda \delta \mu,
\end{align}
where the relevant thermodynamical variables are given in~\eqref{TDparams}.
In order to accommodate varying 
tension, we have to define a \emph{thermodynamic length},\todoopt{formula}
\begin{align}
\lambda_\pm = \frac{r_+}{1\pm Ar_+} - \frac{m(1-e^2A^2)}{(1+e^2 A^2)^2} 
\mp \frac{e^2A}{(1+e^2 A^2)}
\end{align}
for each conical deficit emerging from each pole. This length consists of 
a direct geometrical part, a mass dependent correction, and finally, a shift
in the presence of charge.

Surprisingly perhaps, this thermodynamic length is not simply the geometric
length $r_+$ of the string from pole to singularity.  Instead, the
mass-dependent adjustment emphasises this is a potential, rather than just an
internal energy term that might more appropriately be placed on the left hand
side of the equation. \todoopt{detail regarding distance between horizons, etc.}

It is interesting to compare this mass-dependent shift of the thermodynamic
length to the correction of the thermodynamic volume for a rotating black
hole \cite{Cvetic:2010jb,Dolan:2011jm}:
\begin{align}
V = \frac{4\pi}{3} \left( \frac{r_+(r_+^2+a^2)}{K} + a^2 M \right)
\end{align}
In this case, the first term is the expected geometric volume of the interior of
the black hole, the second term being a rotation-dependent correction. It is
with this appropriately shifted thermodynamic volume, that the black hole always
satisfies the reverse isoperimetric inequality~\cite{Cvetic:2010jb}.

Notice that the correction term for this thermodynamic volume is always
positive, whereas the correction term for thermodynamic length is actually
negative.  This means that for large enough mass, the thermodynamic length
itself becomes negative, as shown in \cref{fig:TDlength} for an uncharged black
hole. The picture for a charged black hole is similar, although the critical
value of $M$ for which $\lambda_-$ becomes negative is larger. Comparing this
with thermodynamic volume, we see that analogous to $V$, the first term is a
length - that of the string from the pole to $r=0$, and the second term is a
mass-dependent correction that normalises this thermodynamic length.  It can be
understood by thinking of the two ways in which a cosmic string can affect the
black hole: First, by introducing a conical deficit, the relation between the
mass of the black hole and the solution mass parameter $m$ is corrected, since
the internal energy of the black hole will be reduced by a factor of $4\mu$ if a
conical deficit is present.  The second effect comes from the energy of the
string behind the event horizon - although this does not formally seem to
contribute to the mass of the black hole - if the deficit increases, then the
black hole must give some ``extra mass'' to the string inside the event horizon,
commensurate with the increased deficit.\todoopt{correct figure}
\begin{figure}
  \centering
  \includegraphics[width=0.7\textwidth]{TDlength.pdf}
  \caption{\label{fig:TDlength}A plot of thermodynamic length of the south
    pole tension for regular north pole, $\mu_+=0$. The limit of the
    Schwarzschild-AdS black hole is shown in pink ($\mu_-=0$). The slowly
    accelerating r\'egime, $A<A_\mathrm{crit}$, is shown as a solid line, and
    $A>A_\mathrm{crit}$ is shown dashed.}
\end{figure}


From \cref{fig:TDlength}, we see that the thermodynamic length becomes negative
for `large' black holes, i.e.\ those for which the thermodynamic mass is of
similar order (or higher) than the AdS scale. Setting this in the context of the
`cosmic string' capture process considered in \cref{sec:singlemu} for the vacuum
black hole, this would mean that the thermodynamic mass must increase during a
capture, as entropy cannot decrease. This seems at first counter to the notion
that the string itself does not carry `ADM' mass, however, the heuristic
argument of \cref{sec:singlemu} relies somewhat on the notion that a cosmic
string and black hole can be sufficiently separated so that one can consider
their thermodynamical (and other) properties independently. For large black
holes in AdS this is manifestly not the case.



\chapter{Thermodynamics of accelerating black holes}
\label{cha:therm-accel-black}

As we have already stated, our goal is to establish whether a thermodynamic
interpretation for accelerating black holes may be constructed. Accelerating
black holes have always presented a problem in this respect, partly due to the
existence of an external driving force which might indicate an inability to
attain any kind of equilibrium, but there is also an algebraic obstacle; the
existence of a conical deficit and an acceleration parameter for which we have
no prior thermodynamic interpretation. Add to that the fact that the boundary is
displaced from $r = \infty$ (in Boyer-Lindquist coordinates) resulting in
awkward asymptotics to deal with and the existence of an acceleration horizon
with its own temperature and it becomes clearer why such a formulation does not
exist.

There is at least one of these aspects about which we might feel more
confident. The previous chapter discussed how one could include the tension of
cosmic strings attached to black hole horizons as a thermodynamic variable,
introducing the concept of the thermodynamic length, the conjugate to the
tension. Each of the configurations we have dealt with up until now ahd equal
deficits between the north and south poles, however, as we reviewed in
\cref{sec:conical}, the C-metric represents an accelerated black hole driven by
an imbalance in the cosmic string tensions at each of the poles. What must then
be addressed is whether this construction, of black hole thermodynamics with
varying conical deficits, may be extended to allow for independent variations in
the tensions at the north and south poles. As we shall see below, this is
possible, and it will allow us to simultaneously address one of the other
potential issues that we brought up at the start of this chapter: where in the
previous chapter we showed that the parameter $K$ is linked to the overall
deficit in the spacetime and that its variations could be re-expressed as
variations in the string tension $\mu$, we now have another variable $A$,
representing acceleration, whose variations, together with those for $K$, may be
re-cast as independent variations in the north and south string tensions,
$\mu_\pm$.

We will first begin to develop this framework with the simplest case, the
uncharged nonrotating accelerating black hole and use the insight gained from
the previous chapter to choose to fix the tensions. This will be physically
motivated, and allows us to make initial assertions as to the necessary
conditions to formulate consistent thermodynamics. We will then extend this
to include the aforementioned generalisation of independently varying conical
deficits, at least for charged nonrotating accelerating black holes, deriving
the thermodynamic lengths for these solutions along the way.

\section{Thermodynamics of the C-metric}

\subsection{Establishing a first law}

For a general black hole spacetime containing conical defects, any disparity in
the sizes of the deficits produces an overall force in the direction of the
largest conical \emph{deficit}, and the geometry is described by the
C-metric~\cite{Kinnersley:1970zw}, introduced in \cref{chap:cmet}. Typically,
C-metrics have both black hole and acceleration horizons. In order to have any
chance at constructing a thermodynamic description, we would like to be able to
eliminate the acceleration horizon which has its own temperature, different to
that of the black hole's. This is possible in asymptotically anti-de Sitter
soace and for that reason we shall be studying accelerating black holes in
negative cosmological constant universes. More specifically, we must restrict
ourselves to the so-called \emph{slowly accelerating C-metric}
\cite{Podolsky:2002nk}, for which the acceleration is small enough that the
negative curvature prevents the existence of an acceleration horizon. With this
geometry, we may consider the entire spacetime, black hole and string(s)
combined, as forming a thermodynamic equilibrium. In the absence of an
acceleration horizon, the spacetime can be interpreted as having a black hole
maintained a finite distance from the centre of AdS by the cosmic string. We
will revisit this shortly, for now let us rewrite the metric here, for
convenience. We have
\begin{align}
  ds^2=\frac{1}{\Omega^2}\left[ -
  f(r) dt^2
  +\frac{dr^2}{f(r)} + r^2 \Big( \frac{d\theta^2}{g(\theta)} 
  + g(\theta)\sin^2\theta \frac{d\phi^2}{K^2}\Big)\right],
  \label{eq:cmetric3}
\end{align}
and
\begin{align}
f(r)&=(1-A^2r^2)\Bigl(1-\frac{2 m}{r}\Bigr)
+\frac{r^2}{\ell^2},\\
  g(\theta)&=1+2mA \cos\theta,\nn\\
  \Omega&=1+Ar\cos\theta.
\label{eq:cmetric3-fun}
\end{align}
The conical deficits this spacetime exhibits correspond to cosmic strings with
tensions given by (see \cref{eq:deficitsNR})
\begin{align}
  \label{eq:tensionsNR}
  \mu_\pm=\frac14\bigg[1-\frac{g(\theta_\pm)}{K}\bigg] = \frac14\bigg[1-\frac{1\pm
  2mA}{K}\bigg]. 
\end{align} 
Further details concerning the spacetime and its subtleties can be found by
referring back to \cref{sec:cmet-review}. 

We have already seen how the parameter $K$ is related to the conical deficits in
\cref{chap:td-conical}, however we would like to give some interpretation of the
parameter $A$, which was identified as the acceleration in
\cref{sec:cmet-history} for the asymptotically flat solution by studying the
weak field limit $m=0$ and exposing it as a a reparametrisation of Rindler
spacetime. In the presence of a cosmological constant it may be more helpful to
view it as the acceleration required to maintain the black hole some distance
away from the centre of AdS, as we alluded to previously. Setting $m=0$ and
$K=1$ to eliminate the conical deficit in \cref{eq:cmetric3} gives
\begin{align}
ds^2=\frac{1}{\Omega^2}\left[ -
\Big( 1 + \frac{r^2}{\ell^2} (1-A^2 \ell^2) \Big) dt^2
+\frac{dr^2}{1 + \frac{r^2}{\ell^2}(1-A^2\ell^2)} + r^2 ( d\theta^2
+ \sin^2 d\phi^2)\right].
\end{align}
This spacetime no longer has a conical singularity and is locally pure AdS,
however in these coordinates the boundary of AdS is not at $r=\infty$, but at
$r = -1/(A\cos\theta)$. For $\theta$ in the southern hemisphere, this occurs at
\emph{finite} $r$, but in the northern hemisphere $r=\infty$ actually lies
within the AdS spacetime (see \cref{fig:accbh}). To transform to global AdS
coordinates $\{R,\Theta\}$, one takes \cite{Podolsky:2002nk}\todoruth{And these
  are? This section feels incomplete.}\todoopt{not sure what is meant by comment}
\begin{equation}
1 + \frac{R^2}{\ell^2} = \frac{1 + (1-A^2\ell^2)r^2/\ell^2}{(1-A^2\ell^2)\Omega^2}
~,\qquad
R \sin\Theta = \frac{r\sin\theta}{\Omega}.
\end{equation}
The boundary, $R\to\infty$ now clearly corresponds to $\Omega\to 0$, and the
origin of Rindler coordinates corresponds to
$R_0 = A \ell^2/\sqrt{1-A^2\ell^2}$, in other words, the Rindler coordinates
represent those of an observer displaced from the origin of AdS.
\begin{figure}
  \centering
  \begin{subfigure}[b]{0.5\textwidth}
    \includegraphics[width=\textwidth]{slowaccrind.pdf}
    \caption{\label{fig:accbha}}
  \end{subfigure}~
  \begin{subfigure}[b]{0.4\textwidth}
    \includegraphics[width=\textwidth]{accbh.pdf}
    \caption{\label{fig:accbh}}
  \end{subfigure}
  \caption{\label{fig:accbh}(a) The slowly accelerating Rindler spacetime shown
    here with $A\ell = 1/4$, and $\ell=1$ for simplicity. The spatial sections
    of AdS have been compactified to a Poincar\'e disc, with the constant $r$
    Rindler coordinate indicated in black and constant $\theta$ in blue. The
    origin of the Rindler coordinates is clearly visible as being displaced from
    the centre of the disc, with the limit of the $r$-coordinate being the thick
    dashed black line. (b) The black hole distorts the Poincar\'e disc with a
    conical deficit, and is displaced from the origin of AdS. The spacetime is
    again static, and a cross section is shown.}
\end{figure}

Let us now define an important thermodynamic quantity. We suspect $m$ to be
related to the mass of the black hole, however the computation required to
obtain such an expression is rather tricky. The awkward asymptotics do not lend
themselves well to a Komar approach. Instead, we used the method of conformal
completion \cite{Ashtekar:1984zz, Ashtekar:1999jx, Das:2000cu}. This takes the
electric part of the Weyl tensor projected along the timelike conformal Killing
vector, and integrates over a sphere at conformal infinity. The calculation
gives\footnote{More details regarding this computation are provided in
  \cref{sec:ashtekar-das}. Though it should be noted that the result stated here
  was later discovered to be incorrect. In particular, this is related to an
  issue with the somewhat ambiguous normalisation of the timelike killing vector
  which affects the computation by an overall scale factor. \Cref{chap:holoTD}
  addresses the issue, however these details were discovered in the later stages
  of the production of this thesis. The results stated in this chapter hold
  despite this inconsistency.}
\begin{align}
  M = \frac{m}{K},
  \label{eq:massAcc}
\end{align}
thus $m$ gives the mass of the black hole. Note that unlike the rapidly
accelerating black hole, this is a genuine ADM-style mass, and not a
``rearrangement'' of dipoles as discussed in \cite{Dutta:2005iy}, where a boost
mass was introduced for the C-metric.

Meanwhile, we identify the entropy with a quarter of the horizon 
area 
\begin{align}
  \label{eq:entropyFT}
S=\frac{\mathcal{A}}{4}=\frac{\pi r_+^2}{K(1-A^2r_+^2)}\,,
\end{align}
and calculate the temperature via the usual Euclidean method (see
\cref{sec:euclidean}) to obtain
\begin{align}
T=\frac{f'(r_+)}{4\pi} = \frac{1}{2\pi r_+}\left(\frac{m}{r_+}\left(1+A^2r_+^2
\right) + \frac{r_+^2}{\ell^2} - A^2 r_+^2\right).
\end{align}
We now identify $P$ with the pressure associated to the cosmological constant
according to $P=\frac{3}{8\pi \ell^2}$, which allows us to rewrite the
temperature as
\begin{align}
TS = \frac{M}{2}+ P \frac{4\pi}{3K}
\frac{r_+^3}{(1-A^2r_+^2)^2},
\end{align}
which is nothing other than the Smarr relation $M= 2(TS - PV)$ provided we
identify the black hole thermodynamic volume as
\begin{align}
V=\frac{4\pi}{3K}\frac{r_+^3}{(1-A^2r_+^2)^2}.
\end{align}

So far, this is a rewriting of a relation for the temperature, having identified
standard thermodynamic variables or charges for the solution. Now let us
consider the first law by considering a variation due to some physical process.
Typically, one derives the first law by observing the change in horizon radius
during a physical process. The horizon radius is given by a root of $f(r_+)=0$,
and thus depends on $m$, $A$ and $\ell$. The specific form of this
algebraic root is not vital, what matters is how the mass varies in terms of the
change in horizon area, thermodynamic volume, and charge.

Originally, it was reasoned that during this process, the conical deficits (or
lack thereof) could not change, as these corresponded to the physical objects
causing the acceleration. Of course, as we have already shown by now, one could
enviseage physical processes that would alter a conical defects on a black hole
horizon, nonetheless it is simplest to restrain ourselves to the scenario where
all tensions are held fixed. Thus we must consider a variation of $m$, $A$ and
$K$ that preserves the cosmic string tensions, and it turns out that it is
precisely through this physical restriction that we are able to derive a first
law.

To obtain the first law, we typically consider a perturbation of the equation
that determines the location of the event horizon of the black hole:
$f(r_+)=0$. If we allow our parameters to vary, this will typically result in a
perturbation also of $r_+$, hence we can write
\begin{align}
\frac{\partial f}{\partial r_+} \delta r_+ +
\frac{\partial f}{\partial m} \delta m +
\frac{\partial f}{\partial A} \delta A +
\frac{\partial f}{\partial \ell} \delta \ell =0
\end{align}
where everything is evaluated at $f(r_+, m,A, \ell)=0$.  Clearly we can replace
$\delta m$ and $\delta \ell$ by variations of the thermodynamic parameters $M$
and $P$, and $\delta r_+$ is expressible in terms of $\delta S$ and $\delta A$
using \cref{eq:entropyFT}. Finally, we replace
$\partial f /\partial r_+ = 4 \pi T$, and use $f(r_+)=0$ to simplify the terms
multiplying $\delta A$ to obtain:
\begin{align}
(1-A^2 r_+^2) (T \delta S + V \delta P) - \delta M - \frac{m A r_+^2}{K} \delta
  A + \left(\frac{r_+^2}{\ell^2}-(1+A^2r_+^2)\right)\frac{r_+\delta K}{4K^2}  =0,
\label{eq:firstint}
\end{align}
where we have also allowed $K$ to vary for completeness.

At the moment, it seems as if we have extra thermodynamic contributions, however,
we now use the physical input from the cosmic string that the conical deficits
on each axis must not change. To achieve this, we must require that both $\delta
\mu_\pm = 0$. A quick look at the linear combinations
\begin{align}
  \mu_+ + \mu_- = \frac12\left[1-\frac{1}{K}\right] \qquad \mbox{and} \qquad \mu_+-\mu_- = -\frac{mA}{K}
\end{align}
reveals that this is achieved by the conditions $\delta K=0$ and
$\delta (mA) = 0$, or $m\delta A = - A \delta m$.  Replacing $\delta A$ in
\cref{eq:firstint} and rearranging, finally, indeed gives the first law:
\begin{align}
\delta M = T \delta S + V \delta P.
\label{eq:firstlawFT}
\end{align}

Thus, this first pass at a thermodynamic construction for accelerating black
holes is indeed promising. Despite a few intuitive barriers, we have succeeded
in establishing a first law for this black hole solution, suggesting that it
ought to display similar thermal behaviour to other known solutions. We will
carry out a survey of its thermodynamic features further on, however we must now
generalise this result to include electric charge as well as investigate whether
it is feasible to allow the string tensions to vary independently.

\subsection{Thermodynamics of the charged C-metric}
                            \label{sec:cmet-TD-der}

Having shown that the C-metric appears to obey the laws of black hole
thermodynamics, at least in the uncharged case when the cosmic string tensions
are held fixed, we will now perform a similar analysis to establish verify this
is the case for the charged black hole, while this time including terms
corresponding to each deficit. Keeping the metric as in \cref{eq:cmetric3}, now
with $f(r)$ defined to be \eqref{
We start by finding the temperature and entropy of the black hole, using
the conventional relations
\begin{align}
  \label{eq:TSNR}
T &= \frac{f'(r_+)}{4\pi} = \frac{1}{2\pi r_+^2} \left [ (1-A^2 r_+^2) \left (
m - \frac{e^2}{r_+}\right) + \frac{r_+^3}{\ell^2(1-A^2 r_+^2)} \right]\nn\\
S &= \frac{\mathcal{A}}{4} =\frac{\pi r_+^2}{K(1-A^2 r_+^2)}.
\end{align}
Checking the Smarr relation, we compute
\begin{equation}
2TS = \frac{m}{K} - \frac{e^2}{K r_+^2} + \frac{r_+^3}{K\ell^2(1-A^2 r_+^2)^2}.
\end{equation}
The charge of the black hole is given by \cref{Qdef}, $Q = e/K$, and defining 
\begin{equation}
  \label{eq:volNR}
V  = \frac{4\pi r_+^3}{3K(1-A^2r_+^2)^2}
\end{equation}
as the modified thermodynamic volume, we obtain
\begin{equation}
\frac{m}{K} = 2 TS  + Q\Phi_H - 2 PV\,.
\end{equation}
Although it is tempting to identify $M=m/K$, this would be to ignore the
asymptotics of the spacetime. As mentioned in \cref{sec:BHTDlambda}, the
experience of the rotating AdS black
hole~\cite{Caldarelli:1999xj,Gibbons:2004ai} is that thermodynamic potentials
should be normalised at infinity, and in the case of rotation, expressing this
solution in ordinary Boyer-Lindquist coordinates results in a spacetime that has
a rotating boundary. Subtracting off this rotation leads to an extra
renormalisation of the thermodynamic mass, a correct Smarr formula and correct
first law. 

Here, however, we cannot simply perform a similar electromagnetic
gauge transformation. Our electrostatic potential no longer
vanishes at infinity, and our boundary has an electric flux from pole to pole
\begin{equation}
F = eA \sin\theta \, dt \wedge d\theta
\end{equation}
We obviously cannot subtract this charge, as that would be a physical change,
but it does lead us to suspect that there may be a renormalization of
electrostatic potential and thermodynamic mass. We will show how to do this
presently, but first consider just the uncharged black hole, and consider
variations in the position of the horizon as in \cref{simplevary}:
\begin{equation}
\delta f(r_+) = f_+' \delta r_+  - 2 \frac{\delta m}{r_+} (1-A^2r_+^2)
- 2 A \delta A r_+ (r_+ - 2m) - 2 \frac{r_+^2}{\ell^3} \delta \ell = 0
\end{equation}
The procedure is similar to the previous section, but we now have
more algebra involved in the variation of the thermodynamic parameters.
For example, in relating $\delta r_+$ to $\delta S$, we have:
\begin{equation}
\delta S = \frac{2\pi r_+ \delta r_+}{K(1-A^2 r_+^2)^2} + 
\frac{2 \pi r_+^4 A\delta A}{K (1-A^2 r_+^2)^2} - 
\frac{\pi r_+^2}{(1-A^2 r_+^2)} \frac{\delta K}{K^2}
\end{equation}
where our expressions for the tensions \cref{eq:tensionsNR} give
\begin{equation}
\frac{\delta K}{K^2} = 2 \left ( \delta \mu_+ + \delta \mu_- \right ), \qquad \frac{m}{K} \delta A = - \left [
\delta \mu_+ - \delta \mu_- + A \delta \Big(\frac{m}{K}\Big) \right].
\end{equation}
Defining $M=m/K$, after some algebra one gets
\begin{equation}
\delta M = V\delta P + T \delta S -
\delta \mu_+ \left [ \frac{r_+}{1+Ar_+} - KM \right ] -
\delta \mu_- \left [ \frac{r_+}{1-Ar_+} - KM \right ] \,.
\label{firstaccm}
\end{equation}
Thus, the accelerating black hole has the same thermodynamic first law
as the nonaccelerating black hole, but now with a thermodynamic length
for the piece of string attaching at each pole:
\begin{equation}
\lambda_\pm = \frac{r_+}{1 \pm Ar_+} - KM
\end{equation}
This obviously agrees with \cref{TDlengthnoacc} for the string
threading the black hole, where $r_+$ has now been replaced by
$r_+/\Omega(r_+,\theta_\pm)$ at each pole.

Now let us consider the addition of charge. Following the same procedure of
varying the horizon as before leads to the relation
\begin{equation}
  \label{eq:minifirstlaw1}
\delta \Big(\frac{m}{K}\Big) = T \delta S + V \delta P + \Phi_\mathrm{H}\delta Q
- \frac{r_+\delta \mu_+}{1+Ar_+}- \frac{r_+\delta \mu_-}{1-Ar_+}
+ \frac{m\delta K}{2K^2}
\end{equation}
where now our expressions for the tensions lead to
\begin{align}
\frac{m}{K} \delta A &= -\delta \mu_+ + \delta \mu_- 
- A \delta \left (\frac{m}{K} \right) \nn\\
[1 - e^2 A^2 ] \frac{\delta K}{2K^2} &= A^2 e \delta Q
- \frac{e^2 A^2}{m} \delta \left ( \frac{m}{K} \right )
+ \delta \mu_+ \left [ 1 - \frac{Ae^2}{m} \right ] + \delta \mu_- \left [ 1 +
                                       \frac{Ae^2}{m} \right ] 
\label{deltaKcharge}
\end{align}

Keeping an open mind, we define our thermodynamic mass and electrostatic
potential as:\todoruth{Perhaps here, say that the methid is to seek a consistent
  TD with T and S deformed by surface gravity + area respectively.}
\begin{align}
  M = \frac{m}{K}\gamma(A,e) , \qquad
  \Phi = \Phi_0 + \Phi_1,
\end{align}
where $\Phi_0=\Phi_\mathrm{H}$ and $\Phi_1$ is a correction, re-zeroing the
potential, analogous to the correction of the angular potential of the Kerr-AdS
black hole, but without the corresponding interpretation of being the value of
the original potential at infinity. It also follows from
\cref{eq:minifirstlaw1,deltaKcharge} that we can assume $\gamma$ to be
independent of $\ell$.

Next, we compare
\begin{align}
\delta M &= \gamma \delta \left ( \frac{m}{K} \right ) 
+ \frac{m}{K} (\gamma_e \delta e + \gamma_A \delta A)
\nn\\&= \gamma \delta \left ( \frac{m}{K} \right ) 
+ m \gamma_e \delta Q + m e \gamma_e \frac{\delta K}{K^2}
+ \gamma_A \frac{m}{K} \delta A
\end{align}
to
\begin{align}
T \delta S + V \delta P + \Phi \delta Q -\lambda_+ \delta \mu_+ - \lambda_-\delta \mu_-
= \delta \left ( \frac{m}{K} \right ) 
- \Phi_0 \delta Q - \frac{m\delta K}{2K^2}&\nn\\
+ \left [ \frac{r_+}{1+Ar_+} - \lambda _+ \right]\delta \mu_+
+ \left [ \frac{r_+}{1-Ar_+} - \lambda _- \right]\delta \mu_-&
\end{align}
where $\lambda_\pm$ are to be determined. After some
algebra, we obtain
\begin{align}
  \label{eq:firstlawder1}
  \delta M &- T \delta S - V \delta P - \Phi \delta Q + \lambda_+ \delta \mu_+ +
             \lambda_-\delta \mu_-\nn\\
           &= \left [ (1-e^2 A^2) \gamma-2 e^3A^2 \gamma_e -
             A(1-e^2 A^2) \gamma_A -1 \right] \frac{ \delta \left ( {m/K} \right )
             }{(1-e^2A^2)} \nn\\
           &+ \left [ m (1+e^2 A^2) \gamma_e + mA^2 e +(1-e^2 A^2)
             \Phi_0 \right] \frac{ \delta Q }{(1-e^2A^2) }\nn\\
           &+ \left [ \lambda_+ -\frac{r_+}{1+Ar_+} - \gamma_A +
             \frac{(2e\gamma_e+1)}{1-e^2 A^2} \left ( m - e^2A \right) \right]
             \delta \mu_+\nn\\
           &+ \left [ \lambda_- - \frac{r_+}{1-Ar_+} + \gamma_A
             +\frac{(2e\gamma_e+1)}{1-e^2 A^2} \left ( m + e^2A \right) \right]
             \delta \mu_- 
\end{align}
for our first law to hold, clearly the RHS of this equation must vanish, leading
to a constraint for $\gamma$:
\begin{equation}
(1-e^2 A^2) \gamma-2 e^3A^2 \gamma_e - A(1-e^2 A^2) \gamma_A =1.
\end{equation}
This equation is solvable, and we obtain
\begin{align}
  \gamma = \frac{1}{1+e^2 A^2},
\end{align}
where requiring $\gamma$ be real eliminates an integration constant. This
specifies our thermodynamic mass, and we determine $\Phi_1$ and $\lambda_\pm$
from \cref{eq:firstlawder1}:
\begin{align}
M &= \frac{m}{K(1+e^2 A^2)} \nn\\
\Phi_1 &= -\frac{meA^2}{1+e^2 A^2}\nn\\
\lambda_\pm &= \frac{r_+}{1\pm Ar_+} - \frac{m(1-e^2A^2)}{(1+e^2 A^2)^2} 
\mp \frac{e^2A}{(1+e^2 A^2)}
\label{TDparams}
\end{align}
This is a rather unusual set of relations, the off-set of the electrostatic
potential depends on mass, and the thermodynamic mass depends on charge. We view
this as a consequence of the fact that for the accelerating black hole, the
electric potential cannot be gauged away at infinity -- there is a polar
electric field at the AdS boundary, thus mass and charge are inextricably
intertwined.

Finally, we have that
\begin{align}
  \frac{m}{K} = M+e^2A^2M = M+\Phi_1Q,
\end{align}
which ensures that the Smarr relation will indeed be satisfied with these new
quantities.

For the slowly accelerating black hole, we therefore want to compare the volume
dependence on $r_+$ to the area dependence via the isoperimetric ratio,
introduced in \cref{sec:press-volume-black},
\begin{align}
\mathcal{R}=\left(\frac{3V}{\omega_{2}}\right)^\frac13
  \left(\frac{\omega_{2}}{\mathcal{A}}\right)^\frac12, 
\end{align}
where $V$ is the thermodynamic volume, $\mathcal{A}$ is the horizon area, and
$\omega_2=4\pi/K$ is the area of a unit `sphere'.  Using \cref{eq:volNR} for
$V$ and \cref{eq:TSNR} for $\mathcal{A}$, we find
\begin{align}
\mathcal{R}=\frac{1}{(1-A^2r_+^2)^{1/6}}\geqslant 1.
\end{align}
Thus these slowly accelerating black holes do indeed satisfy the
reverse isoperimetric inequality.



It is also interesting to compare these results for varying tension to some of
our early work~\cite{Appels:2016uha}, where $K$ and $\mu_\pm$ were held fixed.
With these assumptions, $eA$ and $mA$ were fixed, however, we did not alter the
thermodynamic mass from $m/K$, nor the electrostatic potential from
$\Phi_\mathrm{H}$. The two sets of results are consistent, since, as we have
already pointed out,
\begin{align*}
\Phi_1 Q = \frac{m e^2 A^2}{K(1+e^2 A^2)} 
= \frac{m}{K} \left [ 1 - \frac{1}{1+e^2 A^2} \right] = \frac{m}{K} - M.
\end{align*}
The correction to the electrostatic potential therefore balances the shift in
thermodynamic mass in both the Smarr formula, and indeed the first law with the
assumptions made in~\cite{Appels:2016uha} since $eA$ was required to be
fixed. However, it is worth revisiting these assumptions in the light of our
work here on varying tension.

First, notice that our charged C-metric has parameters: $m$, relating to the
mass of the black hole, $e$ to its charge, $A$ to its acceleration, and $K$,
that relates to an overall conical deficit. $K$ is the one parameter that has no
immediately obvious physical interpretation, indeed seems more like a coordinate
choice, thus fixing $K$ was natural. However, now armed with our better
understanding of the metric and its thermodynamics, we see that in fixing the
tensions of the deficits, we are fixing two physical quantities, thus we should
only find that {\it two} combinations of the solution parameters are
fixed. Therefore, we should not fix $K$ a priori, but instead just the
combinations of parameters that fix the tensions:
\begin{align}
2(\mu_++\mu_-) &= 1 - \frac{1+e^2A^2}{K} \nn\\
\mu_+ - \mu_- &= - \frac{mA}{K}
\end{align}
From these expressions, we see that if charge vanishes, then indeed fixing
tensions fixes $K$ and the combination $mA$, but if charge does not vanish, then
we can no longer conclude that $\delta K=0$. Instead
\begin{align}
  \frac{\delta K}{K} = \frac{\delta (mA)}{mA} = 2\frac{eA\delta (eA)}{1+e^2A^2}
\end{align}
i.e.\ we have {\it two} constraints on the variation of our parameters.  Thus,
for example if we throw a small mass $m_0$ into the black hole, we expect
$\delta M = m_0$, $\delta Q=\delta P = 0$. Using the expression for $M$ and the
tensions we then find:
\begin{align}
\frac{\delta K}{K^2} &= - 2 \frac{e^2 A^2}{m}\delta M &
\delta A &= - (1-e^2 A^2) \frac{AK}{m} \delta M\nn\\
\delta m &= (1- 3 e^2 A^2) K \delta M&
\delta e &= -2\frac{e^3A^2K}{m} \delta M
\end{align}
indicating that the acceleration of the black hole drops, as expected.

\section{Critical behaviour of accelerating black holes}

Given that we are working in anti-de Sitter spacetime, we can ask whether there
is something analogous to a Hawking-Page phase transition \cite{Hawking:1982dh}
for our accelerating black holes, although it is difficult to see how one could
actually have a phase transition between a system with a conical deficit along
one polar axis only, and a presumably totally regular radiation bath.  However,
recall that a black hole in AdS behaves similarly to a black hole in a
reflecting box, with the negative curvature of the AdS providing the qualitative
reflection. For small black holes, the effect of the negative curvature is
sub-dominant to the local curvature of the black hole, and the black hole has
negative specific heat, as in the vacuum Schwarzschild case. For black holes
larger than the AdS radius, the vacuum curvature dominates, and the black hole
has positive specific heat, in particular, there is a minimum temperature for a
black hole in AdS, below this temperature, only a radiation bath can be a
solution to the Einstein equations at finite $T$. Plotting the Gibbs free energy
as a function of temperature shows both the allowed states, as well as the
preferred one for a given temperature.  At very low $T$, the only allowed state
is a radiation bath. Above a critical temperature $T_c = \sqrt{3}/2\pi\ell$, one
can have either a radiation bath, or a black hole (that may be either `small' or
`large'). However for $T>1/\pi\ell$, the large black hole is not only
thermodynamically stable (in the sense of positive specific heat) but
thermodynamically preferred, and a radiation bath will spontaneously transition
into a large black hole.

First consider the situation where our accelerating black hole is
uncharged.\footnote{ Note: in all explicit examples and figures in this section
  we take the $\theta=0$ axis to be regular ($\mu_+=0$). This is for simplicity,
  including a nonzero north pole tension does not alter the essential physics of
  what we present here.}  Fixing the tension of the string, we can plot the
temperature of our black hole as a function of its mass, $M$, as shown in figure
\ref{fig:TvMnocharge}.  This figure shows how increasing acceleration actually
makes a black hole of given mass {\it more} thermodynamically stable in the
sense of positive specific heat. Figure \ref{fig:TvMnocharge} also shows the
corresponding Gibbs free energy, indicating the would-be Hawking-Page transition
occurs at lower temperatures as acceleration increases.
\begin{figure}
  \centering
  \begin{subfigure}[b]{0.45\textwidth}
    \includegraphics[width=\textwidth]{TvMnocharge.pdf}
    \caption{\label{fig:TvMnochargea}}
  \end{subfigure}\quad
  \begin{subfigure}[b]{0.45\textwidth}
    \includegraphics[width=\textwidth]{GvMnocharge.pdf}
    \caption{\label{fig:TvMnocharge}}
  \end{subfigure}
  \caption{\label{fig:TvMnocharge}(a) A plot of temperature as a
    function of mass (in units of $\ell$) for the uncharged black hole.  The
    slowly accelerating r\'egime is shown as a solid line, and the inferred
    local horizon temperature for $A\ell>1$ is shown dashed. Note how for larger
    string tension (hence greater acceleration) the region of positive specific
    heat increases. (b) A similar plot, but now showing the Gibbs free
    energy as a function of temperature.}
\end{figure}

At first sight, this is rather curious, as a naive examination of the uncharged
C-metric shows that the Newtonian potential, $f(r)$ has the cosmological
constant ameliorated by the acceleration:
$f(r) = r^2 (1 / \ell^2 - A^2) \simeq r^2/\ell_{\rm eff}^2$. Given that one
often imagines that it is the black hole radius relative to the confining `box'
of AdS that is causing the thermodynamic stability of the large black holes,
this looks rather confusing: increasing acceleration appears to counteract the
AdS length scale. However, this intuition is too naive: the relevant effect is
the spacetime curvature in the vicinity of the horizon, and whether the black
hole or the cosmological constant is dominant (larger black holes having smaller
tidal forces). Computing the Kretschmann scalar at the event horizon
demonstrates that indeed, increasing acceleration for a given mass lowers the
local tidal forces due to the black hole.  In fact, it is easy to compute the
``Hawking-Page'' transition temperature, assuming the radiation bath to have
zero Gibbs energy from the expressions for $TS$ and $M$ in terms of $r_+$, $A$
and $\ell$. A brief calculation gives
\begin{align}
T_{HP} (r_+, \ell, A) &= \frac{1}{4\pi r_+ }\left [
\frac{ 3r_+^2}{\ell^2 ( 1- A^2 r_+^2) } + 1 \right ]\nn\\
& \simeq \frac{1}{2\pi \ell} \left ( 1 - \frac32 A^2 \ell^2 + {\cal{O}} (A^4 \ell^4) \right)
\;.
\end{align}
While the acceleration parameter $A$ is not a thermodynamic charge, instead
being related to the tension via $M$, nonetheless, the general picture is that
increasing tension increases acceleration, thereby decreasing the temperature at
which the ``Hawking-Page'' transition occurs.

Now consider adding a charge to the black hole, for which we might now expect a
richer phase structure, possibly with critical phenomena analogous to the
isolated charged AdS black hole
\cite{Chamblin:1999tk,Cvetic:1999ne,Chamblin:1999hg}. The critical phenomena
occur due to the three possible phases of black hole behaviour for varying mass.
In the presence of charge, there is now a lower limit on the mass parameter of
the black hole, set by the extremal limit where the temperature vanishes.
Increasing the mass of the black hole moves it away from extremality, thus
increasing temperature, rendering the specific heat positive near this lower
limit. For large mass black holes, we are also in a positive specific heat
r\'egime where the local vacuum curvature is dominant in the near horizon
geometry.  Depending on the size of the charge relative to the vacuum energy,
there can be an additional negative specific heat r\'egime where the black hole
is small enough that its local curvature is dominant, but is far enough from
extremality that the usual Schwarzschild negative specific heat type of
behaviour pervades.  Given that for uncharged accelerating black holes,
increasing tension lowers the critical temperature at which the transition to
positive specific heat occurs, we expect this `swallowtail' behaviour to be
mitigated for charged accelerating black holes in the canonical ensemble, and
indeed this is what is observed.
\begin{figure}
  \centering
  \includegraphics[width=0.6\textwidth]{TvMcharge.pdf}
  \caption{\label{fig:TvMcharge}A plot of temperature as a function of mass for
    the charged black hole, with fixed $Q=0.05\ell$, and varying tension as
    labelled. As before, the slowly accelerating r\'egime is shown as a solid
    line, and $A>A_\mathrm{crit}$ is shown dashed.}
\end{figure}

We first explore the accelerating black hole in the canonical ensemble, i.e.\
where the charge, $Q$, of the black hole is fixed, but we allow $M$ and $\mu_-$
to vary.  In \cref{fig:TvMcharge}, we give a representative plot of
temperature as a function of black hole mass for $Q=0.05\ell$ to illustrate how
increasing tension gradually removes the negative specific heat phase of the
black hole.
\begin{figure}
    \centering
  \begin{subfigure}[b]{0.45\textwidth}
    \includegraphics[width=\textwidth]{GvTvarymu.pdf}
    \caption{\label{fig:GvTchargea}}
  \end{subfigure}\quad
  \begin{subfigure}[b]{0.45\textwidth}
    \includegraphics[width=\textwidth]{GvTvaryQ.pdf}
    \caption{\label{fig:GvTchargeb}}
  \end{subfigure}
  \caption{\label{fig:GvTcharge}A plot of the free energy as a function of
    temperature for varying tension with $Q=0.05\ell$ on the left, and varying
    charge with $4\mu_-=0.3$ on the right.}
\end{figure}
\Cref{fig:GvTcharge} shows the variation of the free energy $F=M-TS$ with
temperature for varying tension and charge. As tension is increased, the
swallowtail becomes smaller, and eventually disappears, analogous to the
situation where the charge is gradually increased, shown on the right in
\cref{fig:GvTcharge}. The free energy plot tells us that at low temperatures, we
have the near extremal black hole, however as the mass of the black hole
increases there is a critical value at which there is a spontaneous transition
to a larger black hole with positive specific heat. The existence of this
transition relies on the presence of the intermediate region of negative
specific heat for the charged black hole.  For large enough tension (or charge
relative to $\ell$), there is a critical point at which this intermediate
r\'egime disappears, and the phase transition along with it. \Cref{fig:coexist}
shows the ``van der Waals'' like behaviour of this coexistence curve for varying
tension (in analogy to the varying potential plots of \cite{Chamblin:1999tk}),
and cosmological constant (in analogy to \cite{Kubiznak:2012wp}).
\begin{figure}
  \centering
  \begin{subfigure}[b]{0.45\textwidth}
    \includegraphics[width=\textwidth]{coexistvarymu.pdf}
    \caption{\label{fig:coexista}}
  \end{subfigure}\quad
  \begin{subfigure}[b]{0.45\textwidth}
    \includegraphics[width=\textwidth]{coexistvaryell.pdf}
    \caption{\label{fig:coexistb}}
  \end{subfigure}
  \caption{\label{fig:coexist}The coexistence line for the charged black hole
    shown for varying tension and cosmological constant with the black hole
    charge is fixed at $Q=0.05$. (a) $\ell=1$, and the value of tension
    at the critical point is $\mu_c = 0.219$. (b) $4\mu_- = 0.3$, and
    the critical value of the AdS radius is $\ell_c = 0.36$.}
\end{figure}

Finally, for completeness we consider the thermodynamics of the accelerating
charged black hole in the grand canonical ensemble, where we now allow charge to
vary. The Gibbs potential is now $G=M-TS-Q\Phi$, with
\begin{equation}
\Phi = \Phi_\mathrm{H} + \Phi_1
= \frac{e}{r_+} - \frac{meA^2}{1+e^2A^2} 
\label{grandcanphi}
\end{equation}
kept fixed. The interesting feature of fixed potential, as noted in
\cite{Chamblin:1999tk} for an isolated RNAdS black hole, is that there is a
critical value of $\Phi$ delineating two qualitatively different behaviours of
the black hole.  For small fixed potentials, the charged AdS black hole can
never approach extremality. This can be seen by noting that $f=f'=0$ at
extremality, where $f(r)$ is the RNAdS black hole potential. Solving these
algebraic equations, and substituting $\Phi_\mathrm{RN} = e/r_+$, one finds the
constraint $3r_+^2 /\ell^2= \Phi_\mathrm{RN}^2-1$, thus for
$|\Phi_\mathrm{RN}|<1$ there is no possibility of extremality. In our case, for
the charged accelerating black hole, the algebraic relations for extremality at
fixed potential are considerably more complicated partly due to the extra
acceleration parameter, but mostly because of the complicated expression for
$\Phi$~\eqref{grandcanphi}. However, the same principle applies, and we also
observe a similar phase transition from small to large $\Phi$, where the
critical value of $\Phi$ is now tension dependent. \Cref{fig:grand}
demonstrates this behaviour showing the analogous plot to \cite{Chamblin:1999tk}
with acceleration for fixed $\mu_-$, and also how the behaviour depends on
$\mu_-$ at fixed $\Phi$, illustrating how increasing $\mu_-$ improves the
thermodynamic viability of the black hole.  \Cref{fig:phicrit} shows how
the critical value of the potential, where only positive specific heat black
holes are allowed, varies with tension.
\begin{figure}
  \centering
  \begin{subfigure}[b]{0.45\textwidth}
    \includegraphics[width=\textwidth]{GvTgrand.pdf}
    \caption{\label{fig:granda}}
  \end{subfigure}\quad
  \begin{subfigure}[b]{0.45\textwidth}
    \includegraphics[width=\textwidth]{GvTgrandmu.pdf}
    \caption{\label{fig:grandb}}
  \end{subfigure}
  \caption{\label{fig:grand}The Gibbs potential in the grand canonical ensemble
    as a function of temperature, on the left with $4\mu_- = 0.3$ for varying
    potential as labelled, and on the right with $\Phi=0.9$ and varying tension
    as labelled in the plot.}
\end{figure}

\begin{figure} 
  \centering
  \includegraphics[width=0.6\textwidth]{phicritofmu.pdf}
  \caption{\label{fig:phicrit}A plot of the critical value of $\Phi_c(\mu_-)$ at
    which a black hole is always preferred for all temperatures as a function of
    the tension.}
\end{figure}


\section{Critical black holes with teardrop-shaped horizons}
\label{chap:crit-bh}

An interesting feature of the ultra-spinning black hole lies in its
thermodynamic description. As alluded to in the introduction to this chapter,
this spacetime seems to have more entropy than it should by violating the
reverse isoperimetric inequality, which caps entropy based on the thermodynamic
volume of a black hole spacetime. 

In exploring possible black hole solutions in four-dimensional Fayet-Iliopoulos
gauged supergravities, Gnecchi et al.~\cite{Gnecchi:2013mja} briefly presented a
noncompact black hole horizon with a finite area. It was later clarified in a
letter by Klemm~\cite{Klemm:2014rda} that this solution can be interpreted as
the \emph{ultra-spinning} limit of the Kerr-AdS solution, where the rotation
parameter is taken to be critically large. This limit only becomes sensible if
one admits the existence of conical defects running along the main axis of
revolution, which in turn become maximal in this limit. The result is a horizon
which could be described as roughly spherical near its equator, with sharp
conical deficits at each pole that extrude to the boundary.

From here, in a series of papers, Hennigar \emph{et
  al.}~\cite{Hennigar:2015gan,Hennigar:2014cfa,Hennigar:2015cja}, explored the
thermodynamic implications of having such an extraordinary spacetime. If the
nondiverging area of a noncompact horizon was the initial ``first of its
kind'' for this solution, these papers established the second such instance. The
\emph{reverse isoperimetric conjecture}~\cite{Dolan:2013ft,Cvetic:2010jb}
establishes an upper bound on the entropy for any black hole given its
thermodynamic volume, which is reached only for spherical black hole
solutions. The ultraspinning black hole was the first solution found to violate
this conjecture, leading the authors to impose more stringent conditions under
which the bound might be valid.

In this chapter, we seek to determine the uniqueness of this latter discovery. A
curious feature of the ultraspinning spacetime is that it is seemingly isolated
from regularly-spinning black holes by any physical process. It is interesting
therefore to ponder whether it truly is a special case, or whether this
violation is present in further extensions of this solution. One way in which
the set of black hole solutions can be extended beyond the usual generalisations
to charged and/or rotating solutions is to consider acceleration.

The solution that describes the accelerated black hole is, as we have seen
earlier, known as the C-metric. It is similar in form to Kerr-AdS, but is
differentiated by a nonremovable conical defect and a boundary offset from the
usual $r\to\infty$, if one treats $r$ as a generic radial coordinate. The
characteristic feature of the ultra-spinning black hole is the pair of maximal
deficits at each pole. The accelerated solution has by default one deficit
greater than the other, which means that we may only have one such maximal
defect. The term ``ultra-spinning'' to designate this class of solutions in the
context of acceleration is misleading, for unlike inertial black holes, this
state may be reached by maximising---more appropriately, as will be explained in
further sections, extremising---not only rotation but either acceleration or
even charge as well. The term \emph{critical}, for lack of an original word will
therefore be used to designate any black hole solution which exhibits a single
(or a pair of) $2\pi$-deficit(s).

In the context of thermodynamics, we observe these accelerated critical
solutions to behave differently to the original ultra-spinning black hole. The
thermodynamic properties of Kerr-AdS solutions have been known for a
while~\cite{Gibbons:2004ai, Silva:2002jq, Caldarelli:1999xj, Hawking:1998kw,
  Kostelecky:1995ei, Hawking:1982dh} and in the ultra-spinning limit must be
redefined to avoid diverging quantities. On the contrary, we find that when
accelerated black holes are critical, no such redefinition is required. In the
previous chapter, we showed, with appropriate thermodynamic relations adapted to
accelerated black hole spacetimes~\cite{Appels:2016uha,Appels:2017xoe}, that the
reverse isoperimetric inequality is satisfied for the C-metric.

We will begin by reviewing the ultra-spinning solution
of~\cite{Caldarelli:1999xj,Klemm:2014rda,Hennigar:2014cfa}, leading us to
exploring how this critical limit might be applied to the rotating C-metric. We
will then review the thermodynamics of the ultra-spinning solution, focussing on
how it violates the isoperimetric conjecture. We then examine what this critical
limit, when applied to accelerating black holes, implies, both in the
nonrotating case and in the general case including angular momentum.

% \renewcommand{\labelitemii}{$\to$}
% \begin{itemize}
% \item Exotic limits, Klemm
% \begin{itemize}
% \item noncompact feature
% \item $2\pi$-deficit
% \item new take with $K$
% \end{itemize}
% \item Kubiznak et al. ultra-spinning \& super-entropic interpretation, reverse isoperimetric inequality
% \begin{itemize}
% \item generic or extremely special?
% \end{itemize}
% \item C-metric --- $2\pi$-deficits but things still well behaved, can have noncritical deficit at other pole.
% \begin{itemize}
% \item re-visit super-entropicity in context of acceleration.
% \end{itemize}
% \item Outline structure of paper.
% \end{itemize}

\subsection{Review of the ultra-spinning black hole}

We begin by considering the rotating black hole in asymptotically AdS space,
described by the Kerr-AdS metric. In Boyer-Lindquist coordinates, it is, 
\begin{align} \label{eq:kerrAdS}
ds^2 = -\frac{f(r)}{\Sigma}\Big[dt-a\sin^2\theta \frac{d\phi}{K} \Big]^2 +
  \frac{\Sigma}{f(r)}dr^2 + \frac{\Sigma r^2}{g(\theta)}d\theta^2 +
  \frac{g(\theta) \sin^2\theta}{\Sigma r^2} \Big[adt-(r^2+a^2)
  \frac{d\phi}{K}\Big]^2, 
\end{align}
where
\begin{align}
f(r)&= 1-\frac{2 m}{r}+\frac{a^2}{r^2}+\frac{r^2+a^2}{\ell^2},\nn\\
g(\theta)&=1 - \frac{a^2}{\ell^2} \cos^2\theta, \nn\\
\Sigma(r, \theta) &= 1 + \frac{a^2}{r^2} \cos^2\theta,
\end{align}
and $K$ is a parameter which we choose to leave unspecified. The parameters $m$
and $a$ correspond to the mass and rotation of the spacetime and
$\ell = \sqrt{-3/\Lambda}$ is the AdS length scale. In \cref{chap:BHTD}, we
introduced the Kerr-AdS metric with $K = \Xi = 1-a^2/\ell^2$. This actually ensures
the poles are regular; indeed, one may check that with $K$ undefined, this
spacetime has a conical deficit given by
\begin{align}\label{eq:deficit-al}
\delta = 2\pi \bigg[1-\frac{1-a^{2}/\ell^{2}}{K}\bigg].
\end{align}

The ultra-spinning limit is obtained by taking the limit in which
$a\to\ell$. From the expression above, it is clear that in this limit the
deficit along the $\theta = 0$ and $\theta=\pi$ axes is maximal ($2\pi$). The
$\phi\phi$-component of the traditional metric, with regular poles, diverges,
and the workaround presented in~\cite{Hennigar:2014cfa} amounts to having
$K\neq \Xi$. Part of the reasoning behind naming this spacetime as ultra-spinning
is that the angular velocity evaluted on the boundary of the spacetime also
diverges, despite the adjusted metric. \Cref{fig:embed3Dus-noA-tilted} shows an
embedding of this spacetime.


\begin{figure}
  \centering
  \includegraphics[width=0.2\textwidth]{embed3Dus-noA-tilted}
  \caption{\label{fig:embed3Dus-noA-tilted}A $\theta$--$\phi$ slice of the
    ultra-spinning black hole spacetime for $r=r_+$, the outer horizon.}
\end{figure}

\subsection{The rotating C-metric and its critical limit}
We now revisit accelerating black holes and the C-metric. As in the previous
chapter, we will rewrite the generalised C-metric explicitly for convenience
here. We have
\begin{align}
  ds^2 &= \frac{1}{\Omega^2}\bigg\{ -\frac{f(r)}{\Sigma}\Big[dt - a\sin^2\theta
         \frac{d\phi}{K} \Big]^2 + \frac{\Sigma}{f(r)}dr^2 + \frac{\Sigma
         r^2}{g(\theta)}d\theta^2 \nn\\
  &\hspace{15em} + \frac{g(\theta) \sin^2\theta}{\Sigma r^2} \Big[adt-(r^2+a^2)
    \frac{d\phi}{K}\Big]^2\bigg\},\nn\\ 
  F&=dB,\qquad B=-\frac{e}{\Sigma r}\Big[dt-a\sin^2\theta \frac{d\phi}{K}\Big].
\label{eq:cmetric4}
\end{align}
where
\begin{align}
f(r)&=(1-A^2r^2)\bigg[1-\frac{2m}{r}+\frac{a^2+e^2}{r^2}\bigg]+\frac{r^2+a^2}{\ell^2},\nn\\
g(\theta) &=
1+2mA\cos\theta+\bigg[A^2(a^2+e^2)-\frac{a^2}{\ell^2}\bigg]\cos^2\theta,\nn\\ 
\Sigma&=1+\frac{a^2}{r^2}\cos^2\theta, \qquad \Omega=1+Ar\cos\theta.
\label{eq:cmetric4-fun}
\end{align}
Parametric restrictions exist for this solution and were given in
\cref{sec:cmet-review}, along with the following explicit expressions for the
string tensions:
\begin{align}
  \label{eq:tensions}
  \mu_\pm = \frac{\delta_\pm}{8\pi}=\frac14\bigg[1-\frac{g(\theta_\pm)}{K}\bigg] =
  \frac14\bigg[1-\frac{1\pm 2mA + A^2(a^2+e^2)-a^2/\ell^2}{K}\bigg]. 
\end{align}

% \subsection{The critical limit for the generalised C-metric}
% \label{sec:critC}

The C-metric provides a mechanism through which we can construct a black hole
with strings of unequal tension at either pole, through various choices of the
parameters $A$ and $K$. The term critical, in this section, is used to describe
a black hole where at least one of the tensions is maximal, as in the
ultra-spinning black hole. This occurs when the deficit is taken to its upper
limit, $2\pi$. While for Kerr-AdS, this corresponds to an upper bound on
rotation, for the generalised C-metric, it actually corresponds to a set of
bounds, upper and sometimes lower, for the parameters of not only rotation, but
charge and acceleration too.

The critical limit for the C-metric is defined as the parametric limit required
for $\delta_- \to 2\pi$, since $\delta_-\geqslant \delta_+$. We read off, from
our definition of the conical deficits~\eqref{eq:tensions}, that this occurs
when $g(\theta_-)=0$. We have already determined this in
\cref{sec:rotating-c-metric}, when investigating parametric restrictions on this
metric. The case we are interested in actually corresponds, as long as $mA < 1$
to the solid blue lines in \cref{fig:param-regions}, described by the relation
\begin{align}\label{eq:us}
a^2 = \ell^2 \frac{1-2mA+e^2A^2}{1-A^2\ell^2},
\end{align}
the condition for criticality.


\begin{figure}
\centering
\begin{subfigure}[b]{0.2\textwidth}
\includegraphics[width=\textwidth]{embed3Dus-smooth}
\caption{\label{fig:embed3Dus-smooth}}
\end{subfigure}\qquad
\begin{subfigure}[b]{0.2\textwidth}
\includegraphics[width=\textwidth]{embed3Dus-Kbig}
\caption{\label{fig:embed3Dus-Kbig}}
\end{subfigure}\qquad
\begin{subfigure}[b]{0.2\textwidth}
\includegraphics[width=\textwidth]{embed3Dus-Kbigger}
\caption{\label{fig:embed3Dus-Kbigger}}
\end{subfigure}
\caption{\label{fig:embed3Dus}Embeddings in $\mathbb{E}^3$ of the ultra-spinning
  C-metric for $m=9\ell$, $A=0.04\ell$ and (a) $K=L_+$, (b) $K=1.2L_+$, (c)
  $K=2L_+$.} 
\end{figure}

This relation, combined with \cref{eq:conditions} provides more stringent
conditions in parametric space for which critical C-metrics exist. Indeed,
requiring that the right-hand side of \cref{eq:us} be positive allows us to draw
the following criteria. We have two possibilities, either
\begin{align}\label{eq:case1}
A<\frac{1}{\ell},\quad \mbox{and}\quad 2mA<\min \{1+e^2A^2,2\},
\end{align}
or
\begin{align}\label{eq:case2}
A>\frac{1}{\ell}, \quad e^2A^2<1, \quad \mbox{and} \quad 1+e^2A^2<2mA<2.
\end{align}

For this limit to be sensible physically, we must ensure that the singularity
remain shielded by an event horizon. While for the general C-metric we are
limited to numerical techniques in determining when a horizon is formed --- this
was displayed in \cref{fig:param-regions} --- we are able to analytically
determine expressions for the horizons in the absence of charge\footnote{Again,
  we do not expect charge to dramatically alter the observations laid out in
  this section, however it does mute our ability to perform this analysys
  analytically}.

We are interested in determining the relationship between the parameters $A$,
$\ell$ and $m$ (with $a$ given by \cref{eq:us}) for which the function $f(r)$
exhibits a double root. The adjacent sections in parameter space will then
correspond to a naked singularity and one which has (at least) a horizon. Let
$r_\dr$ denote the location of the extremal horizon. We then use $f(r_\dr)=0$ to
find $m=m(r_\dr)$ when the spacetime is extremal, and use it to factorise
\begin{align}
f(r)\big|_{m=m(r_\dr)}=\frac{r-r_\dr}{r^2}\overline{f}(r).
\end{align}
Since $r_\dr$ is a double root of $f(r)$, we also have $\overline{f}(r_\dr)=0$,
which yields a constraint on the parameters,
\begin{align}
 (1+Ar_\dr)\left[\Big(A^{2}-\frac{1}{\ell^2}\Big)r_\dr^2-1\right]\left[\Big(A^2-\frac{1}{\ell^2}\Big)r_\dr^3-\Big(A^2-\frac{1}{\ell^2}\Big)r_\dr^2+3Ar_\dr-1\right]=0.
\end{align}
This has a couple of possible solutions. The first two factors give
\begin{align}
r_\dr &=- \frac{1}{A} \qquad m = \frac{1}{2A^3\ell^2},\nn\\
r_\dr &= \pm\frac{\ell}{\sqrt{A^2\ell^2-1}}\qquad m =0
\end{align}
with the corresponding value of $m$ also given. The solutions to the cubic
equation in the third factor may be parametrised using hyperbolic or
trigonometric functions. To do so we treat $0\leqslant A\ell\leqslant 1$ and
$A\ell>1$ separately as follows:

\begin{itemize}
\item for $0\leqslant A\ell \leqslant 1$, we write $A\ell=\sin 3\chi$. The three
  real solutions are then
\begin{align}
r_\dr = \frac{\ell}{\tan\bar{\chi}\cos3\bar{\chi}}, \qquad m =
  \frac{\ell}{2\sin^3\bar{\chi}}\frac{1-4\sin^2\bar{\chi}}{8\sin^2\bar{\chi}-5},
\end{align}
where $\bar{\chi} = \chi + 2\pi n/3$, and $n \in \{-1,0,1\}$.
\item for $A\ell > 1$, we write $A\ell=\cosh 3\eta$. There is one real solution
  and it reads
\begin{align}
r_\dr = \frac{\ell}{\coth\eta\sinh 3\eta}, \qquad m =
  \frac{\ell}{2\cosh^3\eta}\frac{4\cosh^2\eta-1}{8\cosh^2\eta-5}. 
\end{align}
\end{itemize}

The mapped output of these solutions is displayed in \cref{fig:horroots}, which
displays the parametric regions which result in spacetimes containing either no
horizons, an outer event horizon, an acceleration horizon, or both. More
importantly, this confirms that it is therefore indeed possible to reach this
limit while avoiding a naked singularity.
 
\begin{figure}
  \centering
  \includegraphics{mAregions}
  \caption{\label{fig:horroots}Parametric space for the accelerated ultra-spinning
    black hole. Blue hatched regions are excluded by virtue of the conditions set
    out in \cref{eq:case1} and \cref{eq:case2}. Red hatched regions correspond to
    naked singularities and are therefore also excluded. The remaining space is
    separated into two regions, one which describes slowly accelerating black holes
    without an acceleration horizon (lighter shade), and another for spacetimes with
    an acceleration horizon (darker shade).}
\end{figure}

\subsection{Thermodynamics of the critical C-metric}

\subsubsection{The super-entropic black hole}
\label{sec:uskerr}



Let us consider now the $a\to\ell$ limit. As mentioned above, this causes the
original, smooth, Kerr-AdS metric to diverge, and if precautionary measures are
taken to allow for conical defects, it has the effect of freezing out the
tension and decoupling $K$ as a physical parameter. This can be seen from
\cref{eq:deficit-al}: in this limit, $K$ is no longer linked to the deficit, and
therefore to the tension either. Now that we also have the thermodynamic
quantities, we see that some of these also diverge, through their dependence on
the function $\gamma\sim L^{-1}$. In particular, the angular velocity of the
boundary diverges hence the `ultra-spinning' label attributed to this
limit. That $\gamma$ diverges poses the main problem for resolving the first
law.

This decoupling of $K$ from $\mu$, or rather that combined with the fact that
$\mu$ is now seemingly constant does however hold interesting implications for
the primitive first law in \cref{eq:master-al}, which, itself, does not
diverge. The $\delta \mu$ piece obviously vanishes, and we are free to set
$\delta K=0$ too, since it is purely a gauge choice at this point, it no longer
has physical significance. This leaves a functioning first law for the
ultra-spinning black hole, with thermodynamic potentials, given (with $a=\ell$)
in \cref{eq:SandT-al,eq:horpotentials-al}, distinct from the regular Kerr-AdS
black hole. These can be related to those
in~\cite{Hennigar:2014cfa,Hennigar:2015cja} by recognising that
$K=2\pi/\bar{\mu}$ (where the bar allows for distinction from what we define as
tension, and $\bar{\mu}$ is the periodicity of their redefined azimuthal
coordinate).

Having obtained expressions for the volume and horizon area of the black hole,
we may now discuss the reverse isoperimetric conjecture. We find that
\begin{align} \mathcal{R} =
\sqrt[6~]{\frac{r_{+}^{2}}{r_{+}^{2}+\ell^{2}}}\leqslant 1,
\end{align} and the conjecture is violated. The existence of a super-entropic
black hole would naively imply the existence of other near-ultra-spinning
super-entropic black holes however the physical discontinuity between this black
hole and regular Kerr-AdS spacetimes prevents this train of
logic. \todoopt{Implications for conjecture validity?} 

\subsubsection{The critical nonrotating C-metric}
\label{sec:critnrC}

We return now to the nonrotating C-metric discussed in the previous chapter. We
explored some of the conditions one must impose unto this solution in
\cref{sec:parameters}. In particular, we have the condition
\eqref{eq:conditionsNR}:
\begin{align}
  \label{eq:conditionsNR-2}
  e^2A^2 >
  \begin{cases}
    2mA - 1 \qquad &\mbox{if} \quad mA \leqslant 1,\\
    m^2A^2 \qquad &\mbox{if} \quad mA > 1,
  \end{cases}
\end{align}
which ensures that the metric signature is preserved for the entire range of
$\theta\in [0,\pi]$. The conical deficits along each axis $\theta_\pm = 0,\pi$
were also given \eqref{eq:deficitsNR} as
\begin{align}
  \label{eq:deficitsNR-2}
  \delta_\pm=2\pi\bigg[1-\frac{g(\theta_\pm)}{K}\bigg] = 2\pi\bigg[1-\frac{1\pm
  2mA + e^2A^2}{K}\bigg],
\end{align}
and we see that a maximal deficit is obtained along the $\theta = \pi$ axis
provided that $1-2mA + e^2 A^2 = 0$. This also prevents a maximal deficit from
occurring at $\theta = 0$ as $\delta_+<\delta_-$. This corresponds to saturating
the bound \eqref{eq:conditionsNR-2} for $mA \leqslant 1$. Saturating the bound
for $mA > 1$ is a different type of limit in which a double root is introduced
for $g(\theta)$ over $0<\theta\pi$, resulting in two separate positive regions
over this domain, all the while preserving the signature. Once again, we see
that the $2\pi$-deficit is clearly indpendant of $K$.

This limit has already been mentioned in
\cite{Chen:2015vma,Hubeny:2009ru}. Unlike \cref{sec:uskerr} however, we still
have the possibility of a defect at $\theta = 0$. The tension of the string
running from the north pole is $\mu_+ = (K-2)/4K$, and K does in fact still have
a physical role in the spacetime. All this means is that this fact, combined
with the absence of any divergences other than those at the horizons, implies
that we do not need to treat this spacetime any different to its noncritical
counterpart.

Using the thermodynamic quantities we established in~\cref{sec:cmet-TD-der}, and
turning off the electric charge, which doesn't add much qualitatively for now,
we know therefore that the first law of black hole thermodynamics,
\begin{align}\label{eq:firstlawnr}
  \delta M = T\delta S + V\delta P - \lambda_+ d\mu_+ - \lambda_- d\mu_-,
\end{align}
is satisfied with the following quantities,
\begin{gather}
  M=\frac{m}{K},\qquad T=\frac{1}{2\pi
    r_+^2}\bigg[m(1-A^2 r_+^2) + \frac{r_+^3}{\ell^2(1-A^2 r_+^2)}\bigg], \qquad
  S=\frac{\mathcal{A}}{4} = \frac{\pi r_+^2}{K(1- A^2 r_+^2)},\nn\\ 
  V=\frac{4 \pi  r_+^3}{3 K (1- A^2 r_+^2)^2}, \qquad P=\frac{3}{8\pi \ell^2},
  \qquad \lambda_\pm = \frac{r_+}{1\pm A r_+} - m,
\end{gather}

From here we can safely take the critical limit $2mA = 1$. The last term in
\cref{eq:firstlawnr} vanishes and none of the quantities above diverge. Finally,
seeing as $\mu_+$ is still related to $K$ and we therefore need not treat this
particular limit any different, we expect the isoperimetric
conjecture to be obeyed. The isoperimetric ratio is
\begin{align}
  \mathcal{R} =\frac{1}{\sqrt[6~]{1-A^2r_+^2}} =
  \frac{1}{\sqrt[6~]{1-r_+^2/4m^2}}\geqslant 1. 
\end{align}

Therefore, despite the noncompact horizon, its entropy does fall below the
bound imposed unto it by the isoperimetric inequality. 

\subsection{Thermodynamics for the rotating C-metric}

The method we have used so far to determine thermodynamic quantities has been to
assume the first law be upheld, introduce correction terms to potentials for
which this would not be unreasonable and then solve the ensuing equations. The
aim of this chapter has always been to examine the thermodynamics of the
critical rotating and accelerating black hole. To do so, however, requires prior
knowledge as to the thermodynamics of the rotating C-metric. Unfortunately, the
thermodynamics of this spacetime are not well understood, despite recent
attempts~\cite{Astorino:2016ybm}. We will, therefore, approach this geometry
following a similar procedure to that used in \cref{sec:cmet-TD-der,sec:uskerr}
and attempt to make sufficient headway so as to draw certain conclusions
concerning the critical limit.

In what follows, we will attempt to derive an expression for the mass of the
black hole as well as other thermodynamic quantities by demanding that the first
law,
\begin{align}
  \label{eq:fullFLT}
  \delta M = T\delta S + \Omega \delta J + \Phi \delta Q + V \delta P  - \lambda_+
  d\mu_+ - \lambda_- d\mu_-,
\end{align}
hold. To do so in an appropriate fashion, we will impose that entropy be a quarter of
the outer horizon area, and that the temperature be that obtained by
regularising the euclidean form of this metric. To clarify, we write out all
expressions explicitly, 
\begin{gather}
  T= \frac{f^\prime (r_+)}{4\pi} = \frac{1}{2\pi(r_+^2+a^2)}\left(m(1+A^2r_+^2)
    - \frac{a^2+e^2}{r_+}+r_+^3\Big(\frac{1}{\ell^2}-A^2\Big)\right), \nn\\ 
  S = \frac{\mathcal{A}}{4} = \frac{\pi(r_+^2+a^2)}{K(1-A^2 r_+^2)}.
  \label{eq:TS}
\end{gather}
We will also preserve the following forms for the charge $Q$, angular momentum
$J$, pressure $P$,
\begin{gather}
  \label{eq:QJP} Q=\frac{e}{K}, \qquad J = \frac{m a}{K^2}, \qquad P =
  -\frac{\Lambda}{8\pi} = \frac{3}{8\pi \ell^2},
\end{gather}
and complete the set of extensive variables by including the tensions
$\mu_\pm=\delta_\pm/8\pi$ as defined by \cref{eq:tensions}. We will be
decomposing the angular velocity $\Omega$, electrical potential $\Phi$ and
thermodynamic volume $V$ into two terms
\begin{align}
  \label{eq:X}
  X=X_0+X_1 
\end{align}
in order to separate reasonably\footnote{By reasonably, we are referring to
  conventional (see, for example~\cite{Gibbons:2004ai}) ways of deriving the
  quantities in question, but applying them to the current geometry. These are
  all well-motivated for more standard black holes and therefore are
  \emph{reasonable} choices.} well-defined quantities, denoted by $X_0$, and
correction terms, $X_1$ whose existence is required to satisfy the first law,
and whose explicit form left to be determined in what follows. We have, to
begin with~\cite{Caldarelli:1999xj}
\begin{gather}\label{eq:OP0}
\Omega_0 = -\frac{g_{t\phi}}{g_{\phi\phi}}\bigg|_{r=r_+} = \frac{a
  K}{r_+^2+a^2}, \qquad \Phi_0 = \frac{e r_+^2}{r_+^2+a^2}. 
\end{gather}
Thermodynamic volume is usually determined through the Smarr/scaling Euler
relation. We will take $V_0$ to be that which satisfies the reduced Smarr
relation given as
\begin{align}\label{eq:Smarr0}
\frac{m}{K} = 2(TS-PV_0+\Omega_0 J)+\Phi_0 Q.
\end{align}
A straightforward re-arrangement of \cref{eq:TS} reveals
\begin{align}\label{eq:V0}
V_0 = \frac{4\pi r_+ (r_+^2+a^2)}{3 K (1-A^2 r_+^2)^2}.
\end{align}
Finally, we introduce a function $\gamma(A,a,e,l)$ which we will be using to
determine an explicit form for the mass,
\begin{align}\label{eq:defM}
M=\frac{m}{K} \gamma(A,a,e,l),
\end{align}
as well as explicit expressions for the correction terms introduced above.

To establish the first law, one usually begins by considering perturbations of
$f(r_+)=0$. The result can usually simply then be massaged directly into the
first law itself. Proceeding this way, making use of all the definitions given
above, the closest one can get to an expression which resembles the first law is
the full form of the expression we have been using throughout the previous
sections as the initial starting point of this derivation. In fact, the
procedure remains much the same, however we will proceed in full, as this is a
crucial part in this write-up. One eventually obtains
\begin{align}\label{eq:mKFLT}
\delta \frac{m}{K} = T\delta S + \Omega_0 \delta J + \Phi_0 \delta Q + V_0
  \delta P - \frac{r_+}{1+Ar_+} \delta \mu_+ - \frac{r_+}{1-Ar_+}\delta \mu_-
  +\frac{m\delta K}{2K^2}. 
\end{align}
One then combines \cref{eq:fullFLT} with \cref{eq:mKFLT} from which it follows that
\begin{align}
\delta M = \delta \frac{m}{K} + \Omega_1 \delta J + \Phi_1 \delta Q + V_1 \delta
  P - \left(\lambda_\pm - \frac{r_+}{1\pm Ar_-}\right) \delta \mu_\pm -
  \frac{m}{2K^2}\delta K. 
\end{align}
We will be comparing this to what a variation of $M$ as defined in
\cref{eq:defM} yields. In order to be able to do so term-by-term, we must ensure
that all the variations are independent. In particular, using
\cref{eq:tensions}, as well as variations of \cref{eq:QJP}, we can re-express
$\delta K$ and, for good measure, $\delta A$ as
\begin{align}
\delta K &=\frac{2K^2}{m\Delta}(1+a^2A^2+\Delta) \delta \frac{m}{K} \nn -
           \frac{2aK}{m\Delta}\Big(A^2-\frac{1}{\ell^2}\Big)\delta
           J-\frac{2eA^2K^2}{\Delta}\delta Q-\frac{2a^2K}{\ell^3\Delta}\delta
           \ell\\ 
&\qquad\qquad\qquad -\frac{2K^2}{m\Delta}\big(m-A(a^2+e^2)\big)\delta\mu_+
                     -\frac{2K^2}{m\Delta}\big(m+A(a^2+e^2)\big)\delta\mu_-,
                     \nn\\ 
\delta A &=-\frac{A K}{m}\delta\frac{m}{K}-\frac{K}{m}\delta\mu_+ +
           \frac{K}{m}\delta \mu_-, \qquad \qquad \Delta =
           -1+A^2(a^2+e^2)-\frac{a^2}{\ell^2}. 
\label{eq:dKdA}
\end{align}
It immediately follows that
\begin{multline}
\delta M = \delta\frac{m}{K} \Big(1-\frac{1+a^2A^2+\Delta}{\Delta}\Big) 
+\delta J \bigg(\Omega_1+\frac{aK}{\Delta}\Big(A^2-\frac{1}{\ell^2}\Big)\bigg) 
+\delta Q \bigg(\Phi_1+\frac{meA^2}{\Delta}\bigg) \\
+\delta\ell \Big(-\frac{3}{4\pi\ell^3}+\frac{ma^2}{K\ell^3\Delta}\Big) 
+\delta \mu_+\Big(\frac{r_+}{1+Ar_+}-\lambda_+
+\frac{1}{\Delta}\big(m-A(a^2+e^2)\big)\Big)\\ 
+\delta \mu_-\Big(\frac{r_+}{1-Ar_+}-\lambda_+
+\frac{1}{\Delta}\big(m+A(a^2+e^2)\big)\Big). 
\label{eq:dM1}
\end{multline}

To perform the term-by-term comparison, we must write out perturbations of $M$
in terms of the same quantities as those above. For this we can use variations
$Q$ and $J$ (\cref{eq:QJP}) as well as those for $K$ and $A$ outlined above in
\cref{eq:dKdA} to eventually obtain 
\begin{align}
\delta M = &~\delta\frac{m}{K}\bigg((\gamma-A\gamma_{A}+a\gamma_{a}\Big(\frac{2}{\Delta}(1+a^2A^2+\Delta)-1\Big)+\frac{2e}{\Delta}\gamma_{e}(1+a^2A^2+\Delta)\bigg) \nn\\
&+K\delta J\bigg(\gamma_{a}\Big(1-\frac{2a^2}{\Delta}\Big(A^2-\frac{1}{\ell^2}\Big)\Big)-\frac{2ea}{\Delta}\gamma_{e}\Big(A^2-\frac{1}{\ell^2}\Big)\bigg) \nn\\
&+m\delta Q\bigg(-\frac{2eaA^2}{\Delta}\gamma_{a}+\gamma_{e}\Big(1-\frac{2e^2A^2}{\Delta}\Big)\bigg) \nn\\
&+\frac{m}{K}\delta\ell\Big(\gamma_{\ell}-\frac{2a^3}{\ell^3\Delta}\gamma_{a}-\frac{2ea^2}{\ell^3\Delta}\gamma_{e}\Big) \nn\\
&+\delta \mu_+\Big( -\gamma_{A}-\frac{2a}{\Delta}\big(m-A(a^2+e^2)\big)-\frac{2e}{\Delta}\gamma_{e}\big(m-A(a^2+e^2)\big)\Big) \nn\\
&+\delta \mu_-\Big( \gamma_{A}-\frac{2a}{\Delta}\big(m+A(a^2+e^2)\big)-\frac{2e}{\Delta}\gamma_{e}\big(m+A(a^2+e^2)\big)\Big).
\label{eq:dM2}
\end{align}

Now that we have obtained two expressions for the variation of $M$, one from its
explicit definition, and one through its relation with the first law, we can
require these be equal to determine what functional form $\gamma$ must have to
obtain a set of thermodynamic quantities for this geometry that obey the first
law.  Doing so yields a set of differential equations which can be separated out
into five equations that determine correction terms from $\gamma$
\begin{align}
\Omega_1 &= K \gamma_{a} \bigg(\frac{2 a^2}{\Delta} \Big(\frac{1}{\ell^2}-A^2\Big)+1\bigg) + \frac{2 a e K}{\Delta}\gamma_{e} \Big(\frac{1}{\ell^2}-A^2\Big)+\frac{a K}{\Delta} \Big(\frac{1}{\ell^2}-A^2 \Big),\nn\\
\Phi_1 &= \frac{m}{\Delta}\Big(-2 e a A^2\gamma_{a}+(\Delta-2 e^2 A^2)\gamma_{e}-eA^2\Big) ,\nn\\
V_1 &= \frac{4 \pi  m }{3 \Delta K}(-\Delta l^3 \gamma_{\ell}+2 a^3 \gamma_{a}+2 a^2 e \gamma_{e}+a^2),\nn\\
\lambda_\pm &= \frac{r_+}{1\pm Ar_+}\pm \gamma_{A} + \frac{1}{\Delta}(2a\gamma_{a} + 2e \gamma_{e} + 1)\big(m\mp A(a^2+e^2)\big) ,
\label{eq:corrTermsWithGamma}
\end{align}
and a differential equation for $\gamma$, 
\begin{align}\label{eq:diffg1}
\Delta(\gamma-A\gamma_{A})+2e\gamma_{e}(1+a^2A^2+\Delta))+a\gamma_{a}(2+2a^2A^2+\Delta)+1+a^2A^2=0.
\end{align}
We also need these quantities to satisfy the Smarr relation
\begin{align}
M=2(TS-PV+\Omega J) + \Phi Q.
\end{align}
We can use \cref{eq:QJP,eq:X,eq:OP0,eq:Smarr0,eq:V0,eq:defM} together to reduce
the Smarr relation above into another differential equation for $\gamma$, 
\begin{align}\label{eq:diffg2}
\Delta(\gamma - \ell\gamma_{\ell}) + e\gamma_{e}(2+2a^2A^2+\Delta)+2a\gamma_{a}(1+a^2A^2)+1+a^2A^2 = 0.
\end{align}

With these two differential equations, we can discuss solving them. Subtracting
one from the other provides a first hint at the form of $\gamma$, which is so
familiar it could have been guessed at. We infer 
\begin{align}
a\gamma_{a} + e\gamma_{e} + \ell\gamma_{\ell} - A \gamma_{A} = 0 \qquad \Longleftrightarrow \qquad \gamma(A,a,e,\ell) = \phi\Big(\frac{a}{\ell}, \frac{e}{\ell}, A\ell\Big).
\end{align}
We re-write \cref{eq:diffg1} in terms of new unitless parameters $x=
a^2/\ell^2$, $y=e^2/\ell^2$ and $z=A^2\ell^2$, 
\begin{multline}
\phi(1+x-z(x+y))-2x \phi_{x}(1-x+z(3x+y))-4y\phi_{y}(-x+z(2x+y))\\-2z\phi_{z}(1+x-z(x+y))-1-zx=0.
\end{multline}

This equation has exact solutions for $x=0$ and $z=0$, either the nonrotating
case or the nonaccelerating case, as have been outlined in previous
sections. It is possible to solve this equation perturbatively with respect to
either variable at least up to order $x^2$ or $z^2$ however for brevity, we
shall only keep next-to-leading order terms, and set integration constants to
$0$. For example, writing $\phi = \phi^{(0)}+z\phi^{(1)}+\mathcal{O}(z^2)$, we
may use the zeroth order solution to solve the equation at first order, given by
\begin{align}
&x+\phi^{(0)}(x+y)+2x\phi^{(0)}_{x}(3x+y)+4y\phi^{(0)}_{y}(2x+y)+\phi^{(1)}(1+x)+2x\phi^{(1)}_{x}(1-x)-4xy\phi^{(1)}_{y}\nn\\
&\qquad= \frac{(1+x)(x^2+2x+y^2)}{(1-x)^2}+\phi^{(1)}(1+x)+2x\phi^{(1)}_{x}(1-x)-4xy\phi^{(1)}_{y} = 0.
\end{align}
To second order in $A$ (first order in $z$), $\gamma$ is given by
\begin{align}
  \gamma^{(A\ell\ll 1)} &= \frac{1}{1-a^2/\ell^2}\nn\\
     &\qquad -\frac{A^2\ell^2}{4} \left(\frac{1+4e^2/\ell^2+3
       a^4/\ell^4}{\left(1-a^2/\ell^2\right)^2} +\frac{\ell}{2a}
       \left(1-\frac{a^2}{\ell^2}\right) \log\frac{1-a/\ell}{1+a/\ell}\right)
       +\mathcal{O}(A^4\ell^4). 
\label{eq:gamsmallA}
\end{align}
Alternatively, it is also possible to consider perturbations around $x=0$, for
small rotation parameter. Performing much the same as above, we find
\begin{align}\label{eq:gamsmalla}
\gamma^{(a\ll \ell)} = \frac{1}{1+e^2A^2}+\frac{a^2}{4e^2}\left(\frac{1+4e^2/\ell^2+3 A^4 e^4}{\left(1+e^2A^2\right)^2}-\left(1+e^2A^2\right) \frac{\arctan(A e)}{A e }\right)+\mathcal{O}\Big(\frac{a^4}{\ell^4}\Big).
\end{align}

Obtaining these solutions means that we are now able to write down, admittedly
at low order, expressions for the correction terms and the mass in the small
acceleration limit (\cref{sec:smallAsol}) and in the small rotation limit
(\cref{sec:smallasol}) using \cref{eq:defM,eq:corrTermsWithGamma}. For
convenience, we re-write the mass
\begin{align}
M=\frac{m}{K}\gamma.
\end{align}
\todoopt[inline]{correction terms all in appendices.}

\subsubsection{Thermodynamics of the critical generalised C-metric}
Let us now return to the critical limit introduced above and examine how this
limit affects the thermodynamics of the system using the results from the
previous section. While the perturbative techniques used constrain us in
parameter space, the initial set-0 nature of the ultra-spinning case hints that
either this behaviour extends continuously to other critical limits, or it truly
is set-0 and remains disconnected from other geometries in physical parameter
space.  A characteristic feature of the ultraspinning limit was that the angular
velocity, $\Omega$, evaluated at the boundary, diverges. The angular velocity
for the general C-metric is
\begin{align}
\Omega=\text{@@@@@@@@}
\end{align}
\todoopt[inline]{Pick a formula for angular velocity - either general or with
  boundary limit taken} 

Another way to potentially draw any parallels with the nonaccelerated case is
by looking at the thermodynamic potentials. As explained above, certain
thermodynamical quantities for the KNAdS solution blow up in the ultra-spinning
limit. This is a direct consequence of the fact that $\gamma$, in the absence of
acceleration, diverges for $a\to\ell$. However, the presence of an acceleration
parameter shifts this limit to what is presented in \cref{eq:us}. $\gamma$, in
the critical limit and for small acceleration, expands to
\begin{align}
\gamma^{(A\ell\ll 1)}\Big|_{\mathrm{us}}=\frac{1}{2Am} - \frac{A\ell^4}{8m^3}\Big(1+\frac{e^2}{\ell^2}\Big)^2-\frac{A\ell^2}{4m}+\mathcal{O}(A^2\ell^2).
\end{align}
This is clearly finite for $A\neq 0$. We therefore expect similar behaviour for
the remaining potentials. The correction to the angular velocity, for example,
expands to
\begin{align}
\Omega_1^{(A\ell\ll 1)}\Big|_{\mathrm{us}} = \frac{-K}{2\ell}+\frac{K}{2mA\ell}-\frac{A K\ell^3}{8m^3}\Big(1+\frac{e^2}{\ell^2}\Big)^2-\frac{A K}{4 m\ell}(m^2-e^2)+\mathcal{O}(A^2\ell^2).
\end{align}
\section{Conclusion}

To sum up: we have shown how to allow for a varying conical deficit
in black hole spacetimes, and found the relevant thermodynamical 
variables to describe the system. The appropriate first law has a 
varying tension term, with a thermodynamic length as its potential. 
This length consists of a direct geometrical part, a mass dependent correction, 
and finally, a shift in the presence of charge. The thermodynamic phases of
accelerating black holes exhibit similar behaviour to their nonaccelerating 
AdS cousins, however, the impact of acceleration is to improve the thermodynamic
stability of the black holes.

It is interesting to note that the first law indicates that if the tension of a
defect is fixed, then there is no contribution to the variation of $M$ coming
from tension, yet, if the black hole increases its mass and hence its horizon
radius, the horizon will now have consumed a portion of the string along each
pole.  This does not appear in the thermodynamic relation. This reinforces the
interpretation of $M$ as the {\it enthalpy} of the black hole
\cite{Kastor:2009wy}.  Although the black hole increases its internal energy by
swallowing some cosmic string, it has also displaced the exact same amount of
energy from the environment, resulting in no net overall gain in the total
energy of the thermodynamic system (other than the mass that was added to the
black hole in the first place).

It would be interesting to consider whether there are any holographic
applications for these solutions. Typically, one avoids having such distorted
boundaries, however, the fact that the thermodynamics of these systems is now
well defined perhaps suggests this is worth a second look.

Finally, we have not discussed rotating accelerating black holes here. Even for
the nonaccelerating rotating black hole, the thermodynamics are more subtle,
requiring an adjustment of the angular velocity \cite{Gibbons:2004ai} in order
to define it relative to a nonrotating frame at infinity. For a conical defect
running through the black hole (though see
\cite{Gregory:2013xca,Gregory:2014uca} for a full discussion of subtleties of
replacing deficits by finite width defects) one can follow the method of
\cref{sec:singlemu} to generalise the appropriate thermodynamical variables of
Gibbons et al.\ \cite{Gibbons:2004ai} to:
\begin{equation}
  \begin{aligned}
M &= \frac{m}{KL} \;\;\;; \;\;\;\;&
V &= \frac{4\pi}{3} \left [ \frac{r_+(r_+^2+a^2)}{K} + a^2 M \right]\\
J &= \frac{am}{K^2} \;\;\;; \;\;\;\;&
\Omega &= \Omega_H-\Omega_\infty
= \frac{aK}{r_+^2+a^2} + \frac{aK}{\ell^2 L} \;\;\;; \;\;\;\;\\
Q &= \frac{e}{K} \;\;\;;\;\;\;\;& \Phi &= \frac{er_+}{r_+^2+a^2}
\end{aligned}
\end{equation}
where $K$ is the (arbitrary) parameter determining the periodicity of the
azimuthal angle as before, and $L$ is a new parameter due to black hole
rotation:
\begin{equation}
L = 1 - \frac{a^2}{\ell^2}
\end{equation}
In \cite{Gibbons:2004ai}, $K=L$ was mandated by having no conical deficit on the
rotation axis, however, here we are being more general. Allowing the tension of
any deficit to vary leads to the thermodynamic length:
\begin{equation}
\lambda = \left ( r_+ - KM \frac{1+\frac{a^2}{\ell^2}}{1-\frac{a^2}{\ell^2}} \right)
= \left ( r_+ - \frac{KM}{L}(2-L) \right)
\end{equation}
Thus our black hole obeys a standard first law, 
\begin{equation}
\delta M = T \delta S + V \delta P + \Omega \delta J + \Phi \delta Q
-2 \lambda \delta\mu
\end{equation}
as well as a conventional Smarr relation:
\begin{equation}
M
= 2TS + \Phi Q - 2 PV + 2 \Omega J
\end{equation}

Once the black hole is accelerating, it is no longer possible to have a
completely nonrotating frame at infinity due to the distortion of the
boundary. The potential therefore has a more complicated adjustment involving
not only a shift, but also a decision on which is the appropriate frame to use
at infinity. Computing the relevant parameters for the rotating accelerating
black hole is underway.


\chapter{Holographic Thermodynamics of Accelerating Black Holes}
\chaptermark{Holographic Thermodynamics}
\label{chap:holoTD}

The importance of black holes in advancing our understanding of physics cannot
be underestimated.  They provide a setting for testing our most fundamental
ideas about gravity under extreme conditions and offer us insight into the
underlying microscopic degrees of freedom that may associated with quantum
gravity.  The subject of black hole thermodynamics~\cite{Bekenstein:1973ur,
  Bekenstein:1974ax,Hawking:1974sw} has proven to be an invaluable tool to this
end, and broad classes of black holes have been shown to exhibit a rich and
varied range of thermodynamic behaviour, particularly in anti-de Sitter
spacetime~\cite{Kubiznak:2016qmn}.

Within this framework, accelerating black holes have presented a unique
challenge.  The idealized solution is described by the
\emph{C-metric}~\cite{Kinnersley:1970zw,
  Plebanski:1976gy,Dias:2002mi,Griffiths:2005qp}, whose spacetime has a
string-like singularity along one polar axis attached to the black hole.  We can
think of this conical singularity as a cosmic string (indeed, the conical
singularity can be replaced by a finite width topological
defect~\cite{Gregory:1995hd}, or a magnetic flux tube~\cite{Dowker:1993bt}) with
the tension providing the force driving the acceleration.  Surprisingly, even
though these black holes are not isolated because of the ``cosmic strings'' it
is possible to derive sensible looking thermodynamics, although recent studies
have apparently conflicting results
\cite{Appels:2016uha,Appels:2017xoe,Gregory:2017ogk,Astorino:2016ybm}.

We consider here the interpretation of an accelerating black hole in anti-de
Sitter (AdS) spacetime, with a focus on a holographic interpretation of the
thermodynamics. We resolve conflicting issues that exist in the literature,
obtain a distinct set of thermodynamic variables that are now consistent with
the gravitational action, and agree with both the conformal and holographic
methods for computing conserved charges. To this end, we focus our attention to
black holes with no acceleration horizon \cite{Podolsky:2002nk}, so that there
is no ambiguity as to which horizon temperature should be considered, or as to
whether there is an equilibrium thermodynamics for the system. In addition, as
we discuss, the holographic computation and interpretation are also unambiguous
and straightforward. We also comment on the cases when the acceleration horizons
appear and provide a novel interpretation of the boundary geometry.

An accelerating black hole in AdS can be described by the metric
\cite{Hong:2003gx}
\begin{align}\label{AdSC}
ds^2=\frac{1}{\Omega^2}\bigg[ -fdt^2+\frac{dr^2}{f}
+r^2\Big(\frac{d\theta^2}{\Sigma}
+\Sigma\sin^2\!\theta\frac{d\phi^2}{K^2}\Big)\bigg]\,,
\end{align}
where
\begin{align}
\Omega&=1+Ar\cos\theta\,, \qquad  \Sigma= 1+2mA\cos\theta\,, \nn\\
f(r)&=(1-A^2r^2)\bigg(1-\frac{2m}{r}\bigg)+\frac{r^2}{\ell^2}\,.
\end{align}
The potential $f(r)$ shows clearly the black hole nature of the solution,
as well as the effects of acceleration ($A$) and cosmological constant ($\ell$).
Note that we require $2mA<1$ to preserve the
metric signature.  The parameter $K$ encodes
information about the conical deficits on the north and south
poles that have tensions given by~\cite{Appels:2017xoe}
\begin{align}
\label{mueq}
\mu_{\pm}=\frac{\delta_\pm}{8\pi}
=\frac{1}{4}\Big(1-\frac{\Sigma(\theta_\pm)}{K}%
\Big)=\frac{1}{4}\Big(1-\frac{1\pm 2mA}{K}\Big)\,.
\end{align}
The absence of an acceleration horizon yields the constraint
$f(-1/A\cos\theta) > 0$, in turn constraining the parameter space $(m,\ell)$ to
the white region bounded by the blue and red lines in figure \ref{fig:f1}.  It
is straightforward to show via a linear transformation \cite{Hong:2003gx} on the
coordinates $(x=\cos\theta,y=-1/Ar)$ that the latter bound is equivalent to the
absence of black droplets \cite{Hubeny:2009kz}.
\begin{figure}[tbp]
\centering
%\begin{tabular}{c}
\includegraphics{regions.pdf}
%\end{tabular}
\caption{ \textbf{Parameter space.}  The blue and red lines denote the
  boundaries in the parameter space $(mA,A\ell)$ for which the holographic
  computation is valid.  The hashed red region is where acceleration horizons
  are present and the hashed blue region is where the metric signature is not
  preserved, leaving the white region as the physical parameter space.}
\label{fig:f1}
\end{figure}

As discussed in \cite{Appels:2017xoe,Gregory:2017ogk}, setting $m=0$ removes the
black hole horizon, and leaves pure AdS spacetime in Rindler-type coordinates.
Performing the coordinate transformation~\cite{Podolsky:2002nk}:
\begin{align}
1+\frac{R^2}{\ell^2}=\frac{1+(1-A^2\ell^2)r^2/\ell^2}{(1-A^2\ell^2)\Omega^2}\,, \quad
R\sin\vartheta=\frac{r\sin\theta}{\Omega}\,,
\end{align}
recovers AdS in global coordinates:
\begin{align}
\label{gAdS}
 ds^2_{AdS}= -\Big(1+\frac{R^2}{\ell^2}\Big) \alpha^2
dt^2+\frac{dR^2}{1+\frac{R^2}{\ell^2}}  +R^2\Big(d\vartheta^2+\sin^2\vartheta
\frac{d\phi^2}{K^2}\Big)\,,
\end{align}
however, note that the time coordinate is not the expected AdS time, but is
rescaled by a factor of $\alpha = \sqrt{1-A^2 \ell^2}$. Conventionally, we
choose the normalisation of our time coordinate so that it corresponds to
the ``time'' of an asymptotic observer. While this is potentially a slightly slippery
concept in AdS, taken together with the spherical asymptotic spatial coordinates,
this scaling suggests that the correct time coordinate is not in fact $t$, but rather
$\tau=\alpha t$, giving a rescaling of the time-coordinate in \eqref{AdSC}.
If we now proceed with this metric, and compute the temperature
associated with the black hole (also the temperature of the boundary
field theory), then we obtain
\begin{align}
T =\frac{f'(r_+)}{4\pi\alpha}
=\frac{1 + 3\frac{r_+^2}{\ell^2}
- A^2r_+^2 \left (2+\frac{r_+^2}{\ell^2}-A^2r_+^2\right)}
{4\pi \alpha r_+(1-A^2r_+^2)}\,,
\label{temp}
\end{align}
where $f(r_+)=0$.

It is worth pausing to reflect on this result. In past work
\cite{Appels:2016uha,Appels:2017xoe,Gregory:2017ogk},
%\cite{Hubeny:2009kz,Appels:2016uha,Appels:2017xoe,Gregory:2017ogk},
the standard time coordinate appearing in the AdS C-metric was used to derive
the temperature of the black hole horizon.
%\tcr{\bf I am not sure with \cite{Hubeny:2009kz}---they rescaled the time coordinate w.r.t. preious studies and have $T$ given by (2.7) which looks almost as %what we have...has anybody checked?}
This appeared to
be a natural approach as the blackening factor of the metric was in its
canonical form, however, as pointed out in \cite{Gibbons:2004ai}, normalising the
time and timelike Killing vector is key to obtaining the correct
thermodynamics, although the method of obtaining this correct
normalisation was less transparent. Here, having uncovered this
suggestive result, we now proceed carefully with considering
thermodynamics of the accelerating black hole.
As usual, we will take the entropy to be one quarter of the horizon area:
\begin{align}
S=\frac{\mathcal{A}}{4}=\frac{\pi r_+^2}{K(1-A^2r_+^2)}\,.
\label{entropy}
\end{align}
The remaining task is to correctly identify the black hole mass, often the
biggest challenge in studying thermodynamics of black holes with non-trivial
asymptotics. In what follows, we will provide two independent arguments, %\tcr{\bf don't we actually have 3? also the calculation of the action?}
beginning with the {Ashtekar--Das} method \cite{Ashtekar:1999jx,Das:2000cu} to metric
\eqref{AdSC}. However, although consistency
of thermodynamical relations is a common method of deriving
thermodynamics (used for {example in \cite{Astorino:2016ybm}}), we do not
consider this sufficient, hence return to our theme of holography,
computing the holographic stress tensor of the boundary
theory, thereby confirming our result.
As an ancillary argument, we finally check consistency with
a computation of the free energy.

The first argument uses the Ashtekar--Das definition of conformal
mass \cite{Ashtekar:1999jx,Das:2000cu}, which extracts the mass
via conformal regularisation of the AdS C-metric near the boundary.
The idea is to perform a conformal transformation on \eqref{AdSC},
$\bar{g}_{\mu\nu} = {\bar{\Omega}}^{2}g_{\mu\nu}$,
to remove the divergence near the boundary, then to integrate a conserved
current composed from the Weyl tensor of the conformal metric,
$\bar{C}^{\mu }{}_{\alpha \nu \beta }$, the normal
to the boundary, $N_{\mu }=\partial _{\mu }{\bar{\Omega}}$, and a
suitable Killing vector for the mass, $\xi = \partial_\tau$ to obtain a
conserved charge:\todoruth{You have an opportunity to expand here --- write down
  weyl --- explainconformal transformation, give calculation steps}
\begin{align}
Q(\xi )=\frac{\ell}{8\pi}\lim_{\bar{\Omega} \rightarrow 0}\oint
\frac{\ell^{2}}{\bar{\Omega}}N^{\alpha }N^{\beta }
\bar{C}^{\nu}{}_{\alpha \mu \beta }
\xi _{\nu }d\bar{S}^{\mu }\,.
\end{align}
Even though the conformal completion is not unique, the
charge thus obtained is independent of the choice of conformal completion.
We pick $\bar{\Omega}=\ell\Omega r^{-1} $, which provides a smooth
conformal completion in the limit $A=0$. The spacelike surface element
tangent to $ \bar{\Omega}=0$ is
\begin{align}\label{surfel}
d\bar{S}_{\mu }=\delta^{\tau }_{\mu }\frac{\ell^{2} (d\cos\theta) d\phi}{\alpha K}\,,
\end{align}%
obtained by inserting $Ar\cos\theta=-1$ into the metric  $\bar{g}_{\mu\nu}$
and computing the relevant determinant.  This yields
\begin{align}\label{confmass}
M = Q(\partial _{\tau})=\alpha \frac{m}{K}
\end{align}
for the mass, in agreement with the  temperature \eqref{temp}, but in contrast
to previous results~\cite{Appels:2016uha,Astorino:2016ybm}.
The absence of acceleration horizons ensures that  $M$ vanishes in the
limit $A\ell\to 1$ only for $m=0$ and is positive otherwise.

It is now straightforward to verify the first law and Smarr \cite{Smarr:1972kt}
relation
\begin{align}
\delta M &=T\delta S+V\delta P-\lambda _{+}\delta \mu _{+}
-\lambda_{-}\delta \mu _{-}\,,  \nn \\
M &=2TS-2PV\,,
\end{align}
using \eqref{temp}, \eqref{entropy}, and \eqref{confmass},
provided
\begin{align}\label{vol}
V &= \frac{4}{3}\frac{\pi}{K \alpha}\left[\frac{r_+^3}{(1-A^2r_+^2)^2}
+mA^2\ell^4\right]\,,\nn \\
\lambda_\pm &= \frac{1}{\alpha}\left[\frac{r_+}{1-A^2r_+^2}
-m\left(1\pm\frac{2A\ell^2}{r_+}\right)\right]\,,
\end{align}
where $P = 3/8\pi \ell^2$ is the thermodynamic pressure associated with
the cosmological constant  \cite{Kubiznak:2016qmn}, and $\lambda_\pm$
are the thermodynamic lengths introduced in \cite{Appels:2017xoe,Gregory:2017ogk}.
{Note, we have} included the possibility of the tensions varying, as otherwise
the system is constrained and identifying the correct parameters can be
misleading.

We now turn to another method for deriving the thermodynamic mass,
by computing the holographic stress tensor. This provides an alternate
and completely independent method of computation, and will reveal
the dual interpretation of this system.  The idea here is to perform a
Fefferman--Graham expansion of the metric~\cite{Fefferman:2007rka}, identifying the
fall-off of sub-leading terms in the metric at the boundary. These are
then used to compute the dual stress-energy tensor that can be integrated
to give the mass of the system.

The action, including boundary counterterms
\cite{Balasubramanian:1999re, Emparan:1999pm, Mann:1999pc}, is
\begin{align}
I[g]=&\frac{1}{16\pi}\int_{M}d^{4}x\sqrt{-g}\left[ R+\frac{6}{\ell^{2}}\right]
+ \frac{1}{8\pi}\int_{\partial M}d^{3}x\sqrt{-h}\mathcal{K}\nn \\
&- \frac{1}{8\pi}\int_{\partial M}d^{3}x\sqrt{-h}
\left[ \frac{2}{\ell}+\frac{\ell}{2}\mathcal{R}\left( h\right) \right]\,,
\label{action}
\end{align}
where $\mathcal{K}_{ab}$ is the extrinsic curvature of the boundary
metric, evaluated asymptotically in an appropriate coordinate system,
defined presently. $h_{ab}$ is the intrinsic metric on
$\partial {\cal M}$, and ${\cal R}$ its Ricci curvature. Varying
the action gives the energy momentum tensor:
\begin{align}
8\pi \mathcal{T}_{ab} = \ell \mathcal{G}_{ab}\left(h\right)
-\frac{2}{\ell}h_{ab}-\mathcal{K}_{ab}+h_{ab}\mathcal{K}\,.
\end{align}%
\todoruth{Plot T?}

To compute these terms requires new coordinates near the boundary
of AdS, typically parametrised by Fefferman--Graham
coordinates, in which
\begin{align}
ds^2= \frac{\ell^2}{\rho^2} d\rho^2 +
\rho^2 \left ( \gamma_{ab}^{(0)} + \frac1{\rho^2} \gamma^{(2)}_{ab}
+ ...\right )  dx^a dx^b\,.
\label{FGmetric}
\end{align}
Although often one identifies a $\rho$ coordinate globally, due to the
complexity of \eqref{AdSC}, we instead perform an asymptotic
expansion for the coordinate transformation, writing
\begin{align}\label{FGt}
\frac{1}{Ar}= -x-\sum F_{n}\left( x\right) \rho ^{-n}\,, \quad
\cos \theta = x+\sum G_{n}\left( x\right) \rho ^{-n}\, .
\end{align}
The functions $F_n$ and $G_n$ are fixed by the required fall-off properties
of \eqref{FGmetric}, apart from $F_1$, that we choose to write as\todoruth{more
  detail here, maybe give Fs and Gs?}
\begin{align}\label{F1}
F_{1}(x) = - \frac{\left( 1-A^{2}\ell^{2}X\right) ^{3/2}}{A\omega(x) \alpha}\,,
\end{align}
in order to elucidate the conformal degree of freedom in the boundary metric,
$\omega$, with $ X = (1-x^{2})\left( 1+2mAx\right)$.
Computing this boundary metric, $ds_{(0)}^2 = \gamma^{(0)}_{ab} dx^a dx^b$,
we find it sufficient to truncate the series \eqref{FGt} at $n=4$ and find:
\begin{align}\label{bndymet}
{ds_{(0)}^2}
= -\frac{\omega^{2}d\tau^2}{\ell^{2}}
+ \frac{X\omega^2\alpha^2 d\phi^2}{K^2(1-A^2\ell^2X)}
+\frac{\omega^2\alpha^2 dx^2}{X(1-A^2\ell^2X)^2}\,.
\end{align}
Note that the  transformation \eqref{FGt} is valid in general only when
$A^{2}\ell^{2}X<1$, but this is precisely the constraint that acceleration
horizons are absent.

The expectation value of the energy momentum of the CFT$_{3}$
can then be calculated,
yielding a relativistic fluid with a non-trivial viscous-shear tensor
\begin{align}
\left\langle \mathcal{T}_{ab}\right\rangle =\lim_{\rho \longrightarrow
\infty }\frac{\rho }{\ell}\mathcal{T}_{ab}= \frac32 \rho _{E}
U_{a}U_{b}+\frac{\rho_E}2 \ell^{2} \gamma^{(0)} _{ab}+\pi _{ab}\,,
\end{align}
with $U=\omega^{-1}\partial_\tau$, and boundary indices are
raised and lowered with $\ell^{2}\gamma^{(0)}_{ab}$.
The energy density is
\begin{align}
\rho _{E}= \frac{m}{8\pi\ell^2 \alpha^3\omega^3}(1-A^2\ell^2 X)^{3/2}
(2-3A^{2}\ell^{2}X)\,,
\end{align}
%\tcr{\bf shouldn't this be:
%\begin{equation}
%\rho _{E}= \frac{m}{\ell^2 \alpha^3\omega^3}(1-A^2\ell^2 X)^{3/2}
%(2-3A^{2}\ell^{2}X)\,,
%\end{equation}
%Where does the $1/(8\pi)$ come from?}
yielding the mass
\begin{align}
M= \int \rho _{E} \ell^3 \sqrt{\tcr{-}\gamma^{(0)}}~dxd\phi
=\frac{\alpha m}{K}\,,
\end{align}
%\tcr{\bf I agree with the arithmetic, however, isn't it strange to use the determinant of the 3-metric to calculate the 2d integral? What is the origin of $\omega$ in this integral, as written in terms of the 2-metric determinant $\sigma$?}
in agreement with \eqref{confmass}. Note that this calculation is independent
of the conformal frame (the choice of $\omega$).

The shear tensor is
\begin{align}
\pi^x_x= \frac{3mA^{2}X}{16\pi \alpha^3\omega^{3}}\left(
1-A^{2}\ell^{2}X\right) ^{\frac32}=-\pi^\phi_\phi\,,
\end{align}
with all other components  vanishing.
The equation of state is that of a thermal gas of massless particles
and the dual fluid is anisotropic,
as expected from the strongly distorted boundary.

Finally, let us return to the computation of the action \eqref{Caction}.
We find
\begin{align}
I = \frac{\beta}{2\alpha K} \left( m - 2mA^2\ell^2
- \frac{r_+^3}{\ell^2(1-A^2r_+^2)^2} \right)\,,
\label{Caction}
\end{align}
using the time coordinate $\tau$. Some simple algebra then yields the
expected result $F=I/\beta=M-TS$
for the free energy, which we plot in figure \ref{fig:FE}.
\begin{figure}[tbp]
\centering
%\begin{tabular}{c}
\includegraphics[width=0.6\textwidth]{free-energy.pdf}
%\end{tabular}
\caption{\textbf{Free energy.}
The red curve is the Schwarzschild-AdS case, illustrating the
well-known Hawking--Page transition,  situated at a temperature
given by the intersection of the red curve with $F=0$. We do
not know of any such interpretation for all other curves with $\mu\neq 0$.
The upper parts of these curves do not continue to arbitrarily large $M$
but terminate at the boundary given in figure
\ref{fig:f1}; this is visible in the above plot only for $4\mu_- =0.9$.   }
\label{fig:FE}
\end{figure}

Although similar in form, the behaviour of the free energy no longer indicates
the presence of a standard Hawking--Page transition \cite{Hawking:1982dh}.
As the string tension is fixed for the curves in the plot, no transition to
pure radiation (with zero tension) is possible. One may, however, speculate that
a transition to a different type of spacetime (for example {that of the expanding spherical wave with an attached semi-infinite string of given tension, similar to} \cite{Podolsky:2004bk})
%\sout{an Aichelburg--Sexl-like
%solution} %\cite{Aichelburg:1970dh}
%\sout{with an attached string of given tension)}
may still be possible---such an investigation, however, remains to be carried out.

We can also explore the isoperimetric ratio, or the ratio of volume to
areal radius: $\mathcal{R}=\left( \frac{3{V}}{\omega _{2}}\right)^{\frac{1}{3}}
\left(\frac{\omega _{2}}{{\cal A}}\right) ^{\frac{1}{2}}$ (recall $\omega_2=4\pi/K$
here). Using \eqref{entropy} and \eqref{vol} we find $\mathcal{R}\leq 1$,
indicating it satisfies the standard {\em reverse isoperimetric inequality}
\cite{Cvetic:2010jb}, not adding to the notable exceptions
\cite{Hennigar:2014cfa,Hennigar:2015cja,Brenna:2015pqa}.

Our full and consistent description of the thermodynamics of an
accelerating black hole reconciles discrepancies and conflicts that have
appeared in previous investigations of this system~\cite{Appels:2016uha,
Appels:2017xoe,Astorino:2016ybm}.  For example, while a
consistent set of thermodynamic variables for charged accelerating black holes
was obtained \cite{Appels:2016uha,Appels:2017xoe,Gregory:2017ogk}
the resultant free energy was not consistent with this set.
Alternate expressions for mass and temperature have been posited
\cite{Astorino:2016ybm}, with the tension of one deficit  held fixed to zero.
The other tension, while allowed to vary, was not included in the
first law, which was derived by assuming integrability of a scaling
of mass and temperature.  However no physical interpretation was given
either for this scaling or for why the energy content of the tension was
thermodynamically irrelevant.
Furthermore, the vacuum accelerating black hole
has an acceleration horizon, akin to a Rindler horizon, and the full
structure of the spacetime is that of two accelerating black holes in
two Rindler regions. Whether one should be considering a single
thermodynamic mass and first law with an additional horizon and black hole,
or whether, as suggested in \cite{Dutta:2005iy}, this should be considered
as a single system with a mass dipole is an open question.

We also found that the dual stress energy tensor for the accelerating
black hole corresponds to a relativistic fluid with a non-trivial viscous
shear tensor proportional to the acceleration parameter. Given that
the acceleration parameter also determines the conical deficit via
\eqref{mueq}, the source of this anisotropy is clearly due to the impact
of the deficit of the fluid. It would be interesting to compare this to
the weak coupling calculation of stress tensors in the presence of
conical deficits \cite{Dowker:1977zj}.
Future work will involve investigating accelerating black holes with rotation,
scalar fields~\cite{Anabalon:2009qt, Anabalon:2012ta}, and charge. The latter
system will be a challenge due to the asymptotic structure of the gauge field.

Note that our computation is independent of the conformal frame, hence
we can compare to investigations of holographic C-metrics with an
acceleration horizon. For example, by choosing $\omega^2 =
(1-A^{2}\ell^{2}X)\alpha^{-2}$, we recover the form of the boundary metric
employed in \cite{Hubeny:2009kz}, and our coordinate transformation
\eqref{FGt} is now valid throughout $x\in[-1,1]$. However, if the condition
$ A^{2}\ell^{2} X <1$ is violated, then a black droplet/black funnel is present,
and we no longer have an equilibrium temperature for the system in general.
The boundary geometry corresponds to a black hole in a spatially compact
universe, and so there is no spatial asymptotic region as pointed out
in \cite{Hubeny:2009kz}. However, with the full conformal degree of
freedom present in our expression, we can easily remedy this shortcoming
by, for example, multiplying the $\omega$ above by $\frac{1}{\sqrt{1-x}}$,
giving an $AdS_{2}\times S^{1}$ asymptotic region at $x=1$ with the $AdS_{2}$
and $S^{1}$ radius being equal. If we multiply by $\frac{1}{\sqrt{1-x^2}}$ then
there are actually two $AdS_{2}\times S^{1}$ asymptotic regions at $x=\pm1$
and $\gamma^{(0)}_{ab}$ yields the geometry of a wormhole when there are no
horizons at the boundary.
The $AdS_{2}\times S^{1}$ asymptotic geometry is supersymmetric
and to our knowledge has been unnoticed so far in the literature.

\section{Back to Basics}
\label{sec:basics}

\section{Computing the Mass}
\label{sec:mass-comput}

\subsection{The Ashtekar-Das Method}
\label{sec:ashtekar-das}

\subsection{The Holographic Approach}
\label{sec:holo}

\section{Thermodynamic Phenomena}
\label{sec:phenomena}


\chapter{Concluding Remarks}


% \appendix
%\include{appendix1}
%\include{appendix2}

% \printbibliography[heading=bibintoc]

\phantomsection 
\addcontentsline{toc}{chapter}{Bibliography}
\bibliographystyle{JHEP}
\bibliography{main.bib}

\end{document}
